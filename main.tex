\documentclass[a4paper,10pt,openright]{scrbook}

% Das "cleardoublepage=current" definiert, dass die Vakantseiten am Ende eines Kapitels
% die gleichen Kopf und Fußzeilen haben wie bisher und nicht komplett leer sind!

% Mit Fußzeile
% \usepackage[headtopline,headsepline]{scrlayer-scrpage}
% Ohne Fußzeile
\usepackage[headtopline]{scrlayer-scrpage}

\pdfminorversion=7

\RequirePackage{graphicx}
% Die zuletzte angeben Sprache ist die zu Beginn bereitgestellte Sprache. 
% Will man innerhalb des Dokumentes eine deutschsprachige Bezeichnung haben, 
% wie zum Beispiel Inhaltsverzeichnis anstellen von table of contents, 
% muss german beziehungsweise ngerman an der letzten Stelle stehen. 
% Alternativ zu Beginn des Dokumentes die Sprache auf deutsch umstellen:
% \begin{document}\selectlanguage{german}
\usepackage[ngerman,english]{babel} 
% \RequirePackage{english}
\usepackage[T1]{fontenc}
% \usepackage[latin1]{inputenc}
\usepackage[utf8]{inputenc}
\usepackage[tbtags,intlimits]{amsmath}
\usepackage{amssymb}
\usepackage{tabularx}
\usepackage{threeparttable}
\usepackage{pdflscape}   % \begin{landscape} ... \end{landscape}
\usepackage{textcomp}
\usepackage{lmodern}
\usepackage{icomma} % Das verhindert einen großen Abstand nach einem Komma in Formeln.
% Quelle: https://texfragen.de/komma_in_formel


% Use the etoolbox package to patch \thebibliography to use \addcontentsline:
% https://tex.stackexchange.com/questions/71129/bibliography-in-table-of-contents
\usepackage{etoolbox}

%% Seitenlayout
\usepackage[
    %showframe = true,
    showframe = false,
        ]{geometry}

%Das ist wichtig für die Synchronisierung der Seitennummern mit dem PDF-Viewer
\usepackage[hidelinks]{hyperref}
\usepackage{booktabs}   % Das braucht man u.a. in Tabellen für \toprule, \midrule, \bottomrule
\usepackage{colortbl}   % Das braucht man u.a. in Tabellen für \cellcolor{lightgray} 
\usepackage{tocloft}
\usepackage{paracol}
\usepackage{fancyvrb}  % Das braucht man für die Umgebung Verbatim
\usepackage{needspace}
\usepackage{longtable}
\usepackage{float}      % Für den Positionsierungsparameter H
\usepackage{verbatimbox} % Center verbatim
\usepackage{microtype}% Verbesserter Randausgleich (wichtig bei 2-Spalten-Formatierung!)
% \usepackage{pdfcolparallel} % 2-Spalten-Formatierung ermöglichen!
\usepackage{textcase}
\usepackage{calc}%
% \RequirePackage[bottom]{footmisc}
\RequirePackage{ragged2e}%
\RequirePackage[singlelinecheck=false,listof=true]{caption}%
\usepackage[leftcaption,raggedright]{sidecap}
\RequirePackage[numbers]{natbib}
% \RequirePackage{multicol}
\usepackage{multirow}
\usepackage{makeidx}
\usepackage{framed}%
\usepackage{url}%
\usepackage{setspace}

\usepackage{color}
\usepackage{out}
\usepackage{imakeidx}
\usepackage{eso-pic}
\usepackage{rotating} % Für \begin{sideways} ... 


% \setlength{\textwidth}{10.8cm}
% \setlength{\textheight}{15.7cm}

\setlength{\textheight}{\textheight+1cm}


\graphicspath{{./images/}}


% \setlength{\parskip}{0ex plus 10mm minus 0.0ex}
\let\upmu\umu
\let\nohline\relax
\sloppy

\newcommand{\colger}{%
\switchcolumn%
\selectlanguage{ngerman}%
}

\newcommand{\coleng}{%
\switchcolumn*%
\selectlanguage{english}%
}

\newcommand{\colend}{%
\end{paracol}%
}

\newcommand{\colstart}{%
\begin{paracol}{2}[]%
\selectlanguage{english}%
}

\newcolumntype{Y}{>{\centering\arraybackslash}X}%
\newcolumntype{Z}{>{\hfill\arraybackslash}X}%

%%% Trennungsregeln für das gesamte Dokument.
%
\hyphenation{
Video-streaming
Time-out
Access
Data
OpenSHMEM
PowerPC
DHCP
Sense
Spacing
Maxi-mum
Trans-mission
Unit
Unics
Inter-net-working
Infor-mation
Port-scanner
Bytes
Frames
Frame
CD/DVDs
popular
Waiting
en-co-ding
recog-ni-zable
posi-tive
Figure
address-able
hexa-decimal
corres-ponds
Avoi-dance
ack-now-ledg-ments
reques-ting
digital
address
address-ing
hier-ar-chy
repre-sen-ted
every
globally
Appli-ca-tion
Appli-ca-tions
appli-ca-tion
appli-ca-tions
commu-ni-ca-tion
speci-fies
example
Accor-ding-ly
follow-ing
monitor-ing
headers
automa-ti-cally
ack-now-ledg-ment
WiFi
Access
memory
operating
modern
offered
theory
different
logi-cally
applied
nega-tive
colla-bo-rate
necessary
However
however
original
expanding
addresses
allow
indicating
addressed
creating
several
written
according
worsens
operation
privilege
operate
strategy
execution
Cores
limited
reading
After
after
methods
}

% \selectlanguage{english}


\makeindex[name=de,title={Stichwortverzeichnis}]
\makeindex


\newcommand{\makroabschnitt}{
\Needspace{3.5cm}
\section[\englischertitel]{}
\abschnitt{\deutschertitel}
\begin{paracol}{2}[]
\section*{\englischertitel}
\switchcolumn
\section*{\deutschertitel}
\switchcolumn*
\selectlanguage{english}
}

\newcommand{\makrounterabschnitt}{
\Needspace{2.8cm}
\subsection[\englischertitel]{}
\unterabschnitt{\deutschertitel}
\begin{paracol}{2}[]
\subsection*{\englischertitel}
\switchcolumn
\subsection*{\deutschertitel}
\switchcolumn*
\selectlanguage{english}
}


\newcommand{\makrounterunterabschnitt}{
\Needspace{1.5cm}
\begin{paracol}{2}[]
\subsubsection*{\englischertitel}
\switchcolumn
\subsubsection*{\deutschertitel}
\switchcolumn*
\selectlanguage{english}
}



\newcommand{\nombreindice}{Inhaltsverzeichnis}
\newlistof{inhaltsverzeichnis}{deutschestoc}{\nombreindice}

% Die Tilde ist der Abstand zwischen Zahl und Titel im deutschen Inhaltsverzeichnis
% Das vspace fügt im deutschen Inhaltsverzeichnis einen kleinen Abstand unter jedem Kapiteleintrag ein.
% Beim englischen Inhaltsverzeichnis ist es in Kapitel-Datei im \chapter[...
\newcommand\kapitel[1]{%
\addcontentsline{deutschestoc}{chapter}{\protect{\vspace{2pt}\thechapter}~#1}}
\newcommand\abschnitt[1]{%
\addcontentsline{deutschestoc}{section}{{\thesection}~#1}}
\newcommand\unterabschnitt[1]{%
\addcontentsline{deutschestoc}{subsection}{\protect{\thesubsection}~#1}}

% Die Tilde ist der Abstand zwischen Zahl und Titel im englischen Inhaltsverzeichnis
\renewcommand{\numberline}[1]{#1~}


\definecolor{Gray}{gray}{0.9}

\newcommand{\deutschertitel}{}
\newcommand{\englischertitel}{}

% Hurenkinder und Schusterjungen verhindern
\clubpenalty10000
\widowpenalty10000
\displaywidowpenalty=10000
\interlinepenalty=1000


\pagestyle{scrheadings}
\clearpairofpagestyles
% \pagemark = Seitennummer
% \headmark = Kolumnentitel wie Chapter, Section oder Subsection

% \ohead{Kopfzeile außen}
\ohead{\pagemark}
% \chead{Kopfzeile Mitte}
\chead{}
% \ihead{Kopfzeile innen}
\ihead{\headmark}
% \ifoot{Fußzeile innen}
\ifoot{}
% \cfoot{Fußzeile Mitte}
\cfoot{}
% \ofoot{Fußzeile außen}
\ofoot{}


\begin{document}



% \newcommand{\heute}{2.~September 2019}
\newcommand{\heute}{\today}

% \AddToShipoutPictureBG{%
%   \AtPageUpperLeft{\raisebox{-1cm}{\makebox[\paperwidth]{\huge \texttt{===>} Entwurf vom \heute\ \texttt{<===} }}}%
% %   \AtPageCenter{\rotatebox{52}{\makebox[0pt]{\Huge This is diagonally across the page}}}
% \AtPageLowerLeft{\raisebox{\baselineskip}{\makebox[\paperwidth]{\huge \texttt{===>}  Entwurf vom \heute\ \texttt{<===}}}}
% }

\frontmatter 
\setcounter{page}{5}


\renewcommand{\deutschertitel}{Vorwort}
\renewcommand{\englischertitel}{Preface}

\chapter*{}

% \end{paracol}
\begin{paracol}{2}[]

{\raggedright\huge\bfseries\sffamily \englischertitel \par\ } \\[1.8ex]

\switchcolumn

{\raggedright\huge\bfseries\sffamily \deutschertitel \par\ } \\[1.8ex]

\coleng

TBD

\colger

Dieses Dokument bietet einen Einstieg in das komplexe Thema Drohnen mit künstlicher Intelligenz. Schwerpunkte sind die Entwicklung (inkl. Auswahl geeigneter Hard- und Softwarekomponenten), Bau und Betrieb von Drohnen in der Lehre und für Forschungsprojekte.  

\coleng

TBD

\colger

Beim Schreiben dieses Dokuments flossen Erkenntnisse aus dem vom Connectom Vernetzungs- und Innovationsfond des hessian.AI geförderten Forschungsprojekt \textsl{KI-gestützte Drohnenplattform} und aus der Lehrveranstaltung \textsl{Drohnen mit Künstlicher Intelligenz} an der Frankfurt University of Applied Sciences an.

\coleng

TBD

\colger

Maßgebliche Kriterien der Auswahl der in dieses Dokument vorgestellten Komponenten sind unter anderem: Anpassbarkeit an verschiedenste Einsatzszenarien, Anschaffungspreis, Robustheit, langfristige Marktverfügbarkeit sowie Qualität der Dokumentation und Herstellersupport.

\coleng

TBD

\colger

Die Realisierung einer vollständigen Abhandlung zu den Themen Drohnen und KI ist nicht Ziel dieses Dokument. Der Fokus liegt auf den Technologien und Lösungen, die während der Erstellung aktuell waren und mit denen praktische Erfahrung im Studienfeld Informatik des Fachbereich 2 an der Frankfurt University of Applied Sciences gemacht wurden.

\coleng

TBD

\colger

Über Ihre Kommentare und Verbesserungsvorschläge freuen wir uns sehr.

\colend

{\vspace{\baselineskip}}%

\textit{Christian Baun}



\cleardoublepage
\tableofcontents

\cleardoublepage
\listofinhaltsverzeichnis
\clearpage

\mainmatter 
\renewcommand{\deutschertitel}{Hardware-Komponenten zum Bau von FPV-Drohnen}
\renewcommand{\englischertitel}{Hardware Components for Building FPV Drones}

% Das vspace fügt im Inhaltsverzeichnis einen kleinen Abstand unter dem Kapiteleintrag ein.
% Beim deutschen Inhaltsverzeichnis ist es im book.tex an nur einer Stelle in der Zeile 
% \addcontentsline{deutschestoc}{chapter}{\protect{\vspace{2pt}\thechapter}~#1}}
\chapter[\protect{\vspace{2pt}\englischertitel}]{}
\kapitel{\deutschertitel}

\label{KapitelHardware}

\begin{paracol}{2}[]

{\raggedright\huge\bfseries\sffamily \englischertitel \par\ } \\[1.8ex]

\switchcolumn

{\raggedright\huge\bfseries\sffamily \deutschertitel \par\ } \\[1.8ex]

\coleng

This chapter presents the most important hardware components required to build FPV drones. It does not aim to provide a comprehensive overview of the current state of technology. Likewise, recent developments in FPV drone technology are intentionally not addressed.


\colger

Dieses Kapitel stellt die wichtigsten Hardware-Komponenten zum Bau von FPV-Drohnen vor. Es erhebt nicht den Anspruch, einen vollständigen Überblick über den Stand der Technik zu geben. Auch die Entwicklung der FPV-Drohnen in den letzten Jahren wird bewusst ausgeklammert.

\coleng

The goal of this chapter is to provide beginners with an easily understandable introduction to FPV drones. It aims to enable them to identify the components needed to design and build drones for their own projects. This forms the basis for selecting and installing the necessary software (see Chapter~\ref{KapitelSoftware}) and implementing AI functionality using additional hardware (see Chapter~\ref{KapitelKI}).

\colger

Ziel dieses Kapitels ist es, Einsteigern einen leicht verständlichen Zugang zum Thema FPV-Drohnen zu ermöglichen. Sie sollen in der Lage sein, die für ihre Projekte benötigten Komponenten zu identifizieren, um eigene Drohnen zu entwerfen und zu bauen. Dies ist die Voraussetzung für die Auswahl und Installation der notwendigen Software (siehe Kapitel~\ref{KapitelSoftware}) sowie die Realisierung KI-basierter Funktionen mit zusätzlicher Hardware (siehe Kapitel~\ref{KapitelKI}).

\coleng

The typical components of an FPV drone without AI extensions are shown in Figure~\ref{AbbildungKompoentenEinerDrohneOhneKI}.

\colger

Die typischen Komponenten einer FPV-Drohne ohne KI-Erweiterung sind in Abbildung~\ref{AbbildungKompoentenEinerDrohneOhneKI} dargestellt.

\colende

\begin{figure}[htb]
  \centering
    \includegraphics[width=\linewidth]{Komponenten_einer_FPV_Drohne_ohne_KI_v1_en.pdf}
  \caption{Components of a FPV Drone)}
  \label{AbbildungKompoentenEinerDrohneOhneKI}
\end{figure}

\renewcommand{\deutschertitel}{Rahmen}
\renewcommand{\englischertitel}{Frames}
\makroabschnitt
\label{AbschnittFrames}

\coleng

The frame connects all components of the drone. It is usually made of carbon fiber, a lightweight yet highly rigid and strong composite material. Frames made of plastic are less common. The frame determines the propeller size (see Section~\ref{AbschnittPropeller}).

\colger

Der Rahmen verbindet alle Komponenten der Drohne miteinander. Das verwendete Material ist üblicherweise Carbon, ein leichtes, aber äußerst stabiles und verwindungssteifes Verbundmaterial aus Kohlenstofffasern. Seltener kommen Rahmen aus Kunststoff zum Einsatz. Der Rahmen bestimmt die Propellergröße (siehe Abschnitt~\ref{AbschnittPropeller}).

\coleng

\coleng

The main electronic components such as the flight controller, video transmitter, receiver, and camera are mounted centrally on the frame for protection. The battery is typically placed on top of the drone to prevent damage during landings.

\colger

Die wichtigsten elektronischen Komponenten wie Flugcontroller, Videosender, Empfänger und Kamera werden zentral auf dem Rahmen montiert, um sie zu schützen. Der Akku befindet sich in den meisten Fällen oben auf der Drohne, um Beschädigungen beim Landen zu vermeiden.

\coleng

Larger frames provide more space for components and allow the use of more powerful motors (see Section~\ref{AbschnittMotoren}) and larger propellers. However, frame size also increases the overall weight. Common frame sizes and their typical applications are listed in Table~\ref{TabelleRahmen}.

\colger

Größere Rahmen bieten mehr Platz für Komponenten und ermöglichen die Nutzung leistungsstärkerer Motoren (siehe Abschnitt~\ref{AbschnittMotoren}) und größerer Propeller. Mit zunehmender Rahmengröße steigt jedoch auch das Gesamtgewicht. Gängige Rahmengrößen und deren typische Anwendungen sind in Tabelle~\ref{TabelleRahmen} aufgeführt.

\colende

\begin{table}[htb!]
\centering
\captionabove{Overview of different Frame Sizes and Drone Categories}
\label{TabelleRahmen}
\begin{tabular}{l@{\hskip 8mm}r@{\hskip 8mm}r@{\hskip 8mm}}
\toprule
Drone Category         & Typical Frame Size & Typical Propeller Diameter \\
\midrule
TinyWhoop              & 1.2 - 2.5 Inch     & 31 - 64\,mm   \\
Toothpick              & 2   - 3 Inch       & 51 - 76\,mm   \\
CineWhoop              & 2.5 - 3.5 Inch     & 64 - 90\,mm   \\
Freestyle              & 3 - 6 Inch         & 76 - 152\,mm  \\
Racing                 & 5 Inch             & 127\,mm       \\
Long-Range, Cinelifter & 4 - 10 Inch        & 100 - 254\,mm \\
Heavy-Lift             & 10 - 12 Inch       & 254 - 304\,mm \\
\bottomrule 
\end{tabular}
\end{table}

\colstart

Important distinguishing criteria when selecting an appropriate frame include the hole spacing for mounting the flight controller and the video transmitter. Common dimensions are:

\colger

Wichtige Unterscheidungskriterien bei der Auswahl des passenden Rahmens sind auch die Abstände der Bohrlöcher zur Befestigung des Flugcontrollers und des Videosenders. Gängige Maße sind:

\coleng

\begin{itemize}
\item 30.5 \(\times\) 30.5\,mm
\item 25.5 \(\times\) 25.5\,mm
\item 20 \(\times\) 20\,mm
\end{itemize}

\colger

\begin{itemize}
\item 30,5 \(\times\) 30,5\,mm
\item 25,5 \(\times\) 25,5\,mm
\item 20 \(\times\) 20\,mm
\end{itemize}

\coleng

If a frame does not provide matching mounting holes for the selected flight controller and video transmitter, a 3D-printed adapter can help — provided there is sufficient space within the frame.

\colger

Verfügt ein Rahmen nicht über passende Bohrlöcher für den ausgewählten Flugcontroller und den Videosender, kann ein per 3D-Drucker hergestellter Adapter helfen, sofern im Rahmen genügend Platz dafür vorhanden ist.

\coleng

Another distinguishing feature of frames is their geometry. The most straightforward design is the True-X shape, where all arms are of equal length. Another variant is the Squashed-X shape, in which the arms are slightly angled to improve flight characteristics. The asymmetrical Deadcat design positions the front arms farther outward and slightly backward so that the propellers do not enter the camera’s field of view. Other variants, such as the H-frame and box frame, aim to increase frame stability through additional reinforcements.

\colger

Ein weiteres Unterscheidungsmerkmal von Rahmen ist deren Geometrie. Die naheliegendste Form ist die True-X-Form, bei der alle Arme gleich lang sind. Eine weitere Variante ist die Squashed-X-Form (\textsl{gequetschtes X}), bei der die Arme leicht abgewinkelt sind, was sich positiv auf die Flugeigenschaften auswirken soll. Bei der asymmetrischen Form Deadcat sind die vorderen Arme weiter außen und nach hinten versetzt, damit die Propeller nicht in das Sichtfeld der Kamera hineinragen. Weitere Varianten sind unter anderem H-Frame und Box, die die Stabilität des Rahmens durch zusätzliche Verstrebungen verbessern sollen.

\coleng

If drones are to be flown indoors or near people, propeller guards are highly recommended. They protect not only the propellers but also the surroundings. Drones intended for racing, freestyle, or long-range applications typically have no propeller guards, as they increase the overall weight. For TinyWhoops and CineWhoops, however, propeller protection is standard.

\colger

Wenn Drohnen in Innenräumen oder in der Nähe von Personen geflogen werden sollen, ist ein Propellerschutz sehr empfehlenswert. Er schützt nicht nur die Propeller, sondern auch die Umgebung. Drohnen für Racing-, Freestyle- und Long-Range-Anwendungen haben normalerweise keinen Propellerschutz, da er das Gesamtgewicht der Drohne erhöht. Bei TinyWhoops und CineWhoops ist ein Propellerschutz hingegen Standard.

\colende

\renewcommand{\deutschertitel}{Flugcontroller und Motorsteuerung}
\renewcommand{\englischertitel}{Flight Controller and Electronic Speed Controller}
\makroabschnitt
\label{AbschnittFC}

The flight controller (FC) is the central component of every drone. It processes sensor data from the gyroscope, accelerometer, barometer, and GPS module. Using control algorithms, it stabilizes the drone’s attitude and receives pilot commands through the receiver. In addition, the FC controls the motors via the electronic speed controllers (ESC).

\colger

Der Flight Controller (FC) ist die zentrale Komponente jeder Drohne. Der FC verarbeitet die Sensordaten von Gyroskop, Beschleunigungssensor, Barometer und GPS-Modul. Er stabilisiert die Fluglage durch Regelalgorithmen und empfängt über den Empfänger die Steuerbefehle des Piloten. Zudem steuert der FC die Motoren über die Electronic Speed Controller (ESC).

\coleng

The FPV camera and video transmitter are also connected to the FC, which supplies them with power from the battery and controls their operation. The FC additionally transmits telemetry data as an on-screen display (OSD) to the video transmitter.

\colger

Auch die FPV-Kamera und der Videosender sind mit dem FC verbunden. Er versorgt sie mit Strom aus dem Akku und steuert ihre Funktionen. Zudem überträgt der FC Telemetriedaten als On-Screen-Display (OSD) an den Videosender.

\coleng

Common flight controllers use an STM32 microcontroller, which also stores the firmware. Modern MCUs include the F4, F7, and H7 series. These differ in clock speed and memory capacity (see Table~\ref{TabelleFCMCU}). The available processing power and memory limit which firmware versions can be used effectively.

\colger

Gängige Flight Controller verwenden einen STM32-Mikrocontroller, auf dem auch die Firmware gespeichert ist. Moderne MCUs sind die Serien F4, F7 und H7. Sie unterscheiden sich in ihrer Taktfrequenz und Speicherkapazität (siehe Tabelle~\ref{TabelleFCMCU}). Die vorhandenen Rechen- und Speicherressourcen begrenzen, welche Firmware-Versionen sinnvoll eingesetzt werden können.

\colende

\begin{table}[htb!]
\centering
\captionabove{Overview of modern Flight Controller STM32 MCUs}
\label{TabelleFCMCU}
\begin{tabular}{c@{\hskip 8mm}l@{\hskip 8mm}r@{\hskip 8mm}r}
\toprule
CPU (MCU)  & Clock    & Flash Memory & RAM     \\
\midrule
F405       & 168\,MHz & 1\,MB        & 192\,kB \\
F411       & 100\,MHz & 512\,kB      & 128\,kB \\
F745       & 216\,MHz & 1\,MB        & 320\,kB \\
F722       & 216\,MHz & 512\,kB      & 256\,kB \\
H743       & 480\,MHz & 2\,MB        & 1\,kB   \\
\bottomrule
\end{tabular}
\end{table}

\colstart

Betaflight (see Section~\ref{AbschnittBetaflight}) runs on all, and INAV (see Section~\ref{AbschnittINAV}) on almost all of the microcontrollers listed in Table~\ref{TabelleFCMCU}. ArduPilot (see Section~\ref{AbschnittArduPilot}) requires at least 1 MB of flash memory for limited operation. The full range of functions is available only with 2 MB of flash memory.

\colger

Betaflight (siehe Abschnitt~\ref{AbschnittBetaflight}) funktioniert auf allen, und INAV (siehe Abschnitt~\ref{AbschnittINAV}) auf fast allen in Tabelle~\ref{TabelleFCMCU} genannten Mikrocontrollern. Der Betrieb von ArduPilot (siehe Abschnitt~\ref{AbschnittArduPilot}) erfordert für einen eingeschränkten Betrieb mindestens 1 MB Flash-Speicher. Erst mit 2 MB Flash-Speicher steht der vollständige Funktionsumfang zur Verfügung.

\coleng

The electronic speed controllers (ESCs) control the motors. The ESC is one of the most heavily stressed components of a drone, as it often handles continuous currents of 10 to 20 amperes.

\colger

Die Electronic Speed Controller (ESCs) steuern die Motoren. Der ESC ist eines der am stärksten belasteten Bauteile der Drohne, da durch ihn häufig Dauerströme von 10 bis 20 Ampere fließen.

\coleng

Separate ESCs mounted as individual boards are now uncommon. In modern drones, four or more ESCs are combined on a single board—known as a 4-in-1 ESC—for motor control. The motors and the power supply connection to the battery (see Section~\ref{AbschnittAkkus}) are soldered directly to this board.

\colger

Separate ESCs als einzelne Platinen sind heute unüblich. Bei modernen Drohnen sind die vier oder mehr ESCs auf einer einzigen Platine -- einem sogenannten 4-in-1-ESC -- zur Motorsteuerung integriert. An dieser Platine werden die Motoren sowie die Verbindung zum Akku (siehe Abschnitt~\ref{AbschnittAkkus}) angelötet.

\colende

\renewcommand{\deutschertitel}{Stack oder AIO}
\renewcommand{\englischertitel}{Stack or AIO}
\makrounterabschnitt
\label{AbschnittESC}

The combination of a separate board for motor control and a flight controller is called a stack.

\colger

Die Kombination einer separaten Platine zur Motorsteuerung und dem Flugcontroller heißt Stack.

\coleng

If the ESCs are located on the same board as the flight controller, the unit is referred to as an All-In-One Flight Controller (AIO FC).

\colger

Befinden sich die ESCs auf der gleichen Platine wie der Flugcontroller, spricht man von einem All-In-One Flight Controller (AIO FC).

\coleng

In smaller drones (2 to 4 inches), AIO FCs are commonly used due to limited space. Larger drones typically use stacks. One advantage of stacks is that separate motor controllers with higher continuous current ratings are available on the market—typically between 45 and 70 A. In addition, FCs used in stacks generally offer more soldering pads, larger pads, often additional connectors, and more usable UART interfaces for sensors and actuators compared to AIO FCs.

\colger

Bei kleineren Drohnen (2 bis 4 Zoll) sind schon aus Platzgründen AIO FCs üblich. Bei größeren Drohnen kommen meist Stacks zum Einsatz. Ein Vorteil von Stacks ist, dass separate Motorsteuerungen mit höheren Dauerströmen am Markt verfügbar sind -- typischerweise zwischen 45 und 70 A. Zudem bieten die FCs von Stacks tendenziell mehr Platz für Lötpads, größere Lötpads, häufig zusätzliche Steckverbindungen und mehr nutzbare UART-Schnittstellen für Sensoren und Aktoren als AIO FCs.

\colende

\begin{figure}[htb]
  \centering
  \begin{minipage}[t]{0.48\textwidth}
    \centering
    \includegraphics[width=\linewidth]{SpeedyBee_F405_AIO_40A_Bluejay_vorderseite_crop.jpg}
    \vspace{0pt} % sorgt für Top-Ausrichtung
  \end{minipage}\hfill
  \begin{minipage}[t]{0.48\textwidth}
    \centering
    \includegraphics[width=\linewidth]{SpeedyBee_F405_AIO_40A_Bluejay_rueckseite_crop.jpg}
    \vspace{0pt}
  \end{minipage}
  \caption{Flight Controller SpeedyBee F405 AIO 40A Bluejay 25.5x25.5 (Front and Back)}
  \label{AbbildungSpeedyBeeFCBluejayAIO}
\end{figure}

\colstart

Figure~\ref{AbbildungSpeedyBeeFCBluejayAIO} shows an AIO flight controller. Figures~\ref{AbbildungSpeedyBeeFCF405MiniBLS35front} and~\ref{AbbildungSpeedyBeeFCF405MiniBLS35Aback} show the front and back sides of a stack consisting of a flight controller and a separate ESC board.

\colger

Abbildung~\ref{AbbildungSpeedyBeeFCBluejayAIO} zeigt einen AIO-Flight-Controller. Die Abbildungen~\ref{AbbildungSpeedyBeeFCF405MiniBLS35front} und~\ref{AbbildungSpeedyBeeFCF405MiniBLS35Aback} zeigen die Vorder- und Rückseite eines Stacks, bestehend aus Flight-Controller und separater ESC-Platine.

\colende

\begin{figure}[htb]
  \centering
    \includegraphics[width=\linewidth]{SpeedyBee_F405_Mini_BLS_35A_front.jpg}
  \caption{Flight Controller SpeedyBee F405 Mini 35A 20x20 Stack (Front)}
  \label{AbbildungSpeedyBeeFCF405MiniBLS35front}
\end{figure}

\begin{figure}[htb]
  \centering
    \includegraphics[width=\linewidth]{SpeedyBee_F405_Mini_BLS_35A_back.jpg}
  \caption{Flight Controller SpeedyBee F405 Mini 35A 20x20 Stack (Back)}
  \label{AbbildungSpeedyBeeFCF405MiniBLS35Aback}
\end{figure}

\colstart

An important selection criterion for flight controllers is the number of usable UART interfaces for connecting sensors and actuators. The flight controller shown in Figure~\ref{AbbildungSpeedyBeeFCBluejayAIO}, the SpeedyBee F405 AIO, has six UART interfaces. However, one is used for Wi-Fi and USB administration, and another operates only unidirectionally (simplex), meaning it can only receive data. This leaves only four fully functional (bidirectional) UART interfaces, of which three are typically occupied by the video transmitter, receiver, and GPS module. Therefore, only one free UART interface usually remains.

\colger

Ein wichtiges Auswahlkriterium für Flugcontroller ist die Anzahl der nutzbaren UART-Schnittstellen zum Anschluss von Sensoren und Aktoren. Der in Abbildung~\ref{AbbildungSpeedyBeeFCBluejayAIO} gezeigte Flugcontroller SpeedyBee F405 AIO verfügt über sechs UART-Schnittstellen. Allerdings wird eine für die WLAN- und USB-Schnittstellen zur Administration benötigt und eine weitere funktioniert nur unidirektional (Simplex), das heißt, sie kann nur Daten empfangen. Damit stehen lediglich vier vollwertige (bidirektionale) UART-Schnittstellen zur Verfügung, von denen üblicherweise drei durch die Komponenten Videosender, Empfänger und GPS-Modul belegt sind. Somit bleibt meist nur eine einzige freie UART-Schnittstelle.

\coleng

The flight controller shown in Figures~\ref{AbbildungSpeedyBeeFCF405MiniBLS35front} and \ref{AbbildungSpeedyBeeFCF405MiniBLS35Aback}, the SpeedyBee F405 Mini 35A Stack, also has six UART interfaces.  Here, too, after connecting the commonly used components — video transmitter, receiver, and GPS module — only one interface remains available. This is because one UART is used by the Bluetooth interface for administration, and another is used by the ESC for telemetry data transmission.

\colger

Auch der in den Abbildungen~\ref{AbbildungSpeedyBeeFCF405MiniBLS35front} und \ref{AbbildungSpeedyBeeFCF405MiniBLS35Aback} gezeigte Flugcontroller SpeedyBee F405 Mini 35A Stack verfügt über sechs UART-Schnittstellen. Auch hier bleibt nach dem Anschluss der üblichen Komponenten -- Videosender, Empfänger und GPS-Modul -- nur eine Schnittstelle frei verfügbar, da die Bluetooth-Schnittstelle zur Administration eine UART-Schnittstelle belegt und die Motorsteuerung eine weitere UART-Schnittstelle zur Übertragung der Telemetriedaten nutzt.

\coleng

For comparison, the flight controller shown in Figure~\ref{AbbildungFlywooFCGOKUGN745AIO}, the Flywoo GOKU GN745 AIO 45A, offers seven UART interfaces. None of them is used for USB, but one is reserved for telemetry data transmission. After connecting the common components — video transmitter, receiver, and GPS module — three UART interfaces remain available for additional sensors and actuators.

\colger

Zum Vergleich: Der in Abbildung~\ref{AbbildungFlywooFCGOKUGN745AIO} gezeigte Flugcontroller Flywoo GOKU GN745 AIO 45A verfügt über sieben UART-Schnittstellen. Keine davon ist für die USB-Schnittstelle belegt, jedoch wird eine für die Übertragung von Telemetriedaten verwendet. Nach dem Anschluss der üblichen Komponenten -- Videosender, Empfänger und GPS-Modul -- bleiben somit drei UART-Schnittstellen für weitere Sensoren und Aktoren verfügbar.

\colende


\begin{figure}[htb]
  \centering
  \begin{minipage}[t]{0.48\textwidth}
    \centering
    \includegraphics[width=\linewidth]{FLYWOO_GOKU_GN745_45A_AIO_V3_FC_front.jpg}
    \vspace{0pt} % sorgt für Top-Ausrichtung
  \end{minipage}\hfill
  \begin{minipage}[t]{0.48\textwidth}
    \centering
    \includegraphics[width=\linewidth]{FLYWOO_GOKU_GN745_45A_AIO_V3_FC_back.jpg}
    \vspace{0pt}
  \end{minipage}
  \caption{Flight Controller Flywoo GOKU GN745 AIO 45A 25.5x25.5 (Front and Back)}
  \label{AbbildungFlywooFCGOKUGN745AIO}
\end{figure}

\colstart

Table~\ref{TabelleUARTuebersicht} shows the typical use of the available UART interfaces of selected flight controllers and the resulting limitations regarding their suitability for AI projects, which often require connecting additional sensors and actuators.

\colger

Tabelle~\ref{TabelleUARTuebersicht} zeigt die typische Nutzung der verfügbaren UART-Schnittstellen einiger ausgewählter Flugcontroller und die damit einhergehenden Einschränkungen hinsichtlich ihrer Eignung für KI-Projekte, die häufig den Anschluss weiterer Sensoren und Aktoren erfordern.

\colende

\begin{table}[htb!]
\centering
\setlength{\tabcolsep}{4pt}       % Default value: 6pt
\renewcommand{\arraystretch}{1.5} % Default value: 1
\begin{threeparttable}
\captionabove{Recommended use and availability of UART Interfaces of some Flight Controllers}
\scriptsize
\label{TabelleUARTuebersicht}
\begin{tabularx}{\textwidth}{lYYYYYYYY}
\toprule
Flight Controller              & UART1 &  UART2 & UART3 & UART4 & UART5 & UART6 & UART7 & UART8  \\
\midrule
SpeedyBee F405 AIO             & \cellcolor{Gray}MSP\textsuperscript{a}  & \cellcolor{yellow}available\textsuperscript{b} & \cellcolor{Gray}VTX       & \cellcolor{green}available & \cellcolor{Gray}GPS  & \cellcolor{Gray}Receiver & --- & --- \\
SpeedyBee F405 Mini Stack      & \cellcolor{Gray}VTX   & \cellcolor{Gray}Receiver        & \cellcolor{green}available & \cellcolor{Gray}MSP\textsuperscript{a} & \cellcolor{Gray}Telem.\textsuperscript{c} & \cellcolor{Gray}GPS & --- & ---     \\
SpeedyBee F4V3 / F4V4 Stack           & \cellcolor{Gray}VTX   & \cellcolor{Gray}Receiver        & \cellcolor{green}available & \cellcolor{Gray}MSP\textsuperscript{a} & \cellcolor{Gray}Telem.\textsuperscript{c} & \cellcolor{Gray}GPS & --- & ---     \\
Flywoo F722 PRO MINI V2        & \cellcolor{yellow}available\textsuperscript{b} & \cellcolor{Gray}Receiver & \cellcolor{Gray}VTX & \cellcolor{green}available & \cellcolor{Gray}GPS & \cellcolor{Gray}Telem.\textsuperscript{c} & --- & --- \\
Flywoo F722 PRO V2              & \cellcolor{green}available & \cellcolor{Gray}Receiver & \cellcolor{Gray}VTX & \cellcolor{green}available & \cellcolor{Gray}GPS & \cellcolor{Gray}Telem.\textsuperscript{c} & --- & --- \\
 
GEPRC F722 35A AIO             & \cellcolor{Gray}VTX   & \cellcolor{Gray}Receiver  & \cellcolor{Gray}GPS & \cellcolor{green}available & \cellcolor{green}available & --- &   ---   & ---    \\
SpeedyBee F7V3 Stack           & \cellcolor{Gray}VTX   & \cellcolor{Gray}Receiver        & \cellcolor{green}available & \cellcolor{Gray}Telem.\textsuperscript{c} & --- &  \cellcolor{Gray}GPS & ---   & ---    \\
Flywoo GN745 AIO V3            & \cellcolor{green}available & \cellcolor{Gray}Receiver & \cellcolor{Gray}VTX       & \cellcolor{green}available  & \cellcolor{green}available  & \cellcolor{Gray}GPS & \cellcolor{Gray}Telem.\textsuperscript{c} & --- \\
Axisflying F745 40A AIO AM32 & \cellcolor{Gray}VTX & \cellcolor{Gray}Receiver & \cellcolor{Gray}GPS & ? & \cellcolor{green}available & \cellcolor{green}available & \cellcolor{green}available & \cellcolor{green}available \\
MicoAir H743 V2 45A AIO    & \cellcolor{green}available & \cellcolor{Gray}VTX & \cellcolor{Gray}GPS & \cellcolor{green}available & \cellcolor{green}available & \cellcolor{Gray}Receiver & \cellcolor{Gray}Telem.\textsuperscript{c} & \cellcolor{green}available \\
T-Motor PACER H743 Stack    & \cellcolor{green}available  & \cellcolor{Gray}GPS & \cellcolor{green}available & \cellcolor{Gray}Telem.\textsuperscript{c} & \cellcolor{Gray}Receiver & \cellcolor{Gray}VTX &  \cellcolor{Gray}MSP\textsuperscript{a} & \cellcolor{green}available \\
\bottomrule
\end{tabularx}
\begin{tablenotes}
\footnotesize
\item[a] MSP = Port for administration via the MultiWii Serial Protocol using Bluetooth and WIFI if available.
\item[b] Offers unidirectional (simplex) communication. It can only receive data.
\item[c] Receives telemetry data from the ESC.
\end{tablenotes}
\end{threeparttable}
\end{table}


\renewcommand{\deutschertitel}{Motoren}
\renewcommand{\englischertitel}{Motors}
\makroabschnitt
\label{AbschnittMotoren}

TBD

\colger

Motoren unterscheiden sich in ihrem Aufbau (Motoraufbau und -größe), der elektrischen Spannung (Volt), mit der sie arbeiten können, Drehzahl (KV-Wert), der Aufnahme des Propellers und der Befestigung mit der Rahmen.

\colende

\renewcommand{\deutschertitel}{Motoraufbau und -größe}
\renewcommand{\englischertitel}{Size of the Motors}
\makrounterabschnitt
\label{AbschnittMotorenSize}

TBD

\colger

Die Motoren von FPV-Drohnen sind dreiphasige, bürstenlose Gleichstrommotoren. Jeder Motor hat drei Anschlusskabel, die an die Motorsteuerung (ESC) oder direkt an an einen AIO-Flugcontroller angelötet werden. Die Motorsteuerung schickt zeitlich versetzte Wechselströme in diese drei Leitungen. Durch das Umschalten der Ströme entsteht ein rotierendes Magnetfeld, das den Rotor mit Permanentmagneten in Drehung versetzt. Die Leitungen sind gleichwertig. Es spielt also keine Rolle in welcher Reihenfolge sie an den Motorsteuerung oder den AIO-Flugcontroller angelötet sind. Die Drehrichtung des Motors kann durch vertauschen von zwei beliebigen der drei Kabel erreicht werden. Alternativ ist es auch möglich die Drehrichtung in der grafischen Oberfläche der Firmware für jeden Motor separat zu prüfen und zu ändern.

\coleng

TBD

\colger

Die Motoren bestehen aus einem Stator und einem Rotor. Der Stator enthält die Wicklungen und das Kugellager. Der Rotor ist die Motorgloke mit den Magneten, die die eigentliche Drehbewegung ausführt. Jeder Motor ist mit einer Zahlenkombination ausgezeichnet, die den Durchmesser und die Höhe des Stators in Millimeter angibt. Bei einem Motor mit der Zahlenkombination 2306 beispielsweise hat der Stators einen Durchmesser von 23\,mm und eine Höhe von 6\,mm. 

\coleng

TBD

\colger

Je größer das Statorvolumen, desto höher ist die Motorleistung im Bezug auf Drehmoment. Zudem ist die thermische Robustheit von großen Motoren besser. Motoren mit kleinem Statorvolumen sind hingegen leichter und sparsamer bei kleinen Lasten. 

\coleng

TBD

\colger

Die Propelleraufnahme (siehe Abschnitt~\ref{AbschnittMotorenPropelleraufnahme}) geschieht auf einer M5-Welle mit 5\,mm Durchmesser und einer passenden Mutter oder alternativ  mit einer Welle mit 1,5\,mm Durchmesser und zusätzlich zwei wenige M2-Schrauben am Motor befestigt. 

\colende

\renewcommand{\deutschertitel}{Elektrische Spannung}
\renewcommand{\englischertitel}{Electrical Voltage}
\makrounterabschnitt
\label{AbschnittMotorenVolt}

TBD

\colger

Motoren und Akkus (siehe Abschnitt~\ref{AbschnittAkkus}) müssen zusammenpassen. 4S-Motoren benötigen 4S-Akkus mit maximal 16,8\,V und 6S-Motoren benötigen 6S-Akkus mit maximal 25,2\,V. 

\coleng

TBD

\colger

Komponenten für 4S sind meist günstiger und leichter, bieten dafür aber weniger Leistung. Komponenten für 6S bieten mehr Leistung oder mehr Flugzeit, haben dafür aber auch meist mehr Gewicht (insbesondere die Akkus). Für leichte kostengünstige Drohnen ist 4S die bessere Wahl. Es gibt auch Motoren, die 4S und 6S vertragen und dadurch flexibler eingesetzt werden können.

\colende

\renewcommand{\deutschertitel}{KV-Wert}
\renewcommand{\englischertitel}{KV Value}
\makrounterabschnitt
\label{AbschnittMotorenKV}

TBD

\colger

Der KV-Wert beschreibt, wie schnell (Drehzahl U/min) sich ein Motor pro Volt im Leerlauf dreht. Ein Motor mit beispielsweise 3000\,KV dreht also ohne Propeller bei 1\,V Versorgungsspannung 3000\,U/min. An einem 4S-Akku (siehe Abschnitt~\ref{AbschnittAkkus}) mit \(\approx\)\,16\,V Spannung hat dieser Motor also eine Leerlaufdrehzahl von \(3000 \times 16 \approx\ 48000\)\,U/min. 

\coleng

TBD

\colger

Motoren mit hohen KV-Werten (2800-7000 Kv) sind agiler (ermöglichen schnellere Reaktion), haben ein geringeres Drehmoment pro Ampere und sind für kleinere Propeller und leichtere Drohnen gut geeignet. Motoren mit niedrigen KV-Werten (1500-2450 Kv) haben ein höheres Drehmoment pro Ampere und sind für größere Propeller und schwerere Drohnen gut geeignet. 

\coleng

TBD

\colger

Ein hoher KV-Wert bedeutet nicht, dass ein Motor stärker ist als ein Motor mit einem niedrigeren KV-Wert. Der Motor dreht nur schneller, braucht aber auch mehr Strom für Schub.

\colende

\renewcommand{\deutschertitel}{Propelleraufnahme}
\renewcommand{\englischertitel}{Propeller Mounting}
\makrounterabschnitt
\label{AbschnittMotorenPropelleraufnahme}

TBD

\colger

Größere Motoren (z.B. 22xx, 23xx, 24xx) für Drohnen ab 5\,Zoll haben zur Propelleraufnahme meistens eine M5-Welle mit 5\,mm Durchmesser. Die Propeller werden direkt auf die Motorwelle gesteckt und mit einer Mutter gesichert. Kleineren Motoren (z.B. 13xx, 14xx, 18xx) für Drohnen bis 3,5\,Zoll haben meist eine viel kleinere Wellen mit 1,5\,mm Durchmesser. Hier werden die Propeller mit zwei Schrauben am Motor befestigt. Motoren (z.B. 20xx, 21xx) für mittelgroße Drohnen (3,5 oder 4\,Zoll) gibt es für M5-Wellen und Wellen mit 1,5\,mm Durchmesser.

\colende

\renewcommand{\deutschertitel}{Rahmenbefestigung}
\renewcommand{\englischertitel}{Frame Attachment}
\makrounterabschnitt
\label{AbschnittMotorenRahmenbefestigung}

TBD

\colger


Bei der Auswahl der Motoren ist darauf zu achten, dass die Bohrlöcher des Rahmens bezüglich Anzahl, Abstand und Durchmesser übereinstimmen. Üblich ist die Befestigung der Motoren am Rahmen mit drei oder vier Schrauben (M1,4, M2 oder M3). Die Schrauben sind bei kleinen Rahmen als gleichseitiges Dreieck oder als Quadrat angeordnet. Der Lochabstand kann 7\,mm, 9\,mm, 12\,mm oder 16\,mm oder 19\,mm betragen. 

\colende

\renewcommand{\deutschertitel}{Propeller}
\renewcommand{\englischertitel}{Propeller}
\makroabschnitt
\label{AbschnittPropeller}

TBD

\colger

Die Propeller wandeln die Drehbewegung des Motors in Schub (Lift) und Steuerkraft um. Die Propellergröße (Durchmesser), Steigung (Pitch) und Blätterzahl beeinflussen Schub, Effizienz, Geräuschentwicklung und Flugverhalten. Dieses Dokument berücksichtigt nur Drohnen mit vier Propellern. Andere Konfigurationen wie Hexacopter und Octocopter sind möglich.

\colende

\renewcommand{\deutschertitel}{Propellergröße (Durchmesser)}
\renewcommand{\englischertitel}{Propeller Size (Diameter)}
\makrounterabschnitt
\label{AbschnittPropellerDurchmesser}

TBD

\colger

Die Propellergröße bestimmt, einfach gesagt, wie viel Luft bewegt wird. 

Die Angegebene Propellergröße ist immer der Durchmesser in Zoll. Die Propeller müssen zum verwendeten Rahmen (siehe Abschnitt~\ref{AbschnittFrames}) passen.  

\coleng

TBD 

\colger

Kleinere Propeller sind wendiger. Das bedeutet, dass schnellere Reaktion auf Steuerbefehle möglich sind. Die Drohne reagiert agiler. Der Grund dafür ist, dass kleinere Propeller einen geringeren Luftwiderstand haben und insgesamt leichter sind. Sie haben eine geringere Rotationsmasse (Trägheitsmoment). Dadurch kann der Motor die Drehzahl schneller erhöhen oder verringern. Da kleinere Propeller typischerweise mit kleineren Motoren mit höherem KV-Wert kombiniert sind, drehen sie zusätzlich schneller, was Die Agilität bzw. das Ansprechverhalten erneut steigert.

\coleng

TBD

\colger

Größere Propeller erzeugen mehr Schub und haben eine bessere Effizienz. Zudem machen Sie den Flug ruhiger -- auf Kosten der Agilität. Mit der Propellergröße steigt auch die Belastung für die Motoren und die Motorsteuerung (ESC), denn die größere Blattfläche erzeugt einen höheren Luftwiderstand. Zudem haben größere Propeller ein höheres Eigengewicht.

\colende

\renewcommand{\deutschertitel}{Anzahl der Blätter}
\renewcommand{\englischertitel}{Number of Blades}
\makrounterabschnitt
\label{AbschnittPropellerAnzahlBlaetter}

TBD

\colger

Es gibt Propeller mit 2 bis 8 Blättern. Je weniger Blätter ein Propeller hat, desto höher sind die erreichbare Geschwindigkeit und Effizienz (Schub pro aufgenommenem Watt oder pro Ampere). Mit steigender Anzahl an Blättern verbessern sich Kontrolle, Laufruhe und Beschleunigung und gleichzeitig auch der Stromverbrauch. 

\coleng

TBD

\colger

Für Long-Range-Anwendungen, bei denen Effizienz das Maß aller Dinge ist, kommen in der Regel 2-Blatt-Propeller zum Einsatz. Racing- und Freestyle-Drohnen sind häufig mit 3-Blatt-Propellern ausgestattet. CineWhoop verwenden häufig Propeller mit 5 oder mehr Blättern, da diese möglichst stabil in der Luft liegen und präzise steuerbar sein sollen.

\colende

\renewcommand{\deutschertitel}{Steigung (Pitch)} 
\renewcommand{\englischertitel}{Pitch}
\makrounterabschnitt
\label{AbschnittPropellerPitch}

TBD

\colger

Neben der Anzahl der Propeller beeinflusst auch die Steigung (Pitch) der Blätter die Geschwindigkeit und die Effizienz (Stromverbrauch). Ein niedriger Pitch ist durch den geringeren Luftwiederstand effizienter bei Schweben und führt zu ruhigerem Flugverhalten. Ein hoher Pitch ermöglicht eine höhere Geschwindigkeit und aggressivere Flugmanöver, steigert aber auch den Stromverbrauch. 

\coleng

TBD

\colger

Der Pitch wird in Zoll angegeben und beschreibt den Vortrieb pro Umdrehung, also wie weit sich der Propeller bei einer Umdrehung durch die Luft schrauben würde, wenn es keine Schlupfverluste gäbe.

\coleng

TBD

\colger

Bei einem Pitch von 4,3 Zoll pro Umdrehung und 3000\,U/min gibt es einen theoretischen Weg von von \(3000 \times 4,3 = 12.900\,Zoll/min\). Ein Zoll entspricht 25,4\,mm. Somit ist der theoretischen Weg pro Zeit, also die Geschwindigkeit: \(12.900\times 0,0254 \approx 327,7\,m/min\) bzw. \(327,7\,m/min / 60 \approx 5,46\,m/s\).

\coleng

TBD

\colger

Diese theoretischen Werte werden in der Realität durch Schlupfverluste um 20-30\% verringert. Der Grund dafür ist, das Luft kein festes Medium ist, sondern beweglich ist. Durch das Ausweichen und Verwirbeln, \textsl{rutscht} immer ein Teil der Luft weg und der Propeller erreicht in der Praxis weniger Vortrieb als in der Theorie. 

\colende

\renewcommand{\deutschertitel}{Akkus}
\renewcommand{\englischertitel}{Batteries}
\makroabschnitt
\label{AbschnittAkkus}

TBD

\colger

Klassischerweise verwenden FPV-Drohnen Lithium-Polymer (LiPo) oder Lithium-Ionen-Akkus (Li-Ion). Unterschieden werden die Akkus zudem hinsichtlich Kapazität (mAh), Spannung (Zellenzahl) und Entladespannung bzw. C-Wert (\textsl{Capacity Rate}), und Stecker. Eine Übersicht der charakteristischen Eigenschaften von LiPo-Akkus und Li-Ion-Akkus zeigt Tabelle~\ref{TabelleLiPoLiIonAkkus}).

\colende

\begin{table}[htb!]
\centering
\captionabove{Comparison of Lithium-Polymere (LiPo) and Lithium-Ion (Li-Ion) Characteristics}
\label{TabelleLiPoLiIonAkkus}
\setlength{\tabcolsep}{3pt} % Default value: 6pt
\renewcommand{\arraystretch}{1.0} % Default
\scriptsize
\begin{tabularx}{\textwidth}{lXX}
\toprule
Feature                         & Li-Ion & LiPo    \\
\midrule
Energy Density (Wh/kg)          & Higher                             & Lower  \\
Max. Discharge Current (C-Rate) & Lower (10-12\,C)                   & Higher (75-120\,C) \\
Weight (per stored energy)      & Lighter                            & Heavier \\   
Lifespan (number of charge cycles) & Longer                          & Shorter \\      
Safety                         & More stable, more robust            & More sensitive to over-/undercharge \\
Use Cases                      & Long-range, smooth flying, cruising & Racing, freestyle, aggressive maneuvers  \\
\bottomrule
\end{tabularx}
\end{table}

\renewcommand{\deutschertitel}{Lade- oder Entladerate (C-Wert)}
\renewcommand{\englischertitel}{Charge and Discharge Rate (C-Rate)}
\makrounterabschnitt
\label{AbschnittCWertAkkus}

TBD

\colger

Die Akkus bestehen aus einer oder mehreren in Reihe geschalteten Zellen. Der Grund dafür, dass bei FPV-Drohnen meist LiPo-Akkus verwendet werden, sind deren sehr hohe C-Werte von 75 bis 120\,C. Der C-Wert gibt an, wie schnell ein Akku entladen werden kann. Li-Ion-Akkus bieten im Gegensatz dazu nur 10-12\,C.

\coleng

TBD

\colger

Bei einem 1500\,mAh LiPo-Akku (siehe Abbildung~\ref{AbbildungLiPoAkku}) mit einem C-Wert 100 können somit \(1,5 A * 100 = 150\) Ampere dauerhaft abgegeben werden. 

\colende

\begin{figure}[htbp]
  \centering
  \begin{minipage}[t]{0.48\textwidth}
    \centering
    \includegraphics[width=\linewidth]{CNHL_1500mAh_100C_4S_XT60_Lipo_front.jpg}
    \vspace{0pt} % sorgt für Top-Ausrichtung
  \end{minipage}\hfill
  \begin{minipage}[t]{0.48\textwidth}
    \centering
    \includegraphics[width=\linewidth]{CNHL_1500mAh_100C_4S_XT60_Lipo_back.jpg}
    \vspace{0pt}
  \end{minipage}
  \caption{A 4S Lithium-Polymere (LiPo) Battery with 1500\,mAh Capacity and 100\,C Capacity Rate}
  \label{AbbildungLiPoAkku}
\end{figure}

\colstart

TBD

\colger

Für sehr rasante Flugmanöver braucht man solche hohen Ströme. Für typische Anwendungen in Lehre und Forschungsprojekten ist das aber uninteressant. Patrouillenflüge oder gar das Verharren an einer Position zur Datensammlung erfordern keine hohen C-Werte. 

\coleng

TBD

\colger

LiPo-Akkus haben mehrere Nachteile. Die werden sehr leicht tiefentladen, was sie dauerhaft beschäftigt oder zerstört. Abstürze führen auch häufig zu Beschädigungen und bei unsachgemäßer Handhabung wie dem zu schnellen Laden oder dem Betrieb trotz Beschädigungen der Außenhülle, besteht das latente Risiko von Bränden. Eine deutlich robustere Alternative zu LiPo-Akkus sind Lithium-Ionen-Akkus. Diese bieten allerdings viel geringere C-Werte.

\coleng

TBD

\colger

Li-Ion-Akkus sind ideal für Szenarien, bei denen lange Flugzeiten, ruhiges Flugverhalten und eventuell lange Strecken (Long-Range) angestrebt ist. LiPo-Akkus hingegen sind besser geeignet für rasante Flugmanöver und Rennen.

\coleng

TBD

\colger

Der C-Wert gibt auch die empfohlene Laderate an. Diese ist 1\,C, wenn der Hersteller nicht dediziert eine höhere Laderate wie z.B. 2\,C freigegeben hat. Bei dem eingangs erwähnten 1500\,mAh LiPo-Akku entspricht 1\,C Laderate einem Strom von 1,5\,A und dementsprechend dauert es eine Stunde den Akku vollständig zu laden. Das ist aber ein hypothetischer Wert, da ein derart tiefentladener Akku nicht mehr geladen werden kann.

\colende

\renewcommand{\deutschertitel}{Anzahl der Zellen (S4/S6)}
\renewcommand{\englischertitel}{Number of Cells (S4/S6)}
\makrounterabschnitt
\label{AbschnittSWertAkkusZellen}

TBD

\colger

LiPo-Akkus und Li-Ion-Akkus enthalten eine oder mehrere in Reihe geschaltete Zellen. Der sogenannte S-Wert bei Akkus definiert die Anzahl der Zellen. Jede LiPo-Zelle hat eine Nennspannung von ca. 3,7\,V. (voll geladen ca. 4,2\,V, entladen ca. 3,0\,V). Sind mehrere Zellen in Reihe geschaltet, addieren sich die Spannungen. 

\coleng

TBD

\colger

Die gängigsten Akkus sind 4S und 6S. Bei einem 4S-Akku sind 4 Zellen in Reihe geschaltet. Die Nennspannung ist \(4 \times 3,7\,V = 14,8\,V\). Ist ein solcher Akku voll geladen ist seine Spannung 16,8\,V. Eine Übersieht über die verschiedenen LiPo-Akkus enthält Tabelle~\ref{TabelleAkkus}).

\colende

\begin{table}[htb!]
\centering
\captionabove{Overview of Lithium-Polymere (LiPo) and Lithium-Ion (Li-Ion) Batteries}
\label{TabelleAkkus}
\begin{tabular}{l@{\hskip 8mm}l@{\hskip 8mm}r@{\hskip 8mm}r@{\hskip 8mm}r}
\toprule
Battery Type  & Cells & Nominal Voltage  & Fully Charged & Discharged   \\
\midrule
S1            & 1     & 3.7\,V           & 4,2,V        & 3.0\,V         \\
S2            & 2     & 7.4\,V           & 8.4,V        & 6.0\,V         \\
S3            & 3     & 11.1\,V          & 12.6,V       & 9.0\,V         \\
S4            & 4     & 14.8\,V          & 16.8,V       & 12.0\,V        \\
S5            & 5     & 18.5\,V          & 21.0,V       & 15.0\,V        \\
S6            & 6     & 22.2\,V          & 25.2,V       & 18.0\,V        \\
\bottomrule
\end{tabular}
\end{table}

\colstart

TBD

\colger

Bei einer höheren Spannung braucht ein Motor bei gleicher Leistung weniger Strom und läuft effizienter. S6-Komponenten sind leistungsstärker aber auch schwerer und teurer. Größere oder leistungsorientierte Drohnen verwenden meist 6S-Akkus. Kleinere und mittlere Drohnen verwenden meist S4-Akkus und entsprechende leichtgewichtige Komponenten.

\coleng

TBD

\colger

Mit steigender Kapazität eines Akkus steigt nicht nur die Flugdauer, sondern auch das Gesamtgewicht. Sehr kleine S4-Akkus mit 450\,mAh Kapazität wiegen ca. 60\,g. Der in Abbildung~\ref{AbbildungLiPoAkku} gezeigte Akku mit 1500\,mAh Kapazität wiegt hingegen ca. 185\,g. Ein höheres Gesamtgewicht beeinflusst auch das Flugverhalten. 


\coleng

TBD

\colger

Li-Ion-Akkus enthalten üblicherweise in Reihe geschaltete 18650-Zellen. Die Kapazität der einzelnen Zellen hängt immer vom Hersteller und der Qualität der Zellen ab. Typische Kapazitäten sind zwischen 2500 und 3500\,mAh. Ist eine höhere Kapazität gewünscht, müssen zusätzliche Zellen parallel geschaltete werden. Die Anzahl der Zellen und die Art Ihrer Verschaltung ist auf den Li-Ion-Akkus mit den Buchstaben S (Serie) und P (Parallel) angegeben. Am Markt verfügbare LI-Ion-Akkus sind häufig 3S1P, 4S1P, 4S2P, 6S1P, 6S2P.

\colende

\renewcommand{\deutschertitel}{Stecker (XT30/XT60/XT90) zum Entladen und Laden}
\renewcommand{\englischertitel}{Plugs (XT30/XT60/XT90) for Discharge and Charge}
\makrounterabschnitt
\label{AbschnittSteckerAkkus}

TBD

\colger

Der Anschluss der Akkus an den Flugcontroller geschieht in der Regel über einen XT30- oder alternativ über eine XT60-Gleichstromstecker. Es gibt auch sehr große Akkus mit XT90-Stecker. Zwei Kabel mit einer Buchse sind an die Motorsteuerung (bei einem Stack) oder an den Flugcontroller bei einem AIO angelötet (siehe Abschnitt~\ref{AbschnittFC}). 

\coleng

TBD

\colger

Bei kleinen und leichten Drohnen bis zu einer Größe von 4\,Zoll sind XT30-Buchsen üblich. 5- und 6-Zoll-Drohnen verwenden üblicherweise XT60-Buchsen. Noch größe Drohnen verwenden häufig XT90-Buchen. Je größer die Zahl, desto größere Ströme verträgt der Stecker. Größere Buchsen und Stecker sind allerdings auch größer und schwerer. Auch der Leitungsquerschnitt muss zu den Buchen passen. 

\coleng

TBD

\colger

Neben den Gleichstromstecker verfügt jeder Akku mit zwei oder mehr Zellen über einen Balancerstecker zum sicheren Laden. Dieser Stecker, der bei modernen Akkus meist der Norm JST-XHR entspricht, ermöglicht es dem Ladegerät die Zellspannung jeder einzelnen Zelle zu messen und den Ladevorgang entsprechend zu steuern. Die Anzahl der Pins entspricht der Anzahl der Zellen plus eins (wegen der gemeinsamen Masse). Ein 4S-Akku hat somit fünf Leitungen und ein 6S-Akku hat sieben Leitungen. Durch die Werte, die über den Balancerstecker ausgelesen werden, gleicht das Ladegerät die Spannung der einzelnen Zellen an. Das verhindert die Überladung einzelner Zellen beim Ladevorgang und damit das Riskio für eine Beschädigung des Akkos oder sogar einen Brand und beugt einer Tiefentladung einzelner Zellen beim späteren Betrieb vor.

\colende



\renewcommand{\deutschertitel}{GPS}
\renewcommand{\englischertitel}{GPS}
\makroabschnitt
\label{AbschnittGPS}

TBD

\colger

Ein GPS-Modul (Global Positioning System) ist obligatorisch für viele nützliche Funktionen wie zum Beispiel Return-to-Home (automatisches Heimfliegen), Geschwindigkeitsmessung, Bestimmung der Höhe, Unterstützung der Positionshaltung, autonomer Flug, etc. 

\coleng

TBD

\colger

Nicht jedes GPS-Modul ist mit allen gängigen Firmwares für Flugcontroller kompatibel. Während Betaflight quasi jedes GPS-Modul unterstützt, ist das bei INAV und ArduPilot nicht der Fall. 

\coleng

TBD

\colger

Verfügt ein GPS-Modul nur für sechs Anschlüsse (Lötpads oder Kabel), enthält es einen magnetischen Kompass. Verfügt das Modul allerdings nur über vier Anschlüsse, fehlt der magnetische Kompass. Ohne magnetischen Kompass weiß eine Drohne aber ohne Vorwärtsbewegung nicht, wie sie positioniert ist, also in welche Richtung sie schaut. Mehrere sinnvolle Funktionen wie GPS-Position-Hold bei Ardupilot sind dann nicht möglich.

\coleng

TBD

\colger

Gängige GPS-Module (siehe Abbildung~\ref{AbbildungLiPoAkku}) mit integriertem magnetischem Kompass sind in der Regel zwischen 20x20\,mm und 25x25\,mm groß. 

\colende

\begin{figure}[htbp]
  \centering
  \begin{minipage}[t]{0.48\textwidth}
    \centering
    \includegraphics[width=\linewidth]{HGLRS_GPS_M100_PRO_front.jpg}
    \vspace{0pt} % sorgt für Top-Ausrichtung
  \end{minipage}\hfill
  \begin{minipage}[t]{0.48\textwidth}
    \centering
    \includegraphics[width=\linewidth]{HGLRS_GPS_M100_PRO_back.jpg}
    \vspace{0pt}
  \end{minipage}
  \caption{HGLRC M100-5883 M10 GPS Module 21x21\,mm with built-in magnetic Compass (Front and Back)}
  \label{AbbildungLiPoAkku}
\end{figure}


\renewcommand{\deutschertitel}{Empfänger}
\renewcommand{\englischertitel}{Receiver}
\makroabschnitt
\label{AbschnittReceiver}

TBD

\colger

Der Empfänger nimmt die Steuersignale (verschiedene Kanalwerte) der Fernsteuerung entgegen und gibt Sie an den Flugcontroller weiter. Der Empfänger demoduliert die empfangenen Signale und wandelt sie in digitale Kanalwerte um. 

\coleng

TBD

\colger

Empfänger und Fernbedienung müssen zueinander kompatibel sein. Das bedeutet Sie müssen das gleiche Protokoll und den gleichen Frequenzbereich nutzen. Die gängigen Systeme nutzen die Frequenzbereiche 2.4 GHz und für Long-Range-Anwendungen 868\,MHz in der EU bzw. 915\,MHz in den USA. In Deutschland ist die Nutzung dieser Frequenzbereiche bei einer maximalen Sendeleistung von 25\,mW zulässig.

\coleng

TBD

\colger

Zwei populäre Protokolle sind das offene System ExpressLRS (ELRS) und das proprietäre TBS Crossfire. Da ELRS ein Open-Source-Protokoll ist, gibt es zahlreiche Anbieter für Hardwarekomponenten und die Weiterentwicklung ist schnell. Aus diesem Grund fokussiert dieses Dokument ganz auf den Standard ELRS. ELRS-Geräte (Sender und Empfänger) nutzen als Firmware ExpressLRS. Idealerweise verwenden Sender und Empfänger die gleichen Firmware-Version.

\coleng

TBD

\colger

Es gibt Empfänger mit einer festen Antenne aus Keramik (siehe Abbildung~\ref{AbbildungRadioMasterRP2}), die auf der Platine integriert ist, und Empfänger mit einem (siehe Abbildung~\ref{AbbildungRadioMasterRP1}) oder zwei Anschlüssen für abnehmbare Antennen. Der üblicherweise verwendete Steckverbinder ist auch hier das filigrane U.FL (\textsl{Pigtail}). Empfänger mit fester Keramikantenne sind kompakter, da kein Antennenkabel verlegt werden muss. Durch die nur wenige mm hohe Antenne reduziert sich allerdings die Reichweite signifikant.

\colende

\begin{figure}[htbp]
  \centering
  \begin{minipage}[t]{0.48\textwidth}
    \centering
    \includegraphics[width=.5\linewidth]{RadioMaster_RP2_Receiver_ELRS_front.jpg}
    \vspace{0pt} % sorgt für Top-Ausrichtung
  \end{minipage}\hfill
  \begin{minipage}[t]{0.48\textwidth}
    \centering
    \includegraphics[width=.5\linewidth]{RadioMaster_RP2_Receiver_ELRS_back.jpg}
    \vspace{0pt}
  \end{minipage}
  \caption{RadioMaster RP2 ELRS 2.4GHz Nano Receiver 13x11\,mm with integrated Antenna (Front and Back)}
  \label{AbbildungRadioMasterRP2}
\end{figure}

\begin{figure}[htbp]
  \centering
  \begin{minipage}[t]{0.48\textwidth}
    \centering
    \includegraphics[width=.5\linewidth]{RadioMaster_RP1_Receiver_ELRS_front.jpg}
    \vspace{0pt} % sorgt für Top-Ausrichtung
  \end{minipage}\hfill
  \begin{minipage}[t]{0.48\textwidth}
    \centering
    \includegraphics[width=.5\linewidth]{RadioMaster_RP1_Receiver_ELRS_back.jpg}
    \vspace{0pt}
  \end{minipage}
  \caption{RadioMaster RP1 ELRS 2.4GHz Nano Receiver 13x11\,mm with U.FL Antenna Connector (Front and Back)}
  \label{AbbildungRadioMasterRP1}
\end{figure}


\renewcommand{\deutschertitel}{Fernbedienung (Sender)}
\renewcommand{\englischertitel}{Remote Control (Sender)}
\makroabschnitt
\label{AbschnittSender}

TBD

\colger

Die verwendet Fernbedienung muss muss das gleiche Protokoll implementieren wie der verwendete Empfänger. Zudem müssen die gleichen Frequenzbereiche verwendet werden. Zu den Herstellern von zum ELRS-Protokoll kompatiblen Fernbedienungen gehören RadioMaster, Axisflying und Jumper. 

\coleng

TBD

\colger

ELRS-Fernbedienungen (Sender) nutzen als Firmware für das Funkprotokoll und das Sender-Modul ExpressLRS und als Betriebssystem EdgeTX, eine modernere Form von OpenTX. EdgeTX erzeugt sozusagen die Steuersignale und ExpressLRS überträgt sie.

\colende

\renewcommand{\deutschertitel}{Videosender}
\renewcommand{\englischertitel}{Video Transmitter}
\makroabschnitt
\label{AbschnittVTX}

TBD

\colger

Der Videosender (VTX) überträgt das Livebildes der FPV-Camera über den eingestellten Kanal im festgelegten Band an die FPV Brille. Neben analogen Videosendern existieren auch verschiedene digitale Systeme, die zueinander inkompatibel sind. Videosender werden als einzelnes Gerät oder als Kit mit Kamera und Antenne(n) verkauft. Einen analogen Videosender zeigt Abbildung~\ref{AbbildungSpeedyBeeTX800FPVVTX}.

\colende

\begin{figure}[htbp]
  \centering
  \begin{minipage}[t]{0.48\textwidth}
    \centering
    \includegraphics[width=\linewidth]{SpeedyBee_TX800_FPV_VTX_vorderseite_crop.jpg}
    \vspace{0pt} % sorgt für Top-Ausrichtung
  \end{minipage}\hfill
  \begin{minipage}[t]{0.48\textwidth}
    \centering
    \includegraphics[width=\linewidth]{SpeedyBee_TX800_FPV_VTX_rueckseite_crop.jpg}
    \vspace{0pt}
  \end{minipage}
  \caption{SpeedyBee TX800 FPV VTX Video Transmitter (Front and Back)}
  \label{AbbildungSpeedyBeeTX800FPVVTX}
\end{figure}

\colstart

TBD

\colger

In Deutschland ist die Nutzung des Frequenzbereichs 5725-5875\,Mhz bei einer maximalen Sendeleistung von 25\,mW zulässig. Die für analoge Bildübertragung zulässigen Kanäle in den jeweiligen Bändern zeigt Tabelle~\ref{TabelleFrequenzenVTX}). Auffällig sind beim Race Band die großen Kanalabstände von 37\,MHz, die Kanalüberlagerungen verhindern.

\colende

\begin{table}[htb!]
\centering
\captionabove{Overview about permitted VTX Channels in Germany}
\label{TabelleFrequenzenVTX}
\begin{tabular}{l@{\hskip 8mm}l}
\toprule
CPU (MCU)  & Channel Number (Frequency MHz)    \\
\midrule
Band A     & 7 (5745), 6 (5765), 5 (5785), 4 (5805), 3 (5825), 2 (5845), 1 (5865) \\
Band B     & 1 (5733), 2 (5752), 3 (5771), 4 (5790), 5 (5809), 6 (5828), 7 (5847), 8 (5866) \\
Band E     & none Frequency permitted in Germany \\
Band F     & 1 (5749), 2 (5760), 3 (5780), 4 (5800), 5 (5820), 6 (5840), 7 (5860) \\
Race Band  & 3 (5732), 4 (5769), 5 (5806), 6 (5843) \\
\bottomrule
\end{tabular}
\end{table}

\colstart

TBD

\colger

Der von einem Videosender zur Übertragung verwendete Kanal und das Band werden in der grafischen Oberfläche Firmware des Flugcontrollers festgelegt.

\coleng

TBD

\colger

Analoge Videosender sind herstellerübergreifend mit analogen Videobrillen kompatibel. Bei digitalen Systemen ist das nicht der Fall. Hier müssen die Hersteller von Sender und Empfänger übereinstimmen. Digitale Systeme zur Übertragung des Livebildes sind von den Herstellern DJI, HDZero und CaddxFPV unter dem Namen Walksnail verfügbar. 

\coleng

TBD

\colger

Vorteile von digitaler Livebildübertratung sind die exzellente Bildqualität und die Möglichkeit, ohne weitere Action-Camera Videos in HD-Qualität aufnehmen zu können.

\coleng

TBD

\colger

Vorteile von analoger Übertratung sind neben der bereits erwähnten herstellerübergreifenden Kompatibilität, die geringe Verzögerung (Latenz), das das Bild bei Störungen schlechter wird, aber nicht abrupt abreißt, und deutlich geringeren Anschaffungskosten für die benötigten Komponenten.

\coleng

TBD

\colger

Einen wichtigen Einfluss auf die Bildqualität und Reichweite haben die verwendeten Antennen. Der Betrieb des Videosenders ohne Antenne ist nicht empfehlenswert, da der Videosender so überhitzt und Beschädigungen wahrscheinlich sind. Der Anschluss der Antenne geschieht bei analogen Systemen üblicherweise über sehr filigrane Steckverbinder für Hochfrequenzsignale gemäß dem Standard U.FL. Diese Steckverbinder sind auch von WLAN (WiFi) bekannt und heißen umgangssprachlich \textsl{Pigtail}. Digitale Systemen (z.B. von DJI) verwenden in der Regel MMCX-Steckverbinder. Der Anschluss von Antennen mit dem größeren und robusteren SMA-Schraubgewinde ist über Adapterkabel möglich. Häufig liegen diese dem Videosender bei.

\coleng

TBD

\colger

Die Antennen sollten auf auf Sender- und Empfängerseite zusammenpassen. Die verfügbaren Antennen für Videosender unterscheiden sich anhand ihrer Polarisation. Es existieren:

\coleng
\begin{itemize}
\item linearly polarized antennas (LP)
\item circularly polarized antennas (CP)
\end{itemize}

\colger

\begin{itemize}
\item Antennen mit linearer Polarisation (LP)
\item Antennen mit zirkularer Polarisation (CP)
\end{itemize}

\coleng

TBD

\colger

LP-Antennen sind die kostengünstigste und einfachste Variante. Sie werden häufig mit dem Videosender und mit der Videobrille mitgeliefert und haben eine Stabform. 

\coleng

TBD

\colger

CP-Antennen ermöglichen eine bessere Bildqualität und eine höhere Reichweite. Bei zirkularer Polarisation dreht sich die Polarisation, während sich das Signal ausbreitet. Es existieren Antennen mit rechtsseitiger zirkularer Polarisation (RHCP) und mit linksseitiger zirkularer Polarisation (LHCP). 

\coleng

TBD

\colger

RHCP kommen in der Regel bei analogen FPV-Drohnen zum Einsatz und LHCP-Antennen bei FPV-Drohnen mit digitaler Bildübertragung.

\colende

\renewcommand{\deutschertitel}{Kamera}
\renewcommand{\englischertitel}{Camera}
\makroabschnitt
\label{AbschnittKamera}

TBD

\colger

Dieser Abschnitt befasst sich ausschließlich mit analogen analoge FPV-Kameras. Digitale Systeme zur Übertragung des Livebildes kamen bei den bisher durchgeführten Projekten nicht zum Einsatz.

\coleng

TBD

\colger

Die Kamera erzeugt ein Videosignal (NTSC oder PAL), das über den Flugcontroller an den Videosender weitergeleitet wird. Bei digitalen Systemen sind die Kameras direkt mit dem Videosender verbunden. 

\coleng

TBD

\colger

Die Bildqualität analoger Kameras wirkt durch Bildfehler und Unschärfen nicht mehr zeitgemäß und weit entfernt von HD-Qualität. Der Bildeindruck ist subjektiv im schlechtesten Fall mit VHS-Kassetten und im besten Fall mit DVDs vergleichbar. Analoge Bildübertragung hat aber mehrere Vorteile wie den günstigeren Preis, der bestmöglichen Latenz, geringes Gewicht und die herstellerübergreifende Kompatibilität der verfügbaren Komponenten. 

\coleng

TBD

\colger

Die Aufnahme qualitativ hochwerter Videos ist über eine analoge Kamera nicht möglich. Wenn diese Anforderung vorliegt, ist eine zusätzliche Kamera Actioncam oder etwas Vergleichbares nötig und muss auf der Drohne mitfliegen. 

\coleng

TBD

\colger

Die am Markt verfügbaren Kameras unterscheiden sich primär in der Auflösung, der Brennweite der Linse, die das Sichtfeld beeinflusst, und der Bauform (Breite).

\coleng

TBD

\colger

Die Kamera darf für den verwendeten Rahmen (siehe Abschnitt~\ref{AbschnittFrames}) nicht zu breit sein. Modernen Kameras sind 14\,mm (Nano-Kamera) oder 19\,mm (Micro-Kamera) breite.  Abbildung~\ref{AbbildungCaddyAntNanoCamera} zeigt eine Nano-Kamera und Abbildung~\ref{AbbildungRunCamPhoenix2Camera} zeigt eine Micro-Kamera.

\colende


\begin{figure}[htbp]
  \centering
  \begin{minipage}[t]{0.48\textwidth}
    \centering
    \includegraphics[width=.5\linewidth]{Caddx_ANT_Nano_1200TVL_14mm_165FOV_top.jpg}
    \vspace{0pt} % sorgt für Top-Ausrichtung
  \end{minipage}\hfill
  \begin{minipage}[t]{0.48\textwidth}
    \centering
    \includegraphics[width=.5\linewidth]{Caddx_ANT_Nano_1200TVL_14mm_165FOV_back.jpg}
    \vspace{0pt}
  \end{minipage}
  \caption{Caddx ANT Nano Analog Camera 1200\,TVL 14x14\,mm (Top and Back)}
  \label{AbbildungCaddyAntNanoCamera}
\end{figure}

\begin{figure}[htbp]
  \centering
  \begin{minipage}[t]{0.48\textwidth}
    \centering
    \includegraphics[width=.5\linewidth]{RunCam_Phoenix_2_1000TVL_19mm_155FOV_top.jpg}
    \vspace{0pt} % sorgt für Top-Ausrichtung
  \end{minipage}\hfill
  \begin{minipage}[t]{0.48\textwidth}
    \centering
    \includegraphics[width=.5\linewidth]{RunCam_Phoenix_2_1000TVL_19mm_155FOV_back.jpg}
    \vspace{0pt}
  \end{minipage}
  \caption{RunCam Phoenix 2 Analog Camera 1000\,TVL 19x19\,mm (Top and Back)}
  \label{AbbildungRunCamPhoenix2Camera}
\end{figure}

\colstart

TBD

\colger

Die horizontale Auflösung ist in TV-Lines (TVL) angegeben und meist 1000, 1200 oder 1500\,TVL. Je höher dieser Wert ist, desto besser ist die mögliche Bildqualität. 

\coleng

TBD

\colger

Die Brennweite definiert das Sichtfeld. Je kleiner der Wert ist, desto größer ist das Sichtfeld, was vorteilhaft beim Fliegen ist. Eine  niedrige Brennweite (z.B.\, 1.8\,mm oder 2.1\,mm) führt allerdings zu einem Fischaugen-Effekt und dementsprechend zu Verzerrungen, die das Auge schneller ermüden lassen und besonders kleine Objekte schwerer erkennbar machen. 

\coleng

TBD

\colger

Hohe Brennweiten (z.B.\, 2.5\,mm oder 2.8\,mm) bieten ein angenehmeres Bildgefühl, aber engen das Sichtfeld ein und erschweren so evtl. die Orientierung im Flug. 

\colende


\renewcommand{\deutschertitel}{Videobrille (FPV-Brille)}
\renewcommand{\englischertitel}{FPV Goggles}
\makroabschnitt
\label{AbschnittFPVBrille}

TBD

\colger

Die verwendete Videobrille muss zum Videosender passen. Komponenten, zur digitalen Livebildübertratung sind in der Regel nur dann zueinander kompatibel, wenn sie vom gleichen Hersteller stammen. Analoge Videobrillen sind herstellerübergreifend zu allen analogen Videosendern kompatibel.

\coleng

TBD

\colger

Die am Markt verfügbaren Videobrillen unterscheiden sich in erster Linie hinsichtlich der Auflösung, der verwendeten Display-Technologie und den Schnittstellen (Eingänge und Ausgänge). Tabelle~\ref{TabelleVideobrillen}) zeigt eine Gegenüberstellung einiger relevanter Merkmale bei einigen am Markt verfügbaren analogen Videobrillen. 

\colende

\begin{table}[htb!]
\centering
\captionabove{Overview about analog FPV Goggles}
\label{TabelleVideobrillen}
\begin{tabular}{l@{\hskip 8mm}r@{\hskip 8mm}l@{\hskip 8mm}l}
\toprule
Goggle (Product)        & Resolution    & Display  & Video Output    \\
\midrule
Skyzone Cobra X V4      & 1280x720\,px  & LCD  & analog A/V          \\
Skyzone SKY04X Pro      & 1920x1080\,px & OLED & analog A/V          \\
Skyzone SKY04O Pro      & 1280x720\,px  & OLED & analog A/V          \\
Skyzone SKY02O          & 640x400\,px   & LCD  & analog A/V          \\
FatShark Dominator      & 1920x1080\,px & OLED & USB-C video output  \\
FatShark Dominator HDO+ & 1920x1080\,px & OLED & analog A/V          \\
FatShark Recon HD       & 1920x1080\,px & LCD  & USB-C video output  \\
Fat Shark Echo          & 800x480\,px   & LCD  & no video output     \\
Eachine EV300O          & 1024x768\,px  & OLED & analog A/V          \\ 
Eachine EV300D          & 1280x960\,px  & LCD  & analog A/V          \\ 
Rotorama 800D           & 800x480\,px   & LCD  & no video output     \\
\bottomrule
\end{tabular}
\end{table}

\colstart

TBD

\colger

Besonders die Möglichkeit, das Videosignal über eine Schnittstelle auszugeben und in einem anderen Gerät in Echtzeit zu verarbeiten ist für viele denkbare KI-Projekte eine interessante Option. Zahlreiche Videobrillen verfügen zwar über einen HDMI-Port, ermöglichen darüber aber ausschließlich das Abspielen von Videos von externen Quellen. Die Ausgabe über HDMI an ein bietet keine gängige FPV-Brille. 

\coleng

TBD

\colger

Die meisten Videobrillen verfügen über eine 3,5\,mm AV-Schnittstelle über die Video und Audio analog ein- und ausgegeben werden können. Mit Hilfe eines Videograbbers kann das Ausgabesignal an einen Computer weitergeleitet und dort verarbeite werden. 

\coleng

TBD

\colger

Verfügt eine Videobrille über eine USB-C-Schnittstelle zur Ausgabe des Videosignals, kommt es darauf an, um das Videosignal wie bei einer Webcam als USB Video Class (UVC) exportiert wird, oder als HDMI-Signal. Bei einem UVC-Gerät kann der Videostrom direkt  weiterverarbeitet werden. Handelt es sich hingegen um ein HDMI-Signal, ist auch ein geeigneter Videograbbers nötig, um das Ausgabesignal an einen Computer weiterzuleiten und zu verarbeiten.

\colende

\renewcommand{\deutschertitel}{Software zum Betrieb von FPV-Drohnen}
\renewcommand{\englischertitel}{Software for Using FPV Drones}
\chapter[\protect{\vspace{2pt}\englischertitel}]{}
\kapitel{\deutschertitel}

\label{KapitelSoftware}

\begin{paracol}{2}[]

{\raggedright\huge\bfseries\sffamily \englischertitel \par\ } \\[1.8ex]

\switchcolumn

{\raggedright\huge\bfseries\sffamily \deutschertitel \par\ } \\[1.8ex]

\coleng

This chapter introduces the essential software components required for operating FPV drones. Similar to Chapter~\ref{KapitelHardware}, it does not aim to provide a complete overview of all available software. Instead, it presents a selection of popular firmware options for flight controllers, transmitters, and receivers, as well as several useful tools. The chapter provides an overview of their core functionalities and suitable use cases, offering guidance for installation and initial configuration.

\colger

Dieses Kapitel stellt die wichtigsten Software-Komponenten zum Betrieb von FPV-Drohnen vor. Genau wie Kapitel~\ref{KapitelHardware} hat auch dieses Kapitel nicht den Anspruch, einen vollständigen Überblick über die verfügbare Software zu geben. Vorgestellt werden eine Auswahl populärer Firmwares für Flugcontroller, Sender und Empfänger sowie einige nützliche Werkzeuge. Das Kapitel soll einen Überblick über den grundlegenden Funktionsumfang und sinnvolle Einsatzbereiche geben und bei den ersten Schritten der Installation und Administration unterstützen.

\colende

\renewcommand{\deutschertitel}{Flight Controller Firmware}
\renewcommand{\englischertitel}{Flight Controller Firmware}
\makroabschnitt
\label{AbschnittFirmwareFC}

Every flight controller requires an operating system that primarily processes sensor data (including gyroscope, accelerometer, GPS, and magnetic compass) and user control inputs into motor commands. In addition, flight controller operating systems implement various flight modes (e.g., Acro, Angle, Horizon), route video data from the camera to the video transmitter, provide an on-screen display (OSD), generate and store flight logs (Blackbox logs) for post-flight analysis, allow fine-tuning of flight behavior using filters and mixers, and offer multiple interfaces for communication and configuration.

\colger

Jeder Flugcontroller benötigt ein Betriebssystem, das in erster Linie die Sensordaten (u. a. Gyroskop, Beschleunigungssensor, GPS, magnetischer Kompass) und die Steuerbefehle des Benutzers in Motorbefehle umsetzt. Darüber hinaus implementieren die Betriebssysteme der Flugcontroller verschiedene Flugmodi (z. B. Acro, Angle, Horizon), leiten Videodaten der Kamera an den Videosender weiter, bieten ein On-Screen-Display (OSD), erzeugen und speichern Log-Daten (Blackbox-Logs) zur nachträglichen Analyse des Flugverhaltens, ermöglichen die Feineinstellung des Flugverhaltens mithilfe von Filtern und Mixern und stellen verschiedene Schnittstellen zur Kommunikation und Konfiguration bereit.

\coleng

The operating system of the flight controller is implemented as firmware stored in the flash memory. Since all common flight controller firmwares are open-source projects, it occasionally happens that developer groups split off and create forks that, through new features and more active development, may eventually surpass the original project. Likewise, projects may become obsolete over time due to developer inactivity or shifting priorities.

\colger

Das Betriebssystem des Flugcontrollers wird als Firmware implementiert, die im Flash-Speicher abgelegt ist. Da alle gängigen Flugcontroller-Firmwares Open-Source-Projekte sind, kommt es immer wieder vor, dass sich Entwicklergruppen abspalten und Forks gründen, die durch neue Funktionen und aktivere Weiterentwicklung mittelfristig das ursprüngliche Projekt verdrängen. Ebenso kann es passieren, dass Projekte aufgrund fehlender Entwickleraktivität oder veränderter Lebensumstände nicht weiterentwickelt werden und mit der Zeit obsolet werden.

\coleng

This document introduces the flight controller firmwares Betaflight, INAV, and ArduPilot.

\colger

Dieses Dokument stellt die Flugcontroller-Firmwares Betaflight, INAV und ArduPilot vor.

\colende

\renewcommand{\deutschertitel}{Betaflight}
\renewcommand{\englischertitel}{Betaflight}
\makrounterabschnitt
\label{AbschnittBetaflight}

Betaflight is a firmware for flight controllers used in FPV drones. Its focus lies on supporting manual flight modes such as freestyle and racing. The main goals of Betaflight are fast response to user input and extensive configuration options for optimizing flight behavior (tuning).

\colger

Betaflight ist eine Firmware für Flugcontroller von FPV-Drohnen. Der Fokus von Betaflight liegt auf der Unterstützung des manuellen Fliegens (Freestyle, Racing usw.). Ziele von Betaflight sind eine schnelle Reaktion auf Benutzereingaben und umfangreiche Einstellungsmöglichkeiten zur Optimierung des Flugverhaltens (Tuning).

\coleng

Unlike alternative firmwares such as INAV or ArduPilot, Betaflight provides almost no support for autonomous flight. The only exceptions are the \textsl{Rescue Mode}, which allows the drone to return automatically in case of connection loss using GPS, and an experimental mode for maintaining altitude and position automatically.

\colger

Im Gegensatz zu alternativen Firmwares wie INAV oder ArduPilot unterstützt Betaflight kaum autonomes Fliegen. Ausnahmen sind der \textsl{Rescue Mode}, der die Drohne bei Verbindungsproblemen mithilfe eines GPS-Moduls automatisch zurückkehren lässt, sowie ein experimenteller Modus zum automatischen Halten von Höhe und Position.
 
\coleng

Betaflight is compatible with almost all flight controllers and receivers available on the market and provides an intuitive graphical user interface.

\colger

Betaflight ist mit nahezu allen am Markt verfügbaren Flugcontrollern und Empfängern kompatibel und bietet eine intuitive grafische Benutzeroberfläche.

\colende

\renewcommand{\deutschertitel}{Installation und Konfiguration}
\renewcommand{\englischertitel}{Installation and Configuration}
\makrounterunterabschnitt
\label{AbschnittInstallationBetaflight}

The installation and configuration of the firmware are carried out using the locally installed Betaflight Configurator or the web application available at \url{app.betaflight.com}. Figure~\ref{AbbildungBetaflighInstallation} shows the option for installing Betaflight via the web application. The model of the flight controller is usually detected automatically. This page can be accessed via the \textsl{Update Firmware} button at the top of the website. Existing configuration backups can also be restored to the flight controller on this page. A backup of the current settings is automatically created when a new version is installed.

\colger

Die Installation und Konfiguration der Firmware erfolgt entweder mit der lokal installierten Software Betaflight Configurator oder über die Webanwendung \url{app.betaflight.com}. Abbildung~\ref{AbbildungBetaflighInstallation} zeigt die Möglichkeit, Betaflight über die Webanwendung zu installieren. Das Modell des Flugcontrollers wird in der Regel automatisch erkannt. Diese Seite erreicht man über den Button \textsl{Update Firmware} am oberen Rand der Webseite. Ein vorhandenes Backup von Einstellungen kann auf dieser Seite ebenfalls in den Flugcontroller eingespielt werden. Ein Backup der aktuellen Einstellungen wird bei der Installation einer neuen Version automatisch angelegt.

\colende

\begin{figure}[htb!]
  \centering
    \includegraphics[width=\linewidth]{Betaflight_4_5_3_installation_update_backup_firmware_screenshot_app.png}
  \caption{Installation of Betaflight and Option to Backup / Restore of Configuration Settings}
  \label{AbbildungBetaflighInstallation}
\end{figure}

\begin{figure}[htb!]
  \centering
    \includegraphics[width=\linewidth]{Betaflight_4_5_3_hauptseite_firmware_screenshot_app.png}
  \caption{Main Page of Betaflight}
  \label{AbbildungBetaflightAppHauptseite}
\end{figure}

\colstart

Figure~\ref{AbbildungBetaflightAppHauptseite} shows the main page (\textsl{Setup}) of Betaflight when accessed through the web application. On the main page, the accelerometer and magnetic compass can be calibrated. It also provides general information about the drone’s current status and its components. The main page is a valuable tool for verifying the correct orientation configuration of the flight controller within the frame. Especially in flight controller stacks and small drones, it often happens that the flight controller is mounted rotated or upside down. The main page displays the drone’s position as a live 3D model corresponding to the orientation configuration set on the \textsl{Configuration} page.

\colger

Abbildung~\ref{AbbildungBetaflightAppHauptseite} zeigt die Startseite (\textsl{Setup}) von Betaflight beim Zugriff über die Webanwendung. Auf der Startseite können der Beschleunigungssensor und der magnetische Kompass kalibriert werden. Sie enthält zudem allgemeine Informationen über den Zustand der Drohne und ihrer Komponenten. Ein wertvolles Werkzeug ist die Startseite bei der Kontrolle der korrekten Konfiguration der Ausrichtung des Flugcontrollers im Rahmen. Besonders bei Flight-Controller-Stacks und kleineren Drohnen kommt es häufig vor, dass der Flugcontroller gedreht oder auf dem Kopf stehend verbaut ist. Die Startseite zeigt die Position der Drohne als Live-Bild entsprechend der in der Seite \textsl{Configuration} eingestellten Ausrichtung des Flugcontrollers.

\coleng

Figure~\ref{AbbildungBetaflightAppPorts} shows the \textsl{Ports} page, where the available UART interfaces are configured. Correctly setting up the UART functions is essential for using many components connected to the drone, such as the video transmitter (in Figure~\ref{AbbildungBetaflightAppPorts} on UART3), GPS module (on UART5), and receiver (on UART6).

\colger

Abbildung~\ref{AbbildungBetaflightAppPorts} zeigt die Seite \textsl{Ports}. Hier wird die Konfiguration der verfügbaren UART-Schnittstellen vorgenommen. Die korrekte Einstellung der Funktionsweise der UARTs ist Voraussetzung für die Nutzung vieler an die Drohne angeschlossener Komponenten, wie beispielsweise des Videosenders (in Abbildung~\ref{AbbildungBetaflightAppPorts} an UART3), des GPS-Moduls (an UART5) und des Empfängers (an UART6).

\colende

\begin{figure}[htb!]
  \centering
    \includegraphics[width=\linewidth]{Betaflight_4_5_3_ports_firmware_screenshot_app.png}
  \caption{Ports Page of Betaflight}
  \label{AbbildungBetaflightAppPorts}
\end{figure}

\colstart

On the \textsl{Configuration} page (see Figures~\ref{AbbildungBetaflightAppConfiguration1} and~\ref{AbbildungBetaflightAppConfiguration2}), settings are made for, among other things, the orientation of the flight controller and the gyro, which is essential for every drone as it serves as the core instrument for attitude determination.

\colger

Auf der Seite \textsl{Configuration} (siehe Abbildungen~\ref{AbbildungBetaflightAppConfiguration1} und~\ref{AbbildungBetaflightAppConfiguration2}) werden unter anderem Einstellungen zur Ausrichtung des Flugcontrollers und des für jede Drohne unverzichtbaren Gyroskops (Kreiselinstrument zur Lagebestimmung) vorgenommen.

\colende

\begin{figure}[htb!]
  \centering
    \includegraphics[width=\linewidth]{Betaflight_4_5_3_configuration1_firmware_screenshot_app.png}
  \caption{Configuration Page of Betaflight (Part 1/2)}
  \label{AbbildungBetaflightAppConfiguration1}
\end{figure}

\colstart

Safety-relevant settings, such as the maximum tilt angle of the drone at which arming (activating the motors to prepare for flight) is allowed, are also configured on the \textsl{Configuration} page. If an on-screen display (OSD) is to be shown in the FPV live video by the flight controller, it must be enabled here. The \textsl{Airmode} option, shown as active in Figure~\ref{AbbildungBetaflightAppConfiguration2}, ensures that the motors continue to spin slightly even at zero throttle, which improves both the stability and controllability of the drone.

\colger

Auch sicherheitsrelevante Einstellungen, wie der maximale Neigungswinkel der Drohne, bei dem das Arming (das Aktivieren der Motoren zur Flugvorbereitung) überhaupt möglich ist, werden auf der Seite \textsl{Configuration} festgelegt. Soll im Livebild der FPV-Drohne durch den Flugcontroller ein On-Screen-Display (OSD) eingeblendet werden, muss es hier grundsätzlich aktiviert werden. Die in Abbildung~\ref{AbbildungBetaflightAppConfiguration2} aktivierte Einstellung \textsl{Airmode} sorgt dafür, dass die Motoren auch bei null Throttle leicht weiterlaufen, was die Stabilität und Kontrollierbarkeit der Drohne verbessert.

\colende

\begin{figure}[htb!]
  \centering
    \includegraphics[width=\linewidth]{Betaflight_4_5_3_configuration2_firmware_screenshot_app.png}
  \caption{Configuration Page of Betaflight (Part 2/2)}
  \label{AbbildungBetaflightAppConfiguration2}
\end{figure}

\colstart

The \textsl{Power \& Battery} page allows you to define the voltage levels at which the battery is considered fully charged or discharged, as well as the thresholds at which the flight controller should issue warnings.

\colger

Die Seite \textsl{Power \& Battery} ermöglicht es, die Spannungswerte zu definieren, bei denen der Akku als vollständig geladen oder entladen gilt, sowie die Schwellen, bei denen der Flugcontroller entsprechende Warnungen ausgeben soll.

\coleng

On the \textsl{Failsafe} page, you can specify in great detail how the flight controller should respond if the pilot triggers a failsafe condition or if signal loss occurs. Useful settings include having the drone perform a controlled landing or, alternatively, return to its starting point using GPS.

\colger

Auf der Seite \textsl{Failsafe} kann sehr detailliert festgelegt werden, wie der Flugcontroller reagieren soll, wenn der Pilot ein Problem meldet oder ein Signalverlust auftritt. Sinnvolle Einstellungen sind beispielsweise, dass die Drohne kontrolliert landen oder mithilfe des GPS-Moduls selbstständig zum Startpunkt zurückkehren soll.

\coleng

The \textsl{Presets} page allows you to save and restore your own Betaflight configuration. It also provides access to a public database of shared configurations, which can be searched and applied based on various parameters.

\colger

Die Seite \textsl{Presets} ermöglicht es, die eigene Betaflight-Konfiguration zu speichern und wiederherzustellen. Zudem bietet sie Zugriff auf eine öffentliche Datenbank veröffentlichter Konfigurationen, die nach verschiedenen Parametern durchsucht und angewendet werden können.

\coleng

On the \textsl{PID Tuning} (Proportional-Integral-Derivative) page, fine adjustments to the motor control can be made through the flight controller. The goal is typically to achieve more precise flight behavior and to counteract undesirable drone reactions such as oscillations or overly slow or aggressive responses to control inputs.

\colger

Auf der Seite \textsl{PID Tuning} (Proportional-Integral-Derivative) können Feineinstellungen in der Motorsteuerung durch den Flugcontroller vorgenommen werden. Ziel ist es meist, ein präziseres Flugverhalten zu erreichen und unerwünschte Reaktionen der Drohne -- etwa Zittern oder ein zu langsames bzw. zu aggressives Reagieren auf Steuerbefehle -- zu vermeiden.

\coleng

The correct operation of the receiver can be verified on the \textsl{Receiver} page (see Figure~\ref{AbbildungBetaflightAppReceiver}). Movements of the transmitter controls are displayed here when received by the receiver. The connection type and communication protocol are also defined on this page. For the receiver to operate correctly, it must be assigned to the correct UART on the \textsl{Ports} page.

\colger

Die korrekte Funktionsweise des Empfängers kann auf der Seite \textsl{Receiver} (siehe Abbildung~\ref{AbbildungBetaflightAppReceiver}) überprüft werden. Bewegungen an den Bedienelementen des Senders werden, sobald sie vom Empfänger empfangen werden, hier angezeigt. Auch die Anschlussart des Empfängers und das verwendete Protokoll sind auf dieser Seite definiert. Damit der Empfängers korrekt funktioniert, muss er auf der Seite \textsl{Ports} dem richtigen UART zugewiesen sein.

\coleng

If telemetry data is to be transmitted from the receiver to the transmitter, the corresponding option must be enabled on the \textsl{Receiver} page. Telemetry data typically include information such as battery voltage, current consumption, signal quality, GPS data (latitude, longitude, speed, number of satellites), altitude, and temperature.

\colger

Soll der Empfänger Telemetriedaten an den Sender übertragen, muss die entsprechende Option auf der Seite \textsl{Receiver} aktiviert werden. Zu den Telemetriedaten gehören in der Regel Informationen wie Akkuspannung, Stromverbrauch, Signalqualität, GPS-Daten (Breitengrad, Längengrad, Geschwindigkeit, Anzahl der Satelliten), Höhe und Temperatur.

\colende

\begin{figure}[htb!]
  \centering
    \includegraphics[width=\linewidth]{Betaflight_4_5_3_receiver_firmware_screenshot_app.png}
  \caption{Receiver Page of Betaflight}
  \label{AbbildungBetaflightAppReceiver}
\end{figure}

\colstart

The assignment of important drone functions to individual switches on the transmitter is configured on the \textsl{Modes} page (see Figure~\ref{AbbildungBetaflightAppModes}). The most important function here is arming the motors to prepare them for flight. Other functions that are sensibly assigned to transmitter switches include flight modes (Angle, Horizon, and Acro), the beeper (buzzer) for locating a crashed drone more easily, and the automatic return-to-home (GPS Rescue) function.

\colger

Die Verknüpfung wichtiger Funktionen der Drohne mit einzelnen Schaltern des Senders erfolgt auf der Seite \textsl{Modes} (siehe Abbildung~\ref{AbbildungBetaflightAppModes}). Die wichtigste Funktion ist hier das Anlaufen (\glqq scharf schalten\grqq) der Motoren, um sie für den Flug vorzubereiten. Weitere Funktionen, die sinnvollerweise Schaltern des Senders zugeordnet werden, sind unter anderem die Flugmodi (Angle, Horizon und Acro), der Pieper (Buzzer) zum leichteren Wiederfinden abgestürzter Drohnen sowie die automatische Rückkehr (GPS Rescue) zum Startpunkt.

\colende

\begin{figure}[htb!]
  \centering
    \includegraphics[width=\linewidth]{Betaflight_4_5_3_modes_firmware_screenshot_app.png}
  \caption{Modes Page of Betaflight}
  \label{AbbildungBetaflightAppModes}
\end{figure}

\colstart

The SA switch (AUX1) is very often assigned to the arming function. However, using only one switch increases the risk of accidental arming, which can lead to injuries, such as to hands or fingers. To improve safety when holding the drone, it is therefore recommended to link the arming process to two switches -- for example, SA (AUX1) + SB (AUX4) -- that are not located next to each other. This exact configuration is shown in Figure~\ref{AbbildungBetaflightAppModes}.

\colger

Sehr häufig wird der Schalter SA (AUX1) mit dem Arming belegt. Da bei nur einem Schalter auch ein versehentliches Arming vorkommen kann, treten Verletzungen, z. B. an Händen und Fingern, häufiger auf. Zur Verbesserung der Sicherheit beim Halten der Drohne empfiehlt es sich daher, das Arming mit zwei Schaltern -- z. B. SA (AUX1) + SB (AUX4) -- zu verknüpfen, die nicht direkt nebeneinander liegen. Genau diese Einstellung ist auch in Abbildung~\ref{AbbildungBetaflightAppModes} zu sehen.

\coleng

Another common configuration is to assign a three-position switch for selecting the flight mode. As shown in Figure~\ref{AbbildungBetaflightAppModes}, the SB switch (AUX2) can be configured so that the lower position activates the Angle mode, the middle position activates Horizon mode, and the upper position activates Acro mode.

\colger

Eine weitere etablierte Konfiguration ist die Belegung eines Schalters mit drei möglichen Positionen, über den der Flugmodus eingestellt wird. So kann beispielsweise, wie in Abbildung~\ref{AbbildungBetaflightAppModes} dargestellt, der Schalter SB (AUX2) so konfiguriert werden, dass bei der unteren Position der Flugmodus Angle, bei der mittleren Position Horizon und bei der oberen Position Acro verwendet wird.

\coleng

On the \textsl{Adjustments} page, it is possible to define which drone parameters can be adjusted live during flight using specific channels (switches) or potentiometers (dials). This feature allows advanced pilots to fine-tune flight behavior in real time.

\colger

Auf der Seite \textsl{Adjustments} ist es möglich zu definieren, welche Parameter der Drohne im Flug mit bestimmten Kanälen (Schaltern) oder Potentiometern (Drehreglern) live angepasst werden können. Diese Seite eröffnet Möglichkeiten zur Feineinstellung für fortgeschrittene Drohnenpiloten während des Fluges.

\coleng

The \textsl{Servos} page defines control options for servo motors. Here, transmitter channels (switches) are assigned to individual servos, and fine adjustments such as midpoint (neutral position) and endpoints (Min/Max) can be set. Typical applications include camera gimbal control or building a gripping mechanism for delivering or collecting small objects.

\colger

Die Definition von Steuermöglichkeiten für Servomotoren erfolgt auf der Seite \textsl{Servos}. Hier werden Kanäle (Schalter des Senders) einzelnen Servomotoren zugeordnet und nötige Feineinstellungen wie Mittelstellung (Null-Position) und Endpunkte (Min/Max) vorgenommen. Anwendungsszenarien sind zum Beispiel eine Gimbal-Steuerung der Kamera oder der Bau eines Greifmechanismus, um Objekte mit einer Drohne auszuliefern oder aufzusammeln.

\coleng

If a GPS module is connected to the drone's flight controller, configuration and monitoring of the GPS functionality can be performed on the \textsl{GPS} page. For the GPS module to operate correctly, it must be assigned to the correct UART on the \textsl{Ports} page. When a sufficient number of satellites are visible, this page displays the current coordinates (latitude and longitude), speed, direction of movement, and altitude. The distance from the home position is also shown here. During flight, this information is useful when it is transmitted to the transmitter via telemetry data.

\colger

Wenn an den Flugcontroller der Drohne ein GPS-Modul angeschlossen ist, können Kontrolle und Konfiguration der GPS-Funktionalität auf der Seite \textsl{GPS} erfolgen. Damit das GPS-Modul korrekt funktioniert, muss es auf der Seite \textsl{Ports} dem richtigen UART zugewiesen sein. Besteht Sichtkontakt zu ausreichend vielen Satelliten, können hier unter anderem die aktuellen Koordinaten (Breiten- und Längengrad), Geschwindigkeit, Bewegungsrichtung und Höhe kontrolliert werden. Auch die Entfernung zum Startplatz ist hier sichtbar. Während des Fluges sind diese Informationen hilfreich, wenn sie über die Telemetriedaten an den Sender übertragen werden.

\colende

\begin{figure}[htb!]
  \centering
    \includegraphics[width=\linewidth]{Betaflight_4_5_3_motos_firmware_screenshot_app.png}
  \caption{Motors Page of Betaflight}
  \label{AbbildungBetaflightAppMotors}
\end{figure}

\colstart

All motor-related settings and the testing of motor function, mapping, and rotation direction are performed on the \textsl{Motors} page. These tests must always be carried out without propellers, as otherwise damage or even injury can easily occur.

\colger

Alle Einstellungen bezüglich der Motoren sowie das Testen der korrekten Funktion, Zuordnung und Drehrichtung der einzelnen Motoren erfolgen auf der Seite \textsl{Motors}. Diese Tests müssen immer ohne Propeller durchgeführt werden, da es sonst leicht zu Beschädigungen oder gar Verletzungen kommen kann.

\coleng

The precise configuration of the on-screen display (OSD) is done on the \textsl{OSD} page. Here, the positions and layout of the desired display elements can be freely arranged, and multiple configurations can be saved as profiles--for example, for racing, freestyle, or long-range flying. For OSD configuration to be possible at all, the flight controller must include an OSD chip. If it does not, an analog video transmitter cannot provide an OSD. Digital video transmitters (e.g., from DJI, HDZero, and Walksnail) implement their own OSD systems.

\colger

Die präzise Konfiguration des On-Screen-Displays erfolgt auf der Seite \textsl{OSD}. Hier können die Positionen und das Layout der gewünschten Anzeigen frei platziert und mehrere Konfigurationen als Profile -- zum Beispiel für Racing, Freestyle oder Longrange -- gespeichert werden. Damit die Konfiguration des OSD überhaupt möglich ist, muss der Flugcontroller über einen OSD-Chip verfügen. Ist das nicht der Fall, kann mit einem analogen Videosender kein OSD realisiert werden. Digitale Videosender (z.~B. von DJI, HDZero und Walksnail) implementieren ein eigenes OSD.

\coleng

Settings for the video transmitter, such as the selected frequency (see Table~\ref{TabelleFrequenzenVTX}) and transmission power, can be configured on the \textsl{Video Transmitter} page (see Figure~\ref{AbbildungBetaflightAppVTX}). For the video transmitter to function correctly, it must be assigned to the appropriate UART on the \textsl{Ports} page.

\colger

Einstellungen zum Videosender, wie zum Beispiel die verwendete Frequenz (siehe Tabelle~\ref{TabelleFrequenzenVTX}) und die Sendeleistung, können auf der Seite \textsl{Video Transmitter} (siehe Abbildung~\ref{AbbildungBetaflightAppVTX}) vorgenommen werden. Damit der Videosender korrekt funktioniert, muss er auf der Seite \textsl{Ports} dem korrekten UART zugewiesen sein. 

\colende

\begin{figure}[htb!]
  \centering
    \includegraphics[width=\linewidth]{Betaflight_4_5_3_video_transmitter_firmware_screenshot_app.png}
  \caption{Video Transmitter Page of Betaflight}
  \label{AbbildungBetaflightAppVTX}
\end{figure}

\colstart

The \textsl{Sensors} page (see Figure~\ref{AbbildungBetaflightAppSensors}) displays the sensor data both graphically and numerically -- for example, values from the gyroscope, accelerometer, barometer, and magnetometer. This allows users to verify that all sensors are working correctly and delivering plausible data. The page is particularly useful for troubleshooting sensor orientation or calibration issues.

\colger

Die Seite \textsl{Sensors} (siehe Abbildung~\ref{AbbildungBetaflightAppSensors}) zeigt Sensorwerte sowohl grafisch als auch numerisch an -- zum Beispiel Werte des Gyroskops, Beschleunigungssensors, Höhenmessers (Barometer) und magnetischen Kompasses. Dadurch lässt sich überprüfen, ob alle Sensoren korrekt funktionieren und plausible Werte liefern. Die Seite ist insbesondere hilfreich zur Fehlersuche bei Problemen mit der Sensororientierung oder Kalibrierung.

\colende

\begin{figure}[htb!]
  \centering
    \includegraphics[width=\linewidth]{Betaflight_4_5_3_sensors_firmware_screenshot_app.png}
  \caption{Sensors Page of Betaflight}
  \label{AbbildungBetaflightAppSensors}
\end{figure}

\colstart

The live recording of sensor data (e.g. gyroscope, motor values, GPS data, altitude) while the flight controller is connected to a computer is available on the \textsl{Tethered Logging} page. In contrast to classic Blackbox logging, where data is stored locally on the flight controller, flash memory, or an SD card, Tethered Logging allows real-time data transmission and monitoring on the computer. It only works as long as a connection (e.g. via USB) between the flight controller and the computer is active. If the connection is lost, the live recording stops immediately.

\colger

Die Live-Aufzeichnung von Sensordaten (z.\,B. Gyroskop, Motorwerte, GPS-Daten, Höhenmeter) während der Verbindung mit dem Computer ist auf der Seite \textsl{Tethered Logging} möglich. Im Gegensatz zum klassischen Blackbox-Logging, bei dem die Daten lokal im Flugcontroller, auf Flash oder SD-Karte gespeichert werden, erlaubt Tethered Logging die Übertragung und Überwachung der Daten in Echtzeit auf dem Computer. Das Tethered Logging funktioniert nur, solange eine Verbindung (z.\,B. via USB) zwischen Flugcontroller und Computer besteht. Wenn die Verbindung abreißt, bricht auch die Live-Aufzeichnung ab.

\coleng

The flight data recorder for logging various sensor data is configured on the \textsl{Blackbox} page (see Figure~\ref{AbbildungBetaflightAppBlackbox}). The collected data is recorded during flight and analyzed afterwards. The resulting insights can be used to analyze flight behavior, optimize filters, and diagnose issues. Recording typically starts automatically when arming and stops when disarming. Modern flight controllers store Blackbox data on an onboard flash chip or a microSD card. This page allows the selection of which sensors to log, the storage device to use, and the log rate, which defines how frequently data points are captured. The page also offers options to export or erase Blackbox data.

\colger

Der Flugdatenschreiber zur Aufzeichnung verschiedenster Sensordaten wird auf der Seite \textsl{Blackbox} (siehe Abbildung~\ref{AbbildungBetaflightAppBlackbox}) konfiguriert. Die Aufzeichnung der gesammelten Daten geschieht während des Flugs und wird nachträglich ausgewertet. Mit den gewonnenen Erkenntnissen können das Flugverhalten analysiert, Filter optimiert und Probleme diagnostiziert werden. Die Aufzeichnung beginnt üblicherweise automatisch beim Arming und stoppt beim Disarming. Moderne Flugcontroller speichern die Blackbox-Daten auf einem Onboard-Flashchip oder einer microSD-Karte. Die Seite erlaubt neben der Auswahl der zu erfassenden Sensordaten auch die Auswahl des Speichergeräts und die Einstellung der Log-Rate, die definiert, wie häufig Datenpunkte aufgezeichnet werden sollen. Auch der Export der Blackbox und das Löschen des Speichers sind auf dieser Seite möglich.

\colende

\begin{figure}[htb!]
	\centering
	\includegraphics[width=\linewidth]{Betaflight_4_5_3_blackbox_firmware_screenshot_app.png}
	\caption{Blackbox Page of Betaflight}
	\label{AbbildungBetaflightAppBlackbox}
\end{figure}

\colstart

The \textsl{CLI} page in Betaflight (see Figure~\ref{AbbildungBetaflightAppCLI}) provides access to the integrated command-line interface of the flight controller. This interface allows users to inspect and configure the controller directly. Individual parameters can be displayed and changed using the commands \verb!get! and \verb!set!. The \verb!save! command saves all changes and restarts the flight controller.

\colger

Betaflight bietet auf der Seite \textsl{CLI} (siehe Abbildung~\ref{AbbildungBetaflightAppCLI}) Zugriff auf die integrierte Kommandozeilenumgebung des Flugcontrollers. Hierüber kann der Flugcontroller direkt untersucht und konfiguriert werden. Einzelne Parameter können mit den Kommandos \verb!get! und \verb!set! angezeigt und geändert werden. Das Kommando \verb!save! speichert alle Änderungen und startet den Flugcontroller neu.

\colende

\begin{figure}[htb!]
	\centering
	\includegraphics[width=\linewidth]{Betaflight_4_5_3_cli_firmware_screenshot_app.png}
	\caption{CLI Page of Betaflight}
	\label{AbbildungBetaflightAppCLI}
\end{figure}

\colstart

The current configuration of the flight controller can be displayed completely with the command \verb!dump!. Only the differences from the default configuration are shown using the command \verb!diff!. These outputs can be copied into text files and re-imported later using copy and paste. This makes it easy to create backups of all settings and restore them when needed.

\colger

Die aktuelle Konfiguration des Flugcontrollers kann vollständig mit dem Kommando \verb!dump! ausgegeben werden. Nur die Unterschiede zur Standardkonfiguration zeigt das Kommando \verb!diff!. Diese Ausgaben können in Textdateien gespeichert und bei Bedarf per Copy-and-Paste wieder importiert werden. So lassen sich einfach Backups der Einstellungen erstellen und wiederherstellen.

\coleng

Information about the flight controller and the installed firmware version can be retrieved with the command \verb!version!. The \verb!status! command provides current sensor readings and information about sensor configuration. The command \verb!tasks! lists all currently running processes on the flight controller, and \verb!resource! displays the pin and UART assignments.

\colger

Informationen zum Flugcontroller und zur installierten Firmwareversion liefert das Kommando \verb!version!. Das Kommando \verb!status! zeigt aktuelle Sensordaten und Informationen zur Konfiguration der Sensoren. Das Kommando \verb!tasks! listet alle aktuell laufenden Prozesse des Flugcontrollers auf, und \verb!resource! zeigt die Belegung der Pins und UART-Schnittstellen.

\colende

\renewcommand{\deutschertitel}{INAV}
\renewcommand{\englischertitel}{INAV}
\makrounterabschnitt
\label{AbschnittINAV}

 INAV consists of two main components: the firmware that runs on the flight controller and the configuration tool that is used for installation and administration. There is no web app, as is the case with current Betaflight versions.

 \colger

INAV besteht aus zwei Hauptkomponenten: Die Firmware, welche auf dem Flugcontroller läuft und dem Konfigurationstool, welches für die Installation und Administration genutzt werden kann. Es gibt keine Web-App, wie bei aktuellen Betaflight-Versionen.

\coleng

In addition to the Return-To-Home (RTH) and rescue functionality, INAV offers additional autopilot features such as waypoint missions, hold position, hold altitude, and more. These can also be used without a compass (since version 7.1), albeit with lower accuracy.

\colger

Neben der Return-To-Home- (RTH) bzw. Rescuefunktionalität bietet INAV weitere Autopilotenfunktionalitäten, wie Wegpunktmissionen, \textsl{Position halten}, \textsl{Flughöhe halten} und mehr. Diese können auch ohne Kompass genutzt werden (seit Version 7.1), jedoch mit geringerer Genauigkeit.

\coleng

The software receives a major release every year and minor support updates as needed. The current version 8 supports the following STM flight controller chips: F405, F722, F745, F765, and H743. AT-F435 chips are also supported.

\colger

Die Software erhält jedes Jahr einen großen Release und kleinere Support-Updates, wie sie benötigt werden. Die aktuelle Version 8 unterstützt die folgenden STM-Flugcontrollerchips: F405, F722, F745, F765 und H743. Außerdem werden AT-F435 Chips unterstützt.

\colende

\renewcommand{\deutschertitel}{Installation}
\renewcommand{\englischertitel}{Installation}
\makrounterunterabschnitt
\label{AbschnittInstallationINAV}

The installation of INAV is done using the \textsl{INAV Configurator}. A certain similarity to Betaflight is very apparent here, as the structure is quite similar to the Betaflight Configurator, which is used for versions up to 4.5 (Figure \ref{AbbildungINAVFirmwareFlasher}). This guide assumes that the flight controller is connected via USB to the computer used for flashing.

\colger

Die Installation erfolgt über den \textsl{INAV Configurator}.Der Aufbau dieses Prozesses ist sehr ähnlich zum Betaflight Configurator, welcher bis Version 4.5 verwendet wurde. (Abbildung \ref{AbbildungINAVFirmwareFlasher}). Dieser Guide geht davon aus, dass der Flugcontroller via USB an den Computer angeschlossen ist, mit welchem das Flashen durchgeführt wird.

\colende

\begin{figure}[htb!]
	\centering
	\includegraphics[width=\linewidth]{INAV_Firmware_Flasher.png}
	\caption{Firmware Flasher tab in the INAV Configurator}
	\label{AbbildungINAVFirmwareFlasher}
\end{figure}

\colstart

If the flight controller is not being flashed for the first time, it is good practice to create a backup of the configuration, as some upgrades reset it, or to be safe in case of a failure. This is done via the \textsl{CLI} tab. The command \texttt{diff all} outputs the corresponding parameters. These can then be saved to a text file using the \textsl{Save to File} button. The flashing process itself consists of the following steps:

\colger

Wird die Software nicht zum ersten Mal auf dem Flugcontroller installiert, sollte ein Backup der Konfiguration gemacht werden, da manche Upgrades die diese zurücksetzen. Dies geschieht über den \textsl{CLI}-Tab. Der Befehl \texttt{diff all} gibt die Parameter aus. Mit dem Button \textsl{Save to File} können diese in eine Textdatei gespeichert werden. Der Flashprozess besteht dann aus den folgenden Schritten:

\coleng

\begin{enumerate}
	\item Putting the flight controller into \textsl{Device Firmware Upgrade (DFU) mode} is the first step. This can usually be done using a button on the flight controller board or with the \texttt{dfu} command in the \textsl{CLI} tab.
\end{enumerate}

\colger

\begin{enumerate}
	\item Den \textsl{Device Firmware Upgrade (DFU)-Modus} aktivieren ist der erste notwendige Schritt. Dies kann in der Regel über einen Button am Flugcontrollerboard realisiert werden oder mit dem Befehl \texttt{dfu} im \textsl{CLI}-Tab.
\end{enumerate}

\coleng

\begin{enumerate}
	\setcounter{enumi}{1}

	\item Once the board is in DFU mode, the \textsl{Firmware Flasher} tab in the INAV Configurator must be opened. First, the target must be selected. If  \textsl{Auto-select Target} does not work, the model name must be searched for manually in the list.
\end{enumerate}

\colger

\begin{enumerate}
	\setcounter{enumi}{1}

	\item Befindet sich das Board im DFU-Modus, so muss der \textsl{Firmware Flasher} Tab im INAV Configurator aufgerufen werden. In diesem muss zunächst das Target ausgewählt werden, sollte  \textsl{Auto-select Target} nicht funktionieren, muss der Modellname händisch in der Liste gesucht werden.
\end{enumerate}

\coleng

\begin{enumerate}
	\setcounter{enumi}{2}

	\item The available (stable) releases are then displayed in the drop-down menu below the board. The \textsl{Show unstable releases} option allows you to select even more recent versions. However, these may not work as expected.
\end{enumerate}

\colger

\begin{enumerate}
	\setcounter{enumi}{2}

	\item Im Ausklappmenü unter dem Board werden dann die verfügbaren (stable) Releases angezeigt. Mit der Option \textsl{Show unstable releases} können noch aktuellere Versionen gewählt werden. Diese funktionieren aber unter Umständen nicht wie erwartet.
\end{enumerate}

\coleng

\begin{enumerate}
	\setcounter{enumi}{3}

	\item INAV offers three options that can be set for flashing:
	\begin{itemize}
		\item \textsl{No reboot sequence} - Only required if the boot pins are bridged or the boot button remains pressed during the flashing process (depending on the model).
		\item \textsl{Full chip erase} - Deletes all configuration options (backup available?). This should always be set if a different software was previously installed.
		\item \textsl{Manual baud rate} - Set a fixed baud rate for Bluetooth or USB if the model in question does not support the standard speed.
	\end{itemize}
\end{enumerate}

\colger

\begin{enumerate}
	\setcounter{enumi}{3}

	\item INAV bietet drei Optionen die für das Flashen gesetzt werden können:
	\begin{itemize}
		\item \textsl{No reboot sequence} - Wird nur benötigt, wenn die Bootpins überbrückt sind oder der Bootbutton während des Flashens gedrückt wird (modellabhängig).
		\item \textsl{Full chip erase} - Löscht alle Konfigurationsoptionen (Backup Vorhanden?). Sollte immer gesetzt werden, wenn vorher eine andere Software installiert war.
		\item \textsl{Manual baud rate} - Festlegen der Baudrate für Bluetooth oder falls das zu flashende Modell die Standardgeschwindigkeit nicht unterstützt.
	\end{itemize}
\end{enumerate}

\coleng

\begin{enumerate}
	\setcounter{enumi}{4}

	\item The flashing process consists of two steps: First, the firmware is loaded (online or file), then the software is flashed onto the board via a separate button. The actual flashing should not be interrupted, as this will brick the device. However, the bootloader required for the flashing itself is stored in ROM and \textsl{cannot be bricked}. This means the device can always be reflashed.
\end{enumerate}

\colger

\begin{enumerate}
	\setcounter{enumi}{4}

	\item Der Prozess des Flashens besteht aus zwei Schritten: Zunächst wird die Firmware geladen (online oder Datei), dann wird das eigentliche Flashen über einen separaten Button gestartet. Dies sollte nicht unterbrochen werden, da hierdurch das Board \textsl{gebrickt}  wird. Der zum Flashen benötigte Bootloader ist allerdings im ROM gespeichert und kann \textsl{nicht gebrickt} werden. Das bedeutet, dass das Board immer wieder neu geflasht werden kann.
\end{enumerate}

\coleng

\begin{enumerate}
	\setcounter{enumi}{5}

	\item If necessary, the backup can now be restored via the \textsl{CLI} tab. This is done using the \textsl{Load from File} button. This allows the corresponding text file to be selected and executed using the \textsl{Execute} button.
\end{enumerate}

\colger

\begin{enumerate}
	\setcounter{enumi}{5}

	\item Gegebenenfalls kann nun das Backup über den \textsl{CLI}-Tab wieder eingespielt werden. Dies geschieht über den Button \textsl{Load from File}. Damit kann die entsprechende Textdatei herausgesucht werden, um sie mit dem \textsl{Execute}-Button auszuführen.
\end{enumerate}

\colende

\renewcommand{\deutschertitel}{Administration}
\renewcommand{\englischertitel}{Administration}
\makrounterunterabschnitt
\label{AbschnittAdministrationINAV}

The administration of INAV also occurs in the \textsl{INAV Configurator}. When the software is installed for the first time, a few things need to be configured to enable initial flight. These are described in this section.

\colger

Die Administration von INAV erfolgt ebenfalls über den \textsl{INAV Configurator}. Wenn die Software das erste Mal installiert wurde, müssen einige Punkte konfiguriert werden, um das Fliegen initial zu ermöglichen.

\coleng

When the Configurator is launched for the first time after installation, windows appear for setting the \textsl{Default Values} (presets). Here, you need to select the preset that is closest to the model you have built. After that, you can configure the UART ports (for GPS, VTX, etc.). However, this can also be done via the Ports tab. The further configuration steps are as follows:

\colger

Wird der Configurator zum ersten Mal nach der Installation aufgerufen, erscheinen Fenster zum Setzen der \textsl{Default-Values} (Presets). Hier gilt es das Preset auszuwählen, was am nächsten an dem gebauten Modell dran ist. Danach können die UART-Ports (für GPS, VTX, etc.) konfiguriert werden. Dies kann allerdings auch über den Ports-Tab durchgeführt werden. Die weiteren Konfigurationsschritte gestalten sich wie folgt:

\coleng

\begin{enumerate}
	\item The \textsl{Status tab} provides a basic overview. Among other things, the \textsl{Pre-arming checks} overview shows whether the drone is ready for takeoff. It may be necessary to adjust the board orientation beforehand. To do this, place the drone upright with the camera facing the screen of the configuring computer and then press the \textsl{Reset Z axis} button.
\end{enumerate}

\colger

\begin{enumerate}
	\item Der \textsl{Status-Tab} stellt einen Überblick zur Verfügung. Unter anderem ist in der Übersicht \textsl{Pre-arming checks} zu sehen, ob die Drohne abheben kann. Eventuell muss hier vorher die Boardorientierung angepasst werden. Dazu wird die Drohne aufrecht mit Kamera Richtung Bildschirm des konfigurierenden Computers gestellt, um dann den \textsl{Reset Z axis} Button zu drücken. 
\end{enumerate}

\coleng

\begin{enumerate}
	 \setcounter{enumi}{1}

	 \item The \textsl{Calibration tab} (Figure \ref{AbbildungINAVCalibrationTab}) is used to calibrate the accelerometer and compass. To do this, the drone must be placed in different positions.
\end{enumerate}

\colger

\begin{enumerate}
	\setcounter{enumi}{1}

	\item Im \textsl{Calibration-Tab} (Abbildung \ref{AbbildungINAVCalibrationTab}) werden der Beschleunigungsmesser und der Kompass kalibriert. Dazu muss die Drohne jeweils in unterschiedliche Posen gebracht werden.
\end{enumerate}

\colende

\begin{figure}[htb!]
	\centering
	\includegraphics[width=\linewidth]{INAV_Calibration_Tab.png}
	\caption{INAV Calibration Tab for Board Orientation, Compass and Flow Sensor }
	\label{AbbildungINAVCalibrationTab}
\end{figure}

\colstart

\begin{enumerate}
	\setcounter{enumi}{2}

	\item The \textsl{Mixer tab}, together with the Outputs tab, configures the drone's motors. It is crucial that the correct \textsl{Platform type} and \textsl{Mixer preset} (\texttt{Multirotor} and \texttt{QUAD X} for a quadcopter) are configured according to the type of drone used. If the motors do not rotate in the correct direction, this can be adjusted via the \textsl{Motor direction} and the assignment via the \textsl{Motor Mixer Wizard}.
\end{enumerate}

\colger

\begin{enumerate}
	\setcounter{enumi}{2}

	\item Der \textsl{Mixer-Tab} konfiguriert zusammen mit dem Outputs-Tab die Motoren der Drohne. Hier ist entscheidend, dass der richtige  \textsl{Platform type} und das richtige \textsl{Mixer preset} (\texttt{Multirotor} und \texttt{QUAD X} für einen Quadcopter) entsprechend des verwendeten Drohnen-Typs konfiguriert sind. Sollten die Motoren nicht in die richtige Richtung drehen, kann die diese über die \textsl{Motor direction} angepasst werden und die Zuordnung über den \textsl{Motor Mixer Wizard}.
\end{enumerate}

\coleng

\begin{enumerate}
	\setcounter{enumi}{3}

	\item In the \textsl{Outputs tab}, the motors are activated with the \textsl{Enable motor and servo output} option. You should also ensure that \texttt{DSHOT300} is used as the \textsl{ESC protocol}. In addition, the \textsl{Motors IDLE power} can be set to 5\% if a multirotor drone is used.
\end{enumerate}

\colger

\begin{enumerate}
	\setcounter{enumi}{3}

	\item Im \textsl{Outputs-Tab} werden die Motoren aktiviert mit der Option \textsl{Enable motor and servo output}. Außerdem sollte sichergestellt werden, dass  \texttt{DSHOT300} als \textsl{ESC protocol} verwendet wird. Außerdem kann die \textsl{Motors IDLE power} auf 5\% gesetzt werden, wenn eine Multirotor-Drohne verwendet wird.
\end{enumerate}

\coleng

\begin{enumerate}
	\setcounter{enumi}{4}

	\item In the \textsl{Ports tab}, the peripheral devices are configured, unless this was done in the pop-up menu at the beginning. This looks very similar to the Ports tab of the Betaflight Configurator, with the difference that \textsl{USB VCP} is hidden instead of partially grayed out (Figure \ref{AbbildungBetaflightAppPorts}).  How these are configured depends on the connected devices and the flight controller chip. In most cases, the configuration options for Betaflight (usually found in the respective device manuals) are transferable to INAV.
\end{enumerate}

\colger

\begin{enumerate}
	\setcounter{enumi}{4}

	\item Im \textsl{Ports-Tab} werden die Peripheriegeräte konfiguriert, insofern dies nicht im Pop-Up-Menü am Anfang gemacht wurde. Dieser sieht dem Ports-Tab des Betaflight-Configurators sehr ähnlich mit dem Unterschied, dass \textsl{USB VCP} ausgeblendet statt teilweise ausgegraut ist (Abbildung \ref{AbbildungBetaflightAppPorts}).  Wie diese konfiguriert werden, hängt von den angeschlossenen Geräten und vom Flugcontrollerchip ab. Meist kann die Konfiguration für Betaflight aus den jeweiligen Handbüchern entnommen und übertragen werden.
\end{enumerate}

\coleng

\begin{enumerate}
	\setcounter{enumi}{5}

	\item The sensors can be activated in the \textsl{Configuration tab} if this has not been done automatically after configuring the ports. If possible, the \textsl{I2C Speed} should be set to \texttt{800KHZ} here, if possible.
\end{enumerate}

\colger

\begin{enumerate}
	\setcounter{enumi}{5}

	\item Im \textsl{Configuration-Tab} können die Sensoren aktiviert werden, insofern dies nicht nach Konfiguration der Ports automatisch geschehen ist. Wenn möglich, sollte hier der \textsl{I2C Speed} auf \texttt{800KHZ} gestellt werden.
\end{enumerate}

\coleng

\begin{enumerate}
	\setcounter{enumi}{6}
	
	\item The \textsl{Failsafe tab} (Figure \ref{AbbildungINAVFailsafeTab}) is very important, as it defines the behavior should the remote control signal be lost. Here, you should ensure that either \textsl{Drop}, \textsl{Land}, or \textsl{Return-to-Home (RTH)} are configured so that the drone does not continue to fly uncontrollably.
\end{enumerate}

\colger

\begin{enumerate}
	\setcounter{enumi}{6}

	\item Der \textsl{Failsafe-Tab} (Abbildung \ref{AbbildungINAVFailsafeTab}) ist sehr wichtig, da in diesem das Verhalten definiert ist, sollte das Fernsteuerungssignal abreißen. Hier sollte sichergestellt werden, dass entweder textsl{Drop}, \textsl{Land} oder \textsl{Return-to-Home (RTH)} konfiguriert sind, damit die Drohne nicht unkontrolliert weiterfliegt.
\end{enumerate}


\colende

\begin{figure}[htb!]
	\centering
	\includegraphics[width=\linewidth]{INAV_Failsafe_Tab.png}
	\caption{INAV Failsafe Tab}
	\label{AbbildungINAVFailsafeTab}
\end{figure}

\colstart

\begin{enumerate}
	\setcounter{enumi}{7}

	\item In the \textsl{Receiver tab}, the connection to the remote control from the side of the flight controller is configured. If an ExpressLRS remote control is used, \texttt{SERIAL} must be configured as the \textsl{Receiver type} and \texttt{CRSF} as the \textsl{Serial Receiver Provider}. In addition, you can check whether the channel mappings are correct. This tab is very similar to the Receiver tab in the Betaflight Configurator (Figure \ref{AbbildungBetaflightAppReceiver}).
\end{enumerate}

\colger

\begin{enumerate}
	\setcounter{enumi}{7}

	\item Im \textsl{Receiver-Tab} wird Flugcontrollerseitig die Verbindung zur Fernbedienung konfiguriert. Wird eine ExpressLRS-Fernbedienung verwendet, muss \texttt{SERIAL} als \textsl{Receiver type} und \texttt{CRSF} als \textsl{Serial Receiver Provider} konfiguriert werden. Darüber hinaus kann kontrolliert werden, ob die Channel-Mappings korrekt sind. Dieser Tab ist sehr ähnlich zum Receiver-Tab im Betaflight Configurator (Abbildung \ref{AbbildungBetaflightAppReceiver}).
\end{enumerate}

%8.
\coleng

\begin{enumerate}
	\setcounter{enumi}{8}

	\item In the \textsl{GPS tab}, the \textsl{GPS for navigation and telemetry} option must be activated in order to use the GPS. The \textsl{Protocol} option must be set according to the specifications of the model used.
\end{enumerate}

\colger

\begin{enumerate}
	\setcounter{enumi}{8}

	\item Im \textsl{GPS-Tab} muss die Option \textsl{GPS for navigation and telemetry} aktiviert werden, um das GPS nutzen zu können. Die Option \textsl{Protocol} ist entsprechend der Spezifikation des verwendeten Modells zu setzen.
\end{enumerate}

\coleng

\begin{enumerate}
	\setcounter{enumi}{9}

	\item In the \textsl{Modes tab}, which is also very similar to the Betaflight version (Figure \ref{AbbildungBetaflightAppModes}), only \texttt{ARM} needs to be configured (CH5 recommended). In addition, it is also worth configuring \texttt{ANGLE}, especially for beginner pilots, as it limits the maximum amount of pitch and roll, allowing for a smoother flight experience. \texttt{FAILSAFE} should also be configured, so it can be activated in case the drone behaves unexpectedly. Modes for autopilot functions can be configured at a later stage.
\end{enumerate}

\colger

\begin{enumerate}
	\setcounter{enumi}{9}

	Im \textsl{Modes-Tab}, der der Betaflight-Version sehr ähnlich ist (siehe Abbildung \ref{AbbildungBetaflightAppModes}), muss grundsätzlich nur \texttt{ARM} konfiguriert werden. Dafür wird Kanal 5 empfohlen.  
	Es lohnt sich außerdem, \texttt{ANGLE} einzustellen. Das ist besonders für Fluganfänger wichtig, weil dadurch die Neigung in Pitch und Roll begrenzt wird. Das macht das Fliegen einfacher.  
	Auch \texttt{FAILSAFE} sollte aktiviert werden. So kann man reagieren, falls sich die Drohne unerwartet verhält. Modi für Autopilotfunktionalitäten können zu einem späteren Zeitpunkt konfiguriert werden. 
\end{enumerate}

\coleng

\begin{enumerate}
	\setcounter{enumi}{10}

	\item The \textsl{on-screen display (OSD)} can be configured according to the pilot's preferences. Useful elements that are not displayed by default include \textsl{Remaining Flight Time} and \textsl{Battery Remaining Percentage}. Depending on the VTX model, the \textsl{Video Format} must also be configured in this tab. The relevant info can usually be found in the manual of the VTX.
\end{enumerate}

\colger

\begin{enumerate}
	\setcounter{enumi}{10}

	\item Das \textsl{on-screen display (OSD)} kann so konfiguriert werden, wie es vom Piloten bevorzugt wird. Nützliche Elemente, welche nicht standardmäßig eingeblendet werden sind zum Beispiel \textsl{Remaining Flight Time} und \textsl{Battery Remaining Percentage}. Je nach VTX-Modell muss in diesem Tab auch das \textsl{Video Format} konfiguriert werden. Relevante Informationen dazu können in der Regel im Handbuch des VTX gefunden werden.
\end{enumerate}

\colende

\begin{figure}[htb!]
	\centering
	\includegraphics[width=\linewidth]{INAV_OSD_Tab.png}
	\caption{INAV OSD Tab}
	\label{AbbildungINAVOSDTab}
\end{figure}

\renewcommand{\deutschertitel}{ArduPilot}
\renewcommand{\englischertitel}{ArduPilot}
\makrounterabschnitt
\label{AbschnittArduPilot}

TBD

\colger

ArduPilot legt als Flugcontroller-Firmware den Fokus auf vollautonomen Flug und komplexe Missionsprofile. Es ermöglicht Missionsplanung mit Wegpunkten, automatische Starts und automatisches Landen, Hindernisvermeidung, Follow-Me, etc. Hierfür nutzt ArduPilot nicht nur die üblichen Sensoren wie GPS, magnetischer Kompass, Geschwindigkeit, sondern kann auch Lidar-Sensoren zur Entfernungen und Kameras einbeziehen.

\coleng

TBD

\colger

Von den in diesem Dokument vorgestellten Flugcontroller-Firmwares braucht ArduPilot am meisten Speicher. Idealerweise sind zum Betrieb von ArduPilot 2\,MB Flash-Speicher verfügbar. Das bieten nur Flugcontroller mit einem H743 Mikrocontroller. Mit 1\,MB Flash-Speicher (F405 und F745 Mikrocontroller) ist der Betrieb mit einem reduzierten Funktionsumfang möglich. Mit nur 512\,kB (F411 und F722 Mikrocontroller) kann ArduPilot gar nicht verwendet werden.

\colende

\renewcommand{\deutschertitel}{Installation}
\renewcommand{\englischertitel}{Installation}
\makrounterunterabschnitt
\label{AbschnittInstallationArduPilot}

TBD

\colger

TBD

\colende

\renewcommand{\deutschertitel}{Administration}
\renewcommand{\englischertitel}{Administration}
\makrounterunterabschnitt
\label{AbschnittAdministrationArduPilot}

TBD

\colger

TBD

\colende

\renewcommand{\deutschertitel}{Fernbedienung Firmware}
\renewcommand{\englischertitel}{Remote Control Firmware}
\makroabschnitt
\label{AbschnittFirmwareRemoteControl}

TBD

\colger

TBD

\colende

\renewcommand{\deutschertitel}{EdgeTX}
\renewcommand{\englischertitel}{EdgeTX}
\makrounterabschnitt
\label{AbschnittEdgeTX}

TBD

\colger

TBD

\colende

\renewcommand{\deutschertitel}{Installation}
\renewcommand{\englischertitel}{Installation}
\makrounterunterabschnitt
\label{AbschnittInstallationEdgeTX}

TBD

\colger

TBD

\colende

\renewcommand{\deutschertitel}{Sendemodul- und Empfänger-Firmware}
\renewcommand{\englischertitel}{Transmitter and Receiver Firmware}
\makroabschnitt
\label{AbschnittFirmwareReceiverTransmitter}

TBD

\colger

TBD

\colende

\renewcommand{\deutschertitel}{ExpressLRS}
\renewcommand{\englischertitel}{ExpressLRS}
\makrounterabschnitt
\label{AbschnittExpressLRS}

TBD

\colger

TBD

\colende

\renewcommand{\deutschertitel}{Installation}
\renewcommand{\englischertitel}{Installation}
\makrounterunterabschnitt
\label{AbschnittInstallationExpressLRS}

TBD

\colger

TBD

\colende

\renewcommand{\deutschertitel}{Kopplung}
\renewcommand{\englischertitel}{Coupling/Pairing}
\makrounterunterabschnitt
\label{AbschnittInstallationExpressLRS}

TBD

\colger

TBD

\colende
% \renewcommand{\deutschertitel}{Wichtige Punkte vor dem ersten Flug}
\renewcommand{\englischertitel}{Important Things before the first Flight}
\chapter[\protect{\vspace{2pt}\englischertitel}]{}
\kapitel{\deutschertitel}

\label{KapitelWichtigePunkte}

\begin{paracol}{2}[]

{\raggedright\huge\bfseries\sffamily \englischertitel \par\ } \\[1.8ex]

\switchcolumn

{\raggedright\huge\bfseries\sffamily \deutschertitel \par\ } \\[1.8ex]

\coleng

This chapter provides useful advice for both the preparations before the first flight and the checks afterwards.

\colger

Dieses Kapitel enthält hilfreiche Hinweise für die Vorbereitung vor dem ersten Flug sowie für die Überprüfungen danach.

\colende

\renewcommand{\deutschertitel}{Sicherheit beim Flug}
\renewcommand{\englischertitel}{Security during Flight}
\makroabschnitt
\label{AbschnittSicherheitBeimFlug}

Always keep as much distance as possible from other people during flight. Do not fly over other people's buildings or private property without prior permission.

\colger

Während des Fluges ist stets ein möglichst großer Abstand zu anderen Personen einzuhalten. Fremde Gebäude und private Grundstücke dürfen nicht ohne vorherige Zustimmung überflogen werden.

\colende

\renewcommand{\deutschertitel}{Sicherheit vor dem Flug}
\renewcommand{\englischertitel}{Security before Flight}
\makroabschnitt
\label{AbschnittSicherheitVorFlug}

Before each flight, the arming procedure takes place. This means activating the motors in preparation for flight. Although the motors run at a relatively low speed during this process, even at this speed injuries to arms or hands can easily occur. To prevent accidental arming, it is advisable to configure the flight controller so that two switches—ideally not positioned directly next to each other—must be activated simultaneously. A configuration in Betaflight that ensures exactly this is shown in Figure~\ref{AbbildungBetaflightAppModes}.

\colger

Vor dem Flug findet das Arming statt. Dabei handelt es sich um das Aktivieren der Motoren zur Flugvorbereitung. Die Motoren laufen dabei zwar nur mit relativ niedriger Geschwindigkeit, aber selbst bei dieser Drehzahl kann es leicht zu Verletzungen, beispielsweise an Armen und Händen, kommen. Um ein unbeabsichtigtes Arming zu vermeiden, ist es ratsam, den Flugcontroller so zu konfigurieren, dass zum Arming zwei Schalter betätigt werden müssen, die idealerweise nicht direkt nebeneinander liegen. Eine Konfiguration in Betaflight, die genau dies sicherstellt, zeigt Abbildung~\ref{AbbildungBetaflightAppModes}.

\colende

\renewcommand{\deutschertitel}{Sichere Verwendung von LiPo-Akkus}
\renewcommand{\englischertitel}{Secure use of LiPo Batteries}
\makroabschnitt
\label{AbschnittSicherheitLiPoAkkus}

LiPo batteries should never be charged unattended, and they must not be charged while sleeping. Damaged batteries must no longer be used, especially if they have become swollen due to deep discharge, overcharging, or physical damage. Swollen LiPo batteries pose a significant fire hazard. A swollen battery cannot be repaired and must be properly disposed of. Disposal must not be done through household waste but through designated recycling or collection points provided by the local municipality.

\colger

LiPo-Akkus dürfen niemals unbeaufsichtigt geladen werden, insbesondere nicht während man schläft. Beschädigte Akkus dürfen nicht weiterverwendet werden, insbesondere wenn sie durch Tiefentladung, Überladung oder physische Einwirkungen aufgebläht sind. Aufgeblähte LiPo-Akkus stellen ein erhebliches Brandrisiko dar. Ein solcher Akku kann nicht repariert werden und muss ordnungsgemäß entsorgt werden. Die Entsorgung darf nicht über den Hausmüll erfolgen, sondern über die entsprechenden Sammelstellen der örtlichen Gemeinde.

\colende

\renewcommand{\deutschertitel}{Laden von Akkus}
\renewcommand{\englischertitel}{Loading of Batteries}
\makroabschnitt
\label{AbschnittLadenAkkus}

Batteries (LiPo or Li-Ion) with more than one cell must always be charged in balance mode using the voltage values read via the balancer connector (see Figure~\ref{AbbildungLiIonAkkuLaden}). This prevents individual cells from being overcharged during the charging process and helps to avoid deep discharge of individual cells during later use. The maximum recommended charge rate (see Section~\ref{AbschnittCWertAkkus}) is typically 1\,C. If this limit is exceeded, the cells will age more quickly, and in the worst case, a fire may occur. For safety reasons, Li-Ion batteries should only be stored in specially designed fireproof containers.

\colger

Akkus (LiPo oder Li-Ion) mit mehr als einer Zelle müssen immer unter Verwendung der über den Balancerstecker ausgelesenen Spannungswerte im Balancermodus geladen werden (siehe Abbildung~\ref{AbbildungLiIonAkkuLaden}). Dies verhindert eine Überladung einzelner Zellen während des Ladevorgangs und beugt einer Tiefentladung einzelner Zellen im späteren Betrieb vor. Die maximal empfohlene Laderate (siehe Abschnitt~\ref{AbschnittCWertAkkus}) beträgt üblicherweise 1\,C. Wird diese Rate überschritten, altern die Zellen schneller, und es kann im schlimmsten Fall zu einem Brand kommen. Zur Sicherheit sollten Li-Ion-Akkus nur in dafür vorgesehenen, feuerfesten Behältnissen gelagert werden.

\colende

\begin{figure}[htb!]
  \centering
    \includegraphics[width=\linewidth]{LiIo-Akku-laden.png}
  \caption{Balanced Loading of a Li-Ion Battery}
  \label{AbbildungLiIonAkkuLaden}
\end{figure}

\renewcommand{\deutschertitel}{Sichere Befestigung loser Kabel und sonstiger Komponenten}
\renewcommand{\englischertitel}{Secure Attachment of loose Cables and other Components}
\makroabschnitt
\label{AbschnittBefestigungLoseKabel}

If cables or other drone components are not properly secured and can reach the propellers, they eventually will. This can damage either the loose components or the propellers, potentially leading to a crash or costly damage. A common example of components that frequently and unintentionally come into contact with the propellers are the battery balancer connectors (see Figure~\ref{AbbildungBalancerConnectorDestroyed}) , which are not always easy to replace. Therefore, all components must be securely fastened, and cables must be fixed using screws, cable ties, Velcro straps, or at least electrical tape.

\colger

Wenn Kabel oder andere Bauteile der Drohne nicht zuverlässig befestigt sind und die Rotoren erreichen können, werden sie diese früher oder später tatsächlich erreichen. Dabei werden entweder die losen Komponenten oder die Rotoren beschädigt, was zu einem Absturz oder zu teuren Beschädigungen führen kann. Ein typisches Beispiel für Bauteile, die häufig ungewollt mit den Rotoren in Kontakt kommen, sind die Balancerstecker (siehe Abbildung~\ref{AbbildungBalancerConnectorDestroyed}) der Akkus, deren Austausch nicht immer einfach ist. Daher ist die Befestigung aller Bauteile und die Sicherung aller Kabel mit Schrauben, Kabelbindern, Klettbändern oder zumindest Isolierband zwingend erforderlich.

\colende

\begin{figure}[htb!]
  \centering
    \includegraphics[width=\linewidth]{LiPo_Akku_Schaden_durch_Propeller_crop.png}
  \caption{A Balancer Connector got destroyed because it touched a Propeller during Flight}
  \label{AbbildungBalancerConnectorDestroyed}
\end{figure}

\renewcommand{\deutschertitel}{Den geeigneten Flugmodus verwenden}
\renewcommand{\englischertitel}{Use the Appropriate Flight Mode}
\makroabschnitt
\label{AbschnittFlugmodusEinstellen}

There are three flight modes: Angle Mode (stabilized), Horizon Mode (semi-stabilized), and Acro Mode (manual).

\colger

Es gibt die Flugmodi Angle Mode (stabilisiert), Horizon Mode (halb-stabilisiert) und Acro Mode (manuell).

\coleng

In Angle Mode, the drone stabilizes itself and remains level without explicit control input from the pilot. This flight mode is the most beginner-friendly and ideal for indoor flights and smooth, slow cinematic shots. Because the maximum tilt angle is limited, spectacular flight maneuvers such as flips or rolls are not possible.

\colger

Im Angle Mode stabilisiert sich die Drohne selbst und hält sich ohne explizite Steueranweisungen des Piloten automatisch waagrecht. Dieser Flugmodus ist besonders anfängerfreundlich und ideal für Indoor-Flüge sowie für langsame, ruhige Filmaufnahmen. Da die maximale Neigung (Tilt) begrenzt ist, sind spektakuläre Flugmanöver wie Flips oder Rollen nicht möglich.
 
\coleng

In Horizon Mode, flips and rolls are possible, but the drone automatically stabilizes itself when no control input is applied.

\colger

Im Horizon Mode sind Flips und Rollen möglich, jedoch stabilisiert sich die Drohne automatisch, sobald keine Steuerkommandos mehr gegeben werden.

\coleng

In Acro Mode, which is actually the default mode, there is no automatic stabilization. The drone maintains its current rotational speed until the pilot counteracts it. This mode provides full control on all axes and allows spectacular flight maneuvers at any time. It is the least beginner-friendly mode but ideal for freestyle flying and racing.

\colger

Im Acro Mode, der eigentlich der Standardmodus ist, erfolgt keine automatische Stabilisierung. Die Drohne hält stets die aktuelle Drehgeschwindigkeit bei, bis der Pilot aktiv gegensteuert. In diesem Modus hat man die volle Kontrolle über alle Achsen, und spektakuläre Flugmanöver sind jederzeit möglich. Dieser Flugmodus ist am wenigsten anfängerfreundlich und ideal für Freestyle-Flüge und Rennen.

\coleng

Since a drone is set to Acro Mode by default, it is advisable to configure a three-position switch on the transmitter for selecting flight modes so that Angle Mode can be activated automatically when desired. A configuration in Betaflight that enables this is shown in Figure~\ref{AbbildungBetaflightAppModes}.

\colger

Da Drohnen standardmäßig im Acro Mode betrieben werden, ist es sinnvoll, einen Schalter des Senders mit drei möglichen Positionen zur Auswahl des Flugmodus so zu konfigurieren, dass bei Bedarf automatisch der Angle Mode aktiviert werden kann. Eine entsprechende Konfiguration in Betaflight, die dies ermöglicht, zeigt Abbildung~\ref{AbbildungBetaflightAppModes}.

\colende

\renewcommand{\deutschertitel}{Die Leistung des Videosenders reduzieren oder diesen abschalten}
\renewcommand{\englischertitel}{Reduce the Power of the Video Transmitter or Switch It Off}
\makroabschnitt
\label{AbschnittVTXreduzieren}

It is highly recommended to switch off the VTX or significantly reduce its output power during ground or laboratory tests, as it can overheat within a few minutes without the airflow present during flight. Even if it does not overheat, it will still become very hot quickly and draw considerable current from the batteries.

\colger

Es ist sehr empfehlenswert, bei Boden- oder Labortests den VTX abzuschalten oder seine Leistung stark zu reduzieren, da er ohne die Luftkühlung während des Flugs innerhalb weniger Minuten überhitzen kann. Selbst wenn er nicht überhitzt, wird er sehr schnell heiß und zieht viel Strom aus den Akkus.

\coleng

In Betaflight, the VTX can be completely switched off via the CLI:

\colger

In Betaflight kann man den VTX über die CLI komplett ausschalten:

\colende

\lstdefinestyle{shell}{
  backgroundcolor=\color{gray!10},
  basicstyle=\ttfamily\small,
  frame=single,
  breaklines=true,        % <-- Zeilenumbruch aktivieren
  breakatwhitespace=true, % <-- nur bei Leerzeichen umbrechen 
  postbreak=\mbox{\textcolor{gray}{$\hookrightarrow$}\space}, % Pfeil am Zeilenende
  showstringspaces=false,
  xleftmargin=0em,
  xrightmargin=0em,
  framerule=0.5pt,
  rulecolor=\color{gray!60}
}

\begin{lstlisting}[style=shell]
vtx_power = 0
save
\end{lstlisting}

\colstart

These commands switch the VTX to the reduced 25 mW power mode:

\colger

Diese Kommandos schalten den VTX in den reduzierten 25-mW-Leistungsbetrieb:

\colende

\lstdefinestyle{shell}{
  backgroundcolor=\color{gray!10},
  basicstyle=\ttfamily\small,
  frame=single,
  breaklines=true,        % <-- Zeilenumbruch aktivieren
  breakatwhitespace=true, % <-- nur bei Leerzeichen umbrechen 
  postbreak=\mbox{\textcolor{gray}{$\hookrightarrow$}\space}, % Pfeil am Zeilenende
  showstringspaces=false,
  xleftmargin=0em,
  xrightmargin=0em,
  framerule=0.5pt,
  rulecolor=\color{gray!60}
}

\begin{lstlisting}[style=shell]
vtx_power = 1
save
\end{lstlisting}

\colstart

Another option is to greatly reduce the VTX’s output power by switching it to so-called PIT mode. PIT mode lowers the transmit power to approximately 0.1\,mW -- just enough to receive the signal on the bench or within the same room, while preventing overheating. The following command switches the VTX to PIT mode:

\colger

Eine andere Möglichkeit ist, die Leistung des VTX stark zu reduzieren, indem man ihn in den sogenannten PIT-Mode schaltet. Der PIT-Mode reduziert die Sendeleistung auf ungefähr 0,1\,mW. Das ist gerade genug, um das Signal auf dem Tisch oder im Raum zu sehen und eine Überhitzung auszuschließen. Das folgende Kommando schaltet den VTX in den PIT-Mode:

\colende


\lstdefinestyle{shell}{
  backgroundcolor=\color{gray!10},
  basicstyle=\ttfamily\small,
  frame=single,
  breaklines=true,        % <-- Zeilenumbruch aktivieren
  breakatwhitespace=true, % <-- nur bei Leerzeichen umbrechen 
  postbreak=\mbox{\textcolor{gray}{$\hookrightarrow$}\space}, % Pfeil am Zeilenende
  showstringspaces=false,
  xleftmargin=0em,
  xrightmargin=0em,
  framerule=0.5pt,
  rulecolor=\color{gray!60}
}

\begin{lstlisting}[style=shell]
set vtx_pit_mode = ON
save
\end{lstlisting}

\colstart

To deactivate PIT mode again:

\colger

Um den PIT-Mode wieder zu deaktivieren:

\colende



\lstdefinestyle{shell}{
  backgroundcolor=\color{gray!10},
  basicstyle=\ttfamily\small,
  frame=single,
  breaklines=true,        % <-- Zeilenumbruch aktivieren
  breakatwhitespace=true, % <-- nur bei Leerzeichen umbrechen 
  postbreak=\mbox{\textcolor{gray}{$\hookrightarrow$}\space}, % Pfeil am Zeilenende
  showstringspaces=false,
  xleftmargin=0em,
  xrightmargin=0em,
  framerule=0.5pt,
  rulecolor=\color{gray!60}
}

\begin{lstlisting}[style=shell]
set vtx_pit_mode = OFF
save
\end{lstlisting}
\colstart

The PIT mode can also be enabled and disabled on the \textsl{Video Transmitter} page in the Betaflight Configurator or via the Betaflight web application. It is useful to assign PIT mode to a switch (AUX channel) on the transmitter. This can be done in the Betaflight Configurator or the Betaflight web application on the \textsl{Modes} page using the \textsl{VTX Pit Mode} entry.

\colger

Der PIT-Mode kann auch auf der Seite \textsl{Video Transmitter} im Betaflight Configurator oder über die Betaflight Webanwendung ein- und ausgeschaltet werden. Sinnvoll ist es, dem PIT-Mode einen Schalter (AUX-Kanal) auf der Fernbedienung zuzuweisen. Dies ist im Betaflight Configurator oder in der Betaflight Webanwendung auf der Seite \textsl{Modes} über den Eintrag \textsl{VTX Pit Mode} möglich.

\coleng

Additionally, it is possible to operate the VTX in \textsl{low-power mode} before arming and after disarming -- i.e., during flight preparation and after the flight. The following command defines that the VTX transmission power will only be increased to the configured level after arming. Before arming and after disarming, the transmission power is reduced to the value defined by \verb!vtx_power=1!.

\colger

Zudem ist es möglich, den VTX vor dem Arming und nach dem Disarming im \textsl{Low-Power-Modus} zu betreiben, also während der Flugvorbereitung und nach dem Flug. Das folgende Kommando definiert, dass die Sendeleistung des VTX erst nach dem Arming auf den eingestellten Wert erhöht wird. Vor dem Arming und nach dem Disarming ist die Sendeleistung auf den Wert von \verb!vtx_power=1! reduziert.

\colende

\lstdefinestyle{shell}{
  backgroundcolor=\color{gray!10},
  basicstyle=\ttfamily\small,
  frame=single,
  breaklines=true,        % <-- Zeilenumbruch aktivieren
  breakatwhitespace=true, % <-- nur bei Leerzeichen umbrechen 
  postbreak=\mbox{\textcolor{gray}{$\hookrightarrow$}\space}, % Pfeil am Zeilenende
  showstringspaces=false,
  xleftmargin=0em,
  xrightmargin=0em,
  framerule=0.5pt,
  rulecolor=\color{gray!60}
}

\begin{lstlisting}[style=shell]
set vtx_low_power_disarm = ON
save
\end{lstlisting}

\renewcommand{\deutschertitel}{Mit einem Simulator üben}
\renewcommand{\englischertitel}{Practice with a Simulator}
\makroabschnitt
\label{AbschnittSimulator}

Flying a drone is not easy, and crashes are almost inevitable at the beginning, even in Angle Mode. Therefore, it is highly advisable to practice with a simulator. Suitable simulators are available for all operating systems. Some well-known examples include Liftoff, VelociDrone, Uncrashed, FPV Freerider, Quadsim FPV, and Freerider Lite. 

\colger

Das Fliegen einer Drohne ist nicht einfach, und Abstürze sind insbesondere zu Beginn selbst im Angle Mode kaum zu vermeiden. Daher ist es sehr empfehlenswert, mit einem Simulator zu üben. Geeignete Simulatoren sind für alle Betriebssysteme verfügbar. Einige bekannte Produkte sind Liftoff, VelociDrone, Uncrashed, FPV Freerider, Quadsim FPV und Freerider Lite. 

\coleng

The transmitter has a USB interface and should be used as the input device for training. The USB mode must be set to \textsl{Joystick}. It is recommended to create a new model in OpenTX for this purpose, for example named \textsl{Simulator}. This model should have no active internal or external RF module, since the module consumes a significant amount of power and produces heat. This is unnecessary when using a simulator because no radio signals are transmitted.

\colger

Der Sender verfügt über eine USB-Schnittstelle und sollte als Eingabegerät zum Üben verwendet werden. Als USB-Modus muss \textsl{Joystick} ausgewählt werden. Es ist sinnvoll, hierfür in OpenTX ein neues Modell anzulegen, beispielsweise mit dem Namen \textsl{Simulator}. Dieses Modell sollte über kein aktives internes oder externes Sendemodul verfügen, da das Sendemodul signifikant Strom verbraucht und Wärme erzeugt. Dies ist beim Einsatz mit einem Simulator unnötig, da keine Funksignale gesendet werden.

\colende

\renewcommand{\deutschertitel}{Die Kopplung automatisieren}
\renewcommand{\englischertitel}{Automate the Pairing}
\makroabschnitt
\label{AbschnittBindingPhrase1}

To speed up the pairing process between transmitter and receiver and to prevent accidental pairing with other devices, a binding phrase can be defined in the ExpressLRS firmware (see Section~\ref{AbschnittExpressLRS}). This phrase must be identical on both communication partners and cannot be changed without reinstalling (flashing) the firmware.

\colger

Um die Kopplung zwischen Sender und Empfänger zu beschleunigen und ein versehentliches Koppeln mit anderen Geräten zu verhindern, kann in der ExpressLRS-Firmware eine Binding Phrase definiert werden (siehe Abschnitt~\ref{AbschnittExpressLRS}). Diese muss bei beiden Kommunikationspartnern identisch sein und kann nur durch eine Neuinstallation (Flashen) der Firmware geändert werden.

\colende


\renewcommand{\deutschertitel}{Überwachung der Telemetriedaten mit geeigneten Werkzeugen}
\renewcommand{\englischertitel}{Monitoring telemetry data with suitable tools}
\makroabschnitt
\label{AbschnittTelemetriedatenFM2MToolBox}

For convenient real time monitoring of telemetry data various tools are available. One example is the FM2M ToolBox (Fly Me 2 the Moon) by Robert Janiszewski which can be installed on radio transmitters running EdgeTX. The software aggregates sensor data that is relevant for flight preparation and for monitoring the aircraft during flight. This includes flight mode GPS position satellite count speed altitude battery voltage current draw signal quality and additional sensor data such as barometer and magnetic compass readings. Real time monitoring of telemetry data increases flight safety and enables error detection and analysis.

\colger

Zur komfortablen Überwachung der Telemetriedaten in Echtzeit existieren verschiedene Werkzeuge. Ein Beispiel ist die FM2M ToolBox (Fly Me 2 the Moon) von Robert Janiszewski die auf Fernbedienungen mit EdgeTX installiert werden kann. Die Software ermöglicht die Aggregation der Sensordaten die für die Flugvorbereitung und die Überwachung während des Fluges relevant sind. Dazu gehören Flugmodus GPS Position Anzahl der Satelliten Geschwindigkeit Höhe Batteriespannung Stromaufnahme Signalqualität sowie zusätzliche Sensordaten wie Barometer und magnetischer Kompass. Die Echtzeitüberwachung der Telemetriedaten erhöht die Flugsicherheit und ermöglicht Fehlererkennung und Fehleranalyse.

\coleng

To use FM2M or similar scripts the flight controller must provide telemetry data via serial interfaces such as UART and must run compatible firmware. FM2M supports the Betaflight and iNAV firmwares and is available for radio transmitters with color or monochrome displays.

\colger

Voraussetzung für die Nutzung von FM2M oder vergleichbarer Skripte ist dass der Flugcontroller Telemetriedaten über serielle Schnittstellen wie UART bereitstellt und kompatible Firmware verwendet. FM2M unterstützt die Firmwares Betaflight und iNAV und ist für Fernbedienungen mit Farb und Schwarz Weiß Display erhältlich.

\colende


% \renewcommand{\deutschertitel}{Autopilot}
\renewcommand{\englischertitel}{Auto Pilot}
\chapter[\protect{\vspace{2pt}\englischertitel}]{}
\kapitel{\deutschertitel}

\label{KapitelAutopilot}

\begin{paracol}{2}[]

{\raggedright\huge\bfseries\sffamily \englischertitel \par\ } \\[1.8ex]

\switchcolumn

{\raggedright\huge\bfseries\sffamily \deutschertitel \par\ } \\[1.8ex]

\coleng

TBD

\colger

TBD

\colende






\backmatter 

% \include{Z_0_004_TechnicalTerms}
\include{glossary}



% \cleardoublepage
\phantomsection



\apptocmd{\thebibliography}{\csname phantomsection\endcsname
\renewcommand{\bibname}{References}\addcontentsline{toc}{chapter}{References}\addcontentsline{deutschestoc}{chapter}{Literatur}}{}{}

\include{bibliography}


\phantomsection
\cleardoublepage

\addcontentsline{toc}{chapter}{Index}
\addcontentsline{deutschestoc}{chapter}{Index}

\printindex
% \addcontentsline{deutschestoc}{chapter}{Stichwortverzeichnis}
% \printindex[de]
\end{document}
