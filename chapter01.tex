\renewcommand{\deutschertitel}{Hardware-Komponenten zum Bau von FPV-Drohnen}
\renewcommand{\englischertitel}{Hardware Components for Building FPV Drones}

% Das vspace fügt im Inhaltsverzeichnis einen kleinen Abstand unter dem Kapiteleintrag ein.
% Beim deutschen Inhaltsverzeichnis ist es im book.tex an nur einer Stelle in der Zeile 
% \addcontentsline{deutschestoc}{chapter}{\protect{\vspace{2pt}\thechapter}~#1}}
\chapter[\protect{\vspace{2pt}\englischertitel}]{}
\kapitel{\deutschertitel}

\label{KapitelHardware}

\begin{paracol}{2}[]

{\raggedright\huge\bfseries\sffamily \englischertitel \par\ } \\[1.8ex]

\switchcolumn

{\raggedright\huge\bfseries\sffamily \deutschertitel \par\ } \\[1.8ex]

\coleng

This chapter presents the most important hardware components required to build FPV drones. It does not aim to provide a comprehensive overview of the current state of technology. Likewise, recent developments in FPV drone technology are intentionally not addressed.


\colger

Dieses Kapitel stellt die wichtigsten Hardware-Komponenten zum Bau von FPV-Drohnen vor. Es erhebt nicht den Anspruch, einen vollständigen Überblick über den Stand der Technik zu geben. Auch die Entwicklung der FPV-Drohnen in den letzten Jahren wird bewusst ausgeklammert.

\coleng

The goal of this chapter is to provide beginners with an easily understandable introduction to FPV drones. It aims to enable them to identify the components needed to design and build drones for their own projects. This forms the basis for selecting and installing the necessary software (see Chapter~\ref{KapitelSoftware}) and implementing AI functionality using (see Chapter~\ref{KapitelObjekterkennung}, \ref{KapitelAutopilot}, and \ref{KapitelFollowme}).

\colger

Ziel dieses Kapitels ist es, Einsteigern einen leicht verständlichen Zugang zum Thema FPV-Drohnen zu ermöglichen. Sie sollen in der Lage sein, die für ihre Projekte benötigten Komponenten zu identifizieren, um eigene Drohnen zu entwerfen und zu bauen. Dies ist die Voraussetzung für die Auswahl und Installation der notwendigen Software (siehe Kapitel~\ref{KapitelSoftware}) sowie die Realisierung KI-basierter Funktionen mit (siehe Kapitel~\ref{KapitelObjekterkennung}, \ref{KapitelAutopilot} und \ref{KapitelFollowme}).

\coleng

The typical components of an FPV drone without AI extensions are shown in Figure~\ref{AbbildungKompoentenEinerDrohneOhneKI}.

\colger

Die typischen Komponenten einer FPV-Drohne ohne KI-Erweiterung sind in Abbildung~\ref{AbbildungKompoentenEinerDrohneOhneKI} dargestellt.

\colende

\begin{figure}[htb]
  \centering
    \includegraphics[width=\linewidth]{Komponenten_einer_FPV_Drohne_ohne_KI_v1_en.pdf}
  \caption{Components of a FPV Drone)}
  \label{AbbildungKompoentenEinerDrohneOhneKI}
\end{figure}

\renewcommand{\deutschertitel}{Rahmen}
\renewcommand{\englischertitel}{Frames}
\makroabschnitt
\label{AbschnittFrames}

\coleng

The frame connects all components of the drone. It is usually made of carbon fiber, a lightweight yet highly rigid and strong composite material. Frames made of plastic are less common. The frame determines the propeller size (see Section~\ref{AbschnittPropeller}).

\colger

Der Rahmen verbindet alle Komponenten der Drohne miteinander. Das verwendete Material ist üblicherweise Carbon, ein leichtes, aber äußerst stabiles und verwindungssteifes Verbundmaterial aus Kohlenstofffasern. Seltener kommen Rahmen aus Kunststoff zum Einsatz. Der Rahmen bestimmt die Propellergröße (siehe Abschnitt~\ref{AbschnittPropeller}).

\coleng

\coleng

The main electronic components such as the flight controller, video transmitter, receiver, and camera are mounted centrally on the frame for protection. The battery is typically placed on top of the drone to prevent damage during landings.

\colger

Die wichtigsten elektronischen Komponenten wie Flugcontroller, Videosender, Empfänger und Kamera werden zentral auf dem Rahmen montiert, um sie zu schützen. Der Akku befindet sich in den meisten Fällen oben auf der Drohne, um Beschädigungen beim Landen zu vermeiden.

\coleng

Larger frames provide more space for components and allow the use of more powerful motors (see Section~\ref{AbschnittMotoren}) and larger propellers. However, frame size also increases the overall weight. Common frame sizes and their typical applications are listed in Table~\ref{TabelleRahmen}.

\colger

Größere Rahmen bieten mehr Platz für Komponenten und ermöglichen die Nutzung leistungsstärkerer Motoren (siehe Abschnitt~\ref{AbschnittMotoren}) und größerer Propeller. Mit zunehmender Rahmengröße steigt jedoch auch das Gesamtgewicht. Gängige Rahmengrößen und deren typische Anwendungen sind in Tabelle~\ref{TabelleRahmen} aufgeführt.

\colende

\begin{table}[htb!]
\centering
\captionabove{Overview of different Frame Sizes and Drone Categories}
\label{TabelleRahmen}
\begin{tabular}{l@{\hskip 8mm}r@{\hskip 8mm}r@{\hskip 8mm}}
\toprule
Drone Category         & Typical Frame Size & Typical Propeller Diameter \\
\midrule
TinyWhoop              & 1.2 - 2.5 Inch     & 31 - 64\,mm   \\
Toothpick              & 2   - 3 Inch       & 51 - 76\,mm   \\
CineWhoop              & 2.5 - 3.5 Inch     & 64 - 90\,mm   \\
Freestyle              & 3 - 6 Inch         & 76 - 152\,mm  \\
Racing                 & 5 Inch             & 127\,mm       \\
Long-Range, Cinelifter & 4 - 10 Inch        & 100 - 254\,mm \\
Heavy-Lift             & 10 - 12 Inch       & 254 - 304\,mm \\
\bottomrule 
\end{tabular}
\end{table}

\colstart

Important distinguishing criteria when selecting an appropriate frame include the hole spacing for mounting the flight controller and the video transmitter. Common dimensions are:

\colger

Wichtige Unterscheidungskriterien bei der Auswahl des passenden Rahmens sind auch die Abstände der Bohrlöcher zur Befestigung des Flugcontrollers und des Videosenders. Gängige Maße sind:

\coleng

\begin{itemize}
\item 30.5 \(\times\) 30.5\,mm
\item 25.5 \(\times\) 25.5\,mm
\item 20 \(\times\) 20\,mm
\end{itemize}

\colger

\begin{itemize}
\item 30,5 \(\times\) 30,5\,mm
\item 25,5 \(\times\) 25,5\,mm
\item 20 \(\times\) 20\,mm
\end{itemize}

\coleng

If a frame does not provide matching mounting holes for the selected flight controller and video transmitter, a 3D-printed adapter can help — provided there is sufficient space within the frame.

\colger

Verfügt ein Rahmen nicht über passende Bohrlöcher für den ausgewählten Flugcontroller und den Videosender, kann ein per 3D-Drucker hergestellter Adapter helfen, sofern im Rahmen genügend Platz dafür vorhanden ist.

\coleng

Another distinguishing feature of frames is their geometry. The most straightforward design is the True-X shape, where all arms are of equal length. Another variant is the Squashed-X shape, in which the arms are slightly angled to improve flight characteristics. The asymmetrical Deadcat design positions the front arms farther outward and slightly backward so that the propellers do not enter the camera’s field of view. Other variants, such as the H-frame and box frame, aim to increase frame stability through additional reinforcements.

\colger

Ein weiteres Unterscheidungsmerkmal von Rahmen ist deren Geometrie. Die naheliegendste Form ist die True-X-Form, bei der alle Arme gleich lang sind. Eine weitere Variante ist die Squashed-X-Form (\textsl{gequetschtes X}), bei der die Arme leicht abgewinkelt sind, was sich positiv auf die Flugeigenschaften auswirken soll. Bei der asymmetrischen Form Deadcat sind die vorderen Arme weiter außen und nach hinten versetzt, damit die Propeller nicht in das Sichtfeld der Kamera hineinragen. Weitere Varianten sind unter anderem H-Frame und Box, die die Stabilität des Rahmens durch zusätzliche Verstrebungen verbessern sollen.

\coleng

If drones are to be flown indoors or near people, propeller guards are highly recommended. They protect not only the propellers but also the surroundings. Drones intended for racing, freestyle, or long-range applications typically have no propeller guards, as they increase the overall weight. For TinyWhoops and CineWhoops, however, propeller protection is standard.

\colger

Wenn Drohnen in Innenräumen oder in der Nähe von Personen geflogen werden sollen, ist ein Propellerschutz sehr empfehlenswert. Er schützt nicht nur die Propeller, sondern auch die Umgebung. Drohnen für Racing-, Freestyle- und Long-Range-Anwendungen haben normalerweise keinen Propellerschutz, da er das Gesamtgewicht der Drohne erhöht. Bei TinyWhoops und CineWhoops ist ein Propellerschutz hingegen Standard.

\colende

\renewcommand{\deutschertitel}{Flugcontroller und Motorsteuerung}
\renewcommand{\englischertitel}{Flight Controller and Electronic Speed Controller}
\makroabschnitt
\label{AbschnittFC}

The flight controller (FC) is the central component of every drone. It processes sensor data from the gyroscope, accelerometer, barometer, and GPS module. Using control algorithms, it stabilizes the drone’s attitude and receives pilot commands through the receiver. In addition, the FC controls the motors via the electronic speed controllers (ESC).

\colger

Der Flight Controller (FC) ist die zentrale Komponente jeder Drohne. Der FC verarbeitet die Sensordaten von Gyroskop, Beschleunigungssensor, Barometer und GPS-Modul. Er stabilisiert die Fluglage durch Regelalgorithmen und empfängt über den Empfänger die Steuerbefehle des Piloten. Zudem steuert der FC die Motoren über die Electronic Speed Controller (ESC).

\coleng

The FPV camera and video transmitter are also connected to the FC, which supplies them with power from the battery and controls their operation. The FC additionally transmits telemetry data as an on-screen display (OSD) to the video transmitter.

\colger

Auch die FPV-Kamera und der Videosender sind mit dem FC verbunden. Er versorgt sie mit Strom aus dem Akku und steuert ihre Funktionen. Zudem überträgt der FC Telemetriedaten als On-Screen-Display (OSD) an den Videosender.

\coleng

Common flight controllers use an STM32 microcontroller, which also stores the firmware. Modern MCUs include the F4, F7, and H7 series. These differ in clock speed and memory capacity (see Table~\ref{TabelleFCMCU}). The available processing power and memory limit which firmware versions can be used effectively.

\colger

Gängige Flight Controller verwenden einen STM32-Mikrocontroller, auf dem auch die Firmware gespeichert ist. Moderne MCUs sind die Serien F4, F7 und H7. Sie unterscheiden sich in ihrer Taktfrequenz und Speicherkapazität (siehe Tabelle~\ref{TabelleFCMCU}). Die vorhandenen Rechen- und Speicherressourcen begrenzen, welche Firmware-Versionen sinnvoll eingesetzt werden können.

\colende

\begin{table}[htb!]
\centering
\captionabove{Overview of modern Flight Controller STM32 MCUs}
\label{TabelleFCMCU}
\begin{tabular}{c@{\hskip 8mm}l@{\hskip 8mm}r@{\hskip 8mm}r}
\toprule
CPU (MCU)  & Clock    & Flash Memory & RAM     \\
\midrule
F405       & 168\,MHz & 1\,MB        & 192\,kB \\
F411       & 100\,MHz & 512\,kB      & 128\,kB \\
F745       & 216\,MHz & 1\,MB        & 320\,kB \\
F722       & 216\,MHz & 512\,kB      & 256\,kB \\
H743       & 480\,MHz & 2\,MB        & 1\,kB   \\
\bottomrule
\end{tabular}
\end{table}

\colstart

Betaflight (see Section~\ref{AbschnittBetaflight}) runs on all, and INAV (see Section~\ref{AbschnittINAV}) on almost all of the microcontrollers listed in Table~\ref{TabelleFCMCU}. ArduPilot (see Section~\ref{AbschnittArduPilot}) requires at least 1 MB of flash memory for limited operation. The full range of functions is available only with 2 MB of flash memory.

\colger

Betaflight (siehe Abschnitt~\ref{AbschnittBetaflight}) funktioniert auf allen, und INAV (siehe Abschnitt~\ref{AbschnittINAV}) auf fast allen in Tabelle~\ref{TabelleFCMCU} genannten Mikrocontrollern. Der Betrieb von ArduPilot (siehe Abschnitt~\ref{AbschnittArduPilot}) erfordert für einen eingeschränkten Betrieb mindestens 1 MB Flash-Speicher. Erst mit 2 MB Flash-Speicher steht der vollständige Funktionsumfang zur Verfügung.

\coleng

The electronic speed controllers (ESCs) control the motors. The ESC is one of the most heavily stressed components of a drone, as it often handles continuous currents of 10 to 20 amperes.

\colger

Die Electronic Speed Controller (ESCs) steuern die Motoren. Der ESC ist eines der am stärksten belasteten Bauteile der Drohne, da durch ihn häufig Dauerströme von 10 bis 20 Ampere fließen.

\coleng

Separate ESCs mounted as individual boards are now uncommon. In modern drones, four or more ESCs are combined on a single board—known as a 4-in-1 ESC—for motor control. The motors and the power supply connection to the battery (see Section~\ref{AbschnittAkkus}) are soldered directly to this board.

\colger

Separate ESCs als einzelne Platinen sind heute unüblich. Bei modernen Drohnen sind die vier oder mehr ESCs auf einer einzigen Platine -- einem sogenannten 4-in-1-ESC -- zur Motorsteuerung integriert. An dieser Platine werden die Motoren sowie die Verbindung zum Akku (siehe Abschnitt~\ref{AbschnittAkkus}) angelötet.

\colende

\renewcommand{\deutschertitel}{Stack oder AIO}
\renewcommand{\englischertitel}{Stack or AIO}
\makrounterabschnitt
\label{AbschnittESC}

The combination of a separate board for motor control and a flight controller is called a stack.

\colger

Die Kombination einer separaten Platine zur Motorsteuerung und dem Flugcontroller heißt Stack.

\coleng

If the ESCs are located on the same board as the flight controller, the unit is referred to as an All-In-One Flight Controller (AIO FC).

\colger

Befinden sich die ESCs auf der gleichen Platine wie der Flugcontroller, spricht man von einem All-In-One Flight Controller (AIO FC).

\coleng

In smaller drones (2 to 4 inches), AIO FCs are commonly used due to limited space. Larger drones typically use stacks. One advantage of stacks is that separate motor controllers with higher continuous current ratings are available on the market—typically between 45 and 70 A. In addition, FCs used in stacks generally offer more soldering pads, larger pads, often additional connectors, and more usable UART interfaces for sensors and actuators compared to AIO FCs.

\colger

Bei kleineren Drohnen (2 bis 4 Zoll) sind schon aus Platzgründen AIO FCs üblich. Bei größeren Drohnen kommen meist Stacks zum Einsatz. Ein Vorteil von Stacks ist, dass separate Motorsteuerungen mit höheren Dauerströmen am Markt verfügbar sind -- typischerweise zwischen 45 und 70 A. Zudem bieten die FCs von Stacks tendenziell mehr Platz für Lötpads, größere Lötpads, häufig zusätzliche Steckverbindungen und mehr nutzbare UART-Schnittstellen für Sensoren und Aktoren als AIO FCs.

\colende

\begin{figure}[htb]
  \centering
  \begin{minipage}[t]{0.48\textwidth}
    \centering
    \includegraphics[width=\linewidth]{SpeedyBee_F405_AIO_40A_Bluejay_vorderseite_crop.jpg}
    \vspace{0pt} % sorgt für Top-Ausrichtung
  \end{minipage}\hfill
  \begin{minipage}[t]{0.48\textwidth}
    \centering
    \includegraphics[width=\linewidth]{SpeedyBee_F405_AIO_40A_Bluejay_rueckseite_crop.jpg}
    \vspace{0pt}
  \end{minipage}
  \caption{Flight Controller SpeedyBee F405 AIO 40A Bluejay 25.5x25.5 (Front and Back)}
  \label{AbbildungSpeedyBeeFCBluejayAIO}
\end{figure}

\colstart

Figure~\ref{AbbildungSpeedyBeeFCBluejayAIO} shows an AIO flight controller. Figures~\ref{AbbildungSpeedyBeeFCF405MiniBLS35front} and~\ref{AbbildungSpeedyBeeFCF405MiniBLS35Aback} show the front and back sides of a stack consisting of a flight controller and a separate ESC board.

\colger

Abbildung~\ref{AbbildungSpeedyBeeFCBluejayAIO} zeigt einen AIO-Flight-Controller. Die Abbildungen~\ref{AbbildungSpeedyBeeFCF405MiniBLS35front} und~\ref{AbbildungSpeedyBeeFCF405MiniBLS35Aback} zeigen die Vorder- und Rückseite eines Stacks, bestehend aus Flight-Controller und separater ESC-Platine.

\colende

\begin{figure}[htb]
  \centering
    \includegraphics[width=\linewidth]{SpeedyBee_F405_Mini_BLS_35A_front.jpg}
  \caption{Flight Controller SpeedyBee F405 Mini and BSC 35A 20x20 Stack (Front)}
  \label{AbbildungSpeedyBeeFCF405MiniBLS35front}
\end{figure}

\begin{figure}[htb]
  \centering
    \includegraphics[width=\linewidth]{SpeedyBee_F405_Mini_BLS_35A_back.jpg}
  \caption{Flight Controller SpeedyBee F405 Mini and BSC 35A 20x20 Stack (Back)}
  \label{AbbildungSpeedyBeeFCF405MiniBLS35Aback}
\end{figure}

\colstart

An important selection criterion for flight controllers is the number of usable UART interfaces for connecting sensors and actuators. The flight controller shown in Figure~\ref{AbbildungSpeedyBeeFCBluejayAIO}, the SpeedyBee F405 AIO, has six UART interfaces. However, one is used for Wi-Fi and USB administration, and another operates only unidirectionally (simplex), meaning it can only receive data. This leaves only four fully functional (bidirectional) UART interfaces, of which three are typically occupied by the video transmitter, receiver, and GPS module. Therefore, only one free UART interface usually remains.

\colger

Ein wichtiges Auswahlkriterium für Flugcontroller ist die Anzahl der nutzbaren UART-Schnittstellen zum Anschluss von Sensoren und Aktoren. Der in Abbildung~\ref{AbbildungSpeedyBeeFCBluejayAIO} gezeigte Flugcontroller SpeedyBee F405 AIO verfügt über sechs UART-Schnittstellen. Allerdings wird eine für die WLAN- und USB-Schnittstellen zur Administration benötigt und eine weitere funktioniert nur unidirektional (Simplex), das heißt, sie kann nur Daten empfangen. Damit stehen lediglich vier vollwertige (bidirektionale) UART-Schnittstellen zur Verfügung, von denen üblicherweise drei durch die Komponenten Videosender, Empfänger und GPS-Modul belegt sind. Somit bleibt meist nur eine einzige freie UART-Schnittstelle.

\coleng

The flight controller shown in Figures~\ref{AbbildungSpeedyBeeFCF405MiniBLS35front} and \ref{AbbildungSpeedyBeeFCF405MiniBLS35Aback}, the SpeedyBee F405 Mini 35A Stack, also has six UART interfaces.  Here, too, after connecting the commonly used components — video transmitter, receiver, and GPS module — only one interface remains available. This is because one UART is used by the Bluetooth interface for administration, and another is used by the ESC for telemetry data transmission.

\colger

Auch der in den Abbildungen~\ref{AbbildungSpeedyBeeFCF405MiniBLS35front} und \ref{AbbildungSpeedyBeeFCF405MiniBLS35Aback} gezeigte Flugcontroller SpeedyBee F405 Mini 35A Stack verfügt über sechs UART-Schnittstellen. Auch hier bleibt nach dem Anschluss der üblichen Komponenten -- Videosender, Empfänger und GPS-Modul -- nur eine Schnittstelle frei verfügbar, da die Bluetooth-Schnittstelle zur Administration eine UART-Schnittstelle belegt und die Motorsteuerung eine weitere UART-Schnittstelle zur Übertragung der Telemetriedaten nutzt.

\coleng

For comparison, the flight controller shown in Figure~\ref{AbbildungFlywooFCGOKUGN745AIO}, the Flywoo GOKU GN745 AIO 45A, offers seven UART interfaces. None of them is used for USB, but one is reserved for telemetry data transmission. After connecting the common components — video transmitter, receiver, and GPS module — three UART interfaces remain available for additional sensors and actuators.

\colger

Zum Vergleich: Der in Abbildung~\ref{AbbildungFlywooFCGOKUGN745AIO} gezeigte Flugcontroller Flywoo GOKU GN745 AIO 45A verfügt über sieben UART-Schnittstellen. Keine davon ist für die USB-Schnittstelle belegt, jedoch wird eine für die Übertragung von Telemetriedaten verwendet. Nach dem Anschluss der üblichen Komponenten -- Videosender, Empfänger und GPS-Modul -- bleiben somit drei UART-Schnittstellen für weitere Sensoren und Aktoren verfügbar.

\colende

\begin{figure}[htb]
  \centering
  \begin{minipage}[t]{0.48\textwidth}
    \centering
    \includegraphics[width=\linewidth]{FLYWOO_GOKU_GN745_45A_AIO_V3_FC_front.jpg}
    \vspace{0pt} % sorgt für Top-Ausrichtung
  \end{minipage}\hfill
  \begin{minipage}[t]{0.48\textwidth}
    \centering
    \includegraphics[width=\linewidth]{FLYWOO_GOKU_GN745_45A_AIO_V3_FC_back.jpg}
    \vspace{0pt}
  \end{minipage}
  \caption{Flight Controller Flywoo GOKU GN745 AIO 45A 25.5x25.5 (Front and Back)}
  \label{AbbildungFlywooFCGOKUGN745AIO}
\end{figure}

\colstart

Table~\ref{TabelleUARTuebersicht} shows the typical use of the available UART interfaces of selected flight controllers and the resulting limitations regarding their suitability for AI projects, which often require connecting additional sensors and actuators.

\colger

Tabelle~\ref{TabelleUARTuebersicht} zeigt die typische Nutzung der verfügbaren UART-Schnittstellen einiger ausgewählter Flugcontroller und die damit einhergehenden Einschränkungen hinsichtlich ihrer Eignung für KI-Projekte, die häufig den Anschluss weiterer Sensoren und Aktoren erfordern.

\colende

\begin{table}[htb!]
\centering
\setlength{\tabcolsep}{4pt}       % Default value: 6pt
\renewcommand{\arraystretch}{1.5} % Default value: 1
\begin{threeparttable}
\captionabove{Recommended use and availability of UART Interfaces of some Flight Controllers}
\scriptsize
\label{TabelleUARTuebersicht}
\begin{tabularx}{\textwidth}{lYYYYYYYY}
\toprule
Flight Controller              & UART1 &  UART2 & UART3 & UART4 & UART5 & UART6 & UART7 & UART8  \\
\midrule
SpeedyBee F405 AIO             & \cellcolor{Gray}MSP\textsuperscript{a}  & \cellcolor{yellow}available\textsuperscript{b} & \cellcolor{Gray}VTX       & \cellcolor{green}available & \cellcolor{Gray}GPS  & \cellcolor{Gray}Receiver & --- & --- \\
SpeedyBee F405 Mini Stack      & \cellcolor{Gray}VTX   & \cellcolor{Gray}Receiver        & \cellcolor{green}available & \cellcolor{Gray}MSP\textsuperscript{a} & \cellcolor{Gray}Telem.\textsuperscript{c} & \cellcolor{Gray}GPS & --- & ---     \\
SpeedyBee F4V3 / F4V4 Stack           & \cellcolor{Gray}VTX   & \cellcolor{Gray}Receiver        & \cellcolor{green}available & \cellcolor{Gray}MSP\textsuperscript{a} & \cellcolor{Gray}Telem.\textsuperscript{c} & \cellcolor{Gray}GPS & --- & ---     \\
Flywoo F722 PRO MINI V2        & \cellcolor{yellow}available\textsuperscript{b} & \cellcolor{Gray}Receiver & \cellcolor{Gray}VTX & \cellcolor{green}available & \cellcolor{Gray}GPS & \cellcolor{Gray}Telem.\textsuperscript{c} & --- & --- \\
Flywoo F722 PRO V2              & \cellcolor{yellow}available\textsuperscript{b} & \cellcolor{Gray}Receiver & \cellcolor{Gray}VTX & \cellcolor{green}available & \cellcolor{Gray}GPS & \cellcolor{Gray}Telem.\textsuperscript{c} & --- & --- \\
 
GEPRC F722 35A AIO             & \cellcolor{Gray}VTX   & \cellcolor{Gray}Receiver  & \cellcolor{Gray}GPS & \cellcolor{green}available & \cellcolor{green}available & --- &   ---   & ---    \\
SpeedyBee F7V3 Stack           & \cellcolor{Gray}VTX   & \cellcolor{Gray}Receiver        & \cellcolor{green}available & \cellcolor{Gray}Telem.\textsuperscript{c} & --- &  \cellcolor{Gray}GPS & ---   & ---    \\
Flywoo GN745 AIO V3            & \cellcolor{green}available & \cellcolor{Gray}Receiver & \cellcolor{Gray}VTX       & \cellcolor{green}available  & \cellcolor{green}available  & \cellcolor{Gray}GPS & \cellcolor{Gray}Telem.\textsuperscript{c} & --- \\
Axisflying F745 40A AIO AM32 & \cellcolor{Gray}VTX & \cellcolor{Gray}Receiver & \cellcolor{Gray}GPS & ? & \cellcolor{green}available & \cellcolor{green}available & \cellcolor{green}available & \cellcolor{green}available \\
MicoAir H743 V2 45A AIO    & \cellcolor{green}available & \cellcolor{Gray}VTX & \cellcolor{Gray}GPS & \cellcolor{green}available & \cellcolor{green}available & \cellcolor{Gray}Receiver & \cellcolor{Gray}Telem.\textsuperscript{c} & \cellcolor{green}available \\
T-Motor PACER H743 Stack    & \cellcolor{green}available  & \cellcolor{Gray}GPS & \cellcolor{green}available & \cellcolor{Gray}Telem.\textsuperscript{c} & \cellcolor{Gray}Receiver & \cellcolor{Gray}VTX &  \cellcolor{Gray}MSP\textsuperscript{a} & \cellcolor{green}available \\
\bottomrule
\end{tabularx}
\begin{tablenotes}
\footnotesize
\item[a] MSP = Port for administration via the MultiWii Serial Protocol using Bluetooth and WIFI if available.
\item[b] Offers unidirectional (simplex) communication. It can only receive data.
\item[c] Receives telemetry data from the ESC.
\end{tablenotes}
\end{threeparttable}
\end{table}


\renewcommand{\deutschertitel}{Motoren}
\renewcommand{\englischertitel}{Motors}
\makroabschnitt
\label{AbschnittMotoren}

Motors differ in their design (construction and size), the operating voltage (in volts), rotational speed (KV rating), propeller mount, and the method of attachment to the frame.

\colger

Motoren unterscheiden sich in ihrem Aufbau (Konstruktion und Größe), der elektrischen Spannung (in Volt), mit der sie betrieben werden, der Drehzahl (KV-Wert), der Aufnahme des Propellers und der Befestigung am Rahmen.

\colende

\renewcommand{\deutschertitel}{Motoraufbau und -größe}
\renewcommand{\englischertitel}{Size of the Motors}
\makrounterabschnitt
\label{AbschnittMotorenSize}

The motors used in FPV drones are three-phase, brushless DC motors. Each motor has three connection wires that are soldered to the electronic speed controller (ESC) or directly to an AIO flight controller. The ESC sends time-shifted alternating currents through these three wires. By switching the current phases, a rotating magnetic field is created, which drives the rotor equipped with permanent magnets. The three wires are equivalent, so the order in which they are soldered to the ESC or flight controller does not matter.  The motor’s rotation direction can be changed by swapping any two of the three wires. Alternatively, the direction can be checked and reversed in the firmware’s graphical interface for each motor individually.

\colger

Die Motoren von FPV-Drohnen sind dreiphasige, bürstenlose Gleichstrommotoren. Jeder Motor besitzt drei Anschlusskabel, die an die Motorsteuerung (ESC) oder direkt an einen AIO-Flugcontroller angelötet werden. Die Motorsteuerung schickt zeitlich versetzte Wechselströme in diese drei Leitungen. Durch das Umschalten der Ströme entsteht ein rotierendes Magnetfeld, das den Rotor mit Permanentmagneten in Bewegung versetzt. Die Leitungen sind gleichwertig -- die Reihenfolge, in der sie angelötet werden, spielt keine Rolle. Die Drehrichtung des Motors kann durch Vertauschen von zwei beliebigen der drei Kabel geändert werden. Alternativ lässt sich die Drehrichtung in der grafischen Benutzeroberfläche der Firmware für jeden Motor separat prüfen und anpassen.

\coleng

The motors consist of a stator and a rotor. The stator contains the windings and the bearing, while the rotor—the motor bell—carries the magnets and performs the actual rotation.  Each motor is identified by a numerical code that specifies the stator’s diameter and height in millimeters. For example, a motor labeled *2306* has a stator with a diameter of 23 mm and a height of 6 mm.

\colger

Die Motoren bestehen aus einem Stator und einem Rotor. Der Stator enthält die Wicklungen und das Kugellager, während der Rotor -- die Motorglocke -- die Magnete trägt und die eigentliche Drehbewegung ausführt. Jeder Motor ist mit einer Zahlenkombination gekennzeichnet, die den Durchmesser und die Höhe des Stators in Millimetern angibt. Ein Motor mit der Bezeichnung *2306* besitzt beispielsweise einen Stator mit 23 mm Durchmesser und 6 mm Höhe.

\coleng

The larger the stator volume, the higher the motor’s torque and overall performance. Larger motors also have better thermal robustness. Smaller motors, by contrast, are lighter and more efficient under low loads.

\colger

Je größer das Statorvolumen, desto höher sind Drehmoment und Gesamtleistung des Motors. Größere Motoren sind zudem thermisch robuster, während kleinere Motoren leichter und bei geringer Last effizienter arbeiten.

\coleng

Propellers (see Section~\ref{AbschnittMotorenPropelleraufnahme}) are typically mounted on an M5 shaft with a 5 mm diameter using a matching nut, or alternatively on a 1.5 mm shaft secured with two small M2 screws.

\colger

Die Propelleraufnahme (siehe Abschnitt~\ref{AbschnittMotorenPropelleraufnahme}) erfolgt in der Regel auf einer M5-Welle mit 5 mm Durchmesser und einer passenden Mutter oder alternativ auf einer 1,5 mm-Welle, die mit zwei kleinen M2-Schrauben am Motor befestigt wird.

\colende

\renewcommand{\deutschertitel}{Elektrische Spannung}
\renewcommand{\englischertitel}{Electrical Voltage}
\makrounterabschnitt
\label{AbschnittMotorenVolt}

Motors and batteries (see Section~\ref{AbschnittAkkus}) must be compatible. 4S motors require 4S batteries with a maximum voltage of 16.8 V, while 6S motors require 6S batteries with a maximum voltage of 25.2 V.

\colger

Motoren und Akkus (siehe Abschnitt~\ref{AbschnittAkkus}) müssen zusammenpassen. 4S-Motoren benötigen 4S-Akkus mit einer maximalen Spannung von 16,8 V, während 6S-Motoren 6S-Akkus mit maximal 25,2 V benötigen.

\coleng

4S components are usually cheaper and lighter but provide less power. 6S components offer more power or longer flight time but are generally heavier -- especially the batteries. For lightweight and cost-effective drones, 4S is often the better choice. Some motors can handle both 4S and 6S voltages, making them more versatile in use.

\colger

Komponenten für 4S sind meist günstiger und leichter, bieten dafür aber weniger Leistung. 6S-Komponenten liefern mehr Leistung oder längere Flugzeiten, sind jedoch in der Regel schwerer -- insbesondere die Akkus. Für leichte und kostengünstige Drohnen ist 4S häufig die bessere Wahl. Es gibt auch Motoren, die sowohl mit 4S- als auch mit 6S-Akkus betrieben werden können und dadurch flexibler einsetzbar sind.

\colende

\renewcommand{\deutschertitel}{KV-Wert}
\renewcommand{\englischertitel}{KV Value}
\makrounterabschnitt
\label{AbschnittMotorenKV}

The KV value describes how fast a motor spins per volt under no load, measured in revolutions per minute (RPM). For example, a 3000 KV motor spins at 3000 RPM per volt. When powered by a 4S battery (see Section~\ref{AbschnittAkkus}) with approximately 16 V, this motor would have a no-load speed of about \(3000 \times 16 \approx 48000\)\,RPM.

\colger

Der KV-Wert beschreibt, wie schnell sich ein Motor im Leerlauf pro Volt dreht, gemessen in Umdrehungen pro Minute (U/min). Ein Motor mit beispielsweise 3000 KV dreht also bei 1 V Versorgungsspannung 3000 U/min. An einem 4S-Akku (siehe Abschnitt~\ref{AbschnittAkkus}) mit etwa 16 V Spannung hat dieser Motor somit eine Leerlaufdrehzahl von ungefähr \(3000 \times 16 \approx 48000\)\,U/min.

\coleng

Motors with high KV values (2800--7000 KV) are more agile and provide faster response times but generate less torque per ampere. They are well suited for smaller propellers and lighter drones. Motors with lower KV values (1500--2450 KV) deliver higher torque per ampere and are better suited for larger propellers and heavier drones.

\colger

Motoren mit hohen KV-Werten (2800--7000 KV) sind agiler und ermöglichen schnellere Reaktionen, erzeugen jedoch ein geringeres Drehmoment pro Ampere. Sie eignen sich gut für kleinere Propeller und leichtere Drohnen. Motoren mit niedrigen KV-Werten (1500--2450 KV) liefern ein höheres Drehmoment pro Ampere und sind für größere Propeller und schwerere Drohnen besser geeignet.

\coleng

A higher KV value does not mean that a motor is more powerful than one with a lower KV value. It simply means the motor spins faster, but also draws more current to produce thrust.

\colger

Ein hoher KV-Wert bedeutet nicht, dass ein Motor stärker ist als ein Motor mit einem niedrigeren KV-Wert. Er dreht lediglich schneller, benötigt dafür aber auch mehr Strom, um denselben Schub zu erzeugen.

\colende

\renewcommand{\deutschertitel}{Propelleraufnahme}
\renewcommand{\englischertitel}{Propeller Mounting}
\makrounterabschnitt
\label{AbschnittMotorenPropelleraufnahme}

Larger motors (e.g., 22xx, 23xx, 24xx) for drones of 5 inches and above typically feature an M5 shaft with a 5 mm diameter for propeller mounting. The propellers are placed directly onto the motor shaft and secured with a nut. Smaller motors (e.g., 13xx, 14xx, 18xx) for drones up to 3.5 inches usually have much smaller shafts of 1.5 mm in diameter, where the propellers are attached with two screws. Motors for medium-sized drones (e.g., 20xx, 21xx) with 3.5 or 4-inch propellers are available in both M5 and 1.5 mm shaft variants.

\colger

Größere Motoren (z. B. 22xx, 23xx, 24xx) für Drohnen ab 5 Zoll besitzen zur Propelleraufnahme meist eine M5-Welle mit 5 mm Durchmesser. Die Propeller werden direkt auf die Motorwelle gesteckt und mit einer Mutter gesichert. Kleinere Motoren (z. B. 13xx, 14xx, 18xx) für Drohnen bis 3,5 Zoll haben in der Regel eine deutlich kleinere Welle mit 1,5 mm Durchmesser, an der die Propeller mit zwei Schrauben befestigt werden. Motoren (z. B. 20xx, 21xx) für mittelgroße Drohnen (3,5 oder 4 Zoll) sind sowohl mit M5-Wellen als auch mit 1,5 mm-Wellen erhältlich.

\colende

\renewcommand{\deutschertitel}{Rahmenbefestigung}
\renewcommand{\englischertitel}{Frame Attachment}
\makrounterabschnitt
\label{AbschnittMotorenRahmenbefestigung}

When selecting motors, it is important to ensure that the mounting holes on the frame match in number, spacing, and diameter. Motors are typically attached to the frame using three or four screws (M1.4, M2, or M3). On smaller frames, the holes are arranged in an equilateral triangle or a square pattern. Common mounting hole distances are 7 mm, 9 mm, 12 mm, 16 mm, or 19 mm.

\colger

Bei der Auswahl der Motoren ist darauf zu achten, dass die Bohrlöcher des Rahmens in Anzahl, Abstand und Durchmesser übereinstimmen. Üblicherweise werden die Motoren mit drei oder vier Schrauben (M1,4, M2 oder M3) am Rahmen befestigt. Bei kleinen Rahmen sind die Schraubenlöcher in Form eines gleichseitigen Dreiecks oder als Quadrat angeordnet. Gängige Lochabstände sind 7 mm, 9 mm, 12 mm, 16 mm oder 19 mm.

\colende

\renewcommand{\deutschertitel}{Propeller}
\renewcommand{\englischertitel}{Propeller}
\makroabschnitt
\label{AbschnittPropeller}

Propellers convert the motor’s rotational movement into thrust (lift) and control forces. Their diameter, pitch, and number of blades influence thrust, efficiency, noise level, and flight behavior. This document considers only drones with four propellers, although other configurations such as hexacopters and octocopters are also possible.

\colger

Die Propeller wandeln die Drehbewegung des Motors in Schub (Lift) und Steuerkräfte um. Propellergröße (Durchmesser), Steigung (Pitch) und Blätterzahl beeinflussen Schub, Effizienz, Geräuschentwicklung und Flugverhalten. Dieses Dokument berücksichtigt ausschließlich Drohnen mit vier Propellern, auch wenn andere Konfigurationen wie Hexacopter und Octocopter ebenfalls möglich sind.

\colende

\renewcommand{\deutschertitel}{Propellergröße (Durchmesser)}
\renewcommand{\englischertitel}{Propeller Size (Diameter)}
\makrounterabschnitt
\label{AbschnittPropellerDurchmesser}

TBD

\colger

The propeller size determines, in simple terms, how much air is moved. The specified propeller size always refers to the diameter in inches. The propellers must fit the frame used (see Section~\ref{AbschnittFrames}).

\colger

Die Propellergröße bestimmt vereinfacht gesagt, wie viel Luft bewegt wird. Die angegebene Propellergröße entspricht immer dem Durchmesser in Zoll. Die Propeller müssen zum verwendeten Rahmen passen (siehe Abschnitt~\ref{AbschnittFrames}).

\coleng

Smaller propellers make the drone more agile, allowing quicker responses to control inputs. This is because smaller propellers have less air resistance and lower rotational mass (moment of inertia). As a result, the motor can increase or decrease its rotational speed faster. Since smaller propellers are typically used with smaller motors that have a higher KV rating, they also spin faster, which further enhances agility and responsiveness.

\colger

Kleinere Propeller machen die Drohne wendiger und ermöglichen schnellere Reaktionen auf Steuerbefehle. Der Grund dafür ist, dass kleinere Propeller weniger Luftwiderstand haben und insgesamt leichter sind. Sie besitzen eine geringere Rotationsmasse (Trägheitsmoment), wodurch der Motor die Drehzahl schneller erhöhen oder verringern kann. Da kleinere Propeller typischerweise mit kleineren Motoren mit höherem KV-Wert kombiniert werden, drehen sie zusätzlich schneller, was die Agilität und das Ansprechverhalten weiter steigert.

\coleng

Larger propellers generate more thrust and are more efficient. They also provide smoother flight characteristics but at the cost of agility. As propeller size increases, so does the load on the motors and ESCs, since the larger blade area produces more air resistance. In addition, larger propellers are heavier.

\colger

Größere Propeller erzeugen mehr Schub und arbeiten effizienter. Sie sorgen zudem für ein ruhigeres Flugverhalten, allerdings auf Kosten der Agilität. Mit zunehmender Propellergröße steigt auch die Belastung für die Motoren und die Motorsteuerung (ESC), da die größere Blattfläche einen höheren Luftwiderstand erzeugt. Außerdem haben größere Propeller ein höheres Eigengewicht.

\colende

\renewcommand{\deutschertitel}{Anzahl der Blätter}
\renewcommand{\englischertitel}{Number of Blades}
\makrounterabschnitt
\label{AbschnittPropellerAnzahlBlaetter}

Propellers can have between two and eight blades. The fewer blades a propeller has, the higher the possible speed and efficiency (thrust per watt or per ampere). With an increasing number of blades, control, smoothness, and acceleration improve -- but power consumption also increases.

\colger

Es gibt Propeller mit zwei bis acht Blättern. Je weniger Blätter ein Propeller hat, desto höher sind die erreichbare Geschwindigkeit und Effizienz (Schub pro Watt oder pro Ampere). Mit steigender Anzahl an Blättern verbessern sich Kontrolle, Laufruhe und Beschleunigung, gleichzeitig steigt jedoch auch der Stromverbrauch.

\coleng

For long-range applications where efficiency is the top priority, two-blade propellers are typically used. Racing and freestyle drones often use three-blade propellers. CineWhoops commonly use propellers with five or more blades, as these provide maximum flight stability and precise control.

\colger

Für Long-Range-Anwendungen, bei denen Effizienz oberste Priorität hat, werden in der Regel Zweiblattpropeller verwendet. Racing- und Freestyle-Drohnen sind häufig mit Dreiblattpropellern ausgestattet. CineWhoops nutzen oft Propeller mit fünf oder mehr Blättern, da sie ein besonders stabiles Flugverhalten und präzise Steuerung ermöglichen.

\colende

\renewcommand{\deutschertitel}{Steigung (Pitch)} 
\renewcommand{\englischertitel}{Pitch}
\makrounterabschnitt
\label{AbschnittPropellerPitch}

In addition to the number of blades, the pitch of the propeller also affects speed and efficiency (power consumption). A low pitch provides greater efficiency when hovering due to lower air resistance and results in smoother flight behavior. A high pitch allows for greater speed and more aggressive flight maneuvers but increases power consumption.

\colger

Neben der Anzahl der Blätter beeinflusst auch die Steigung (Pitch) der Propeller die Geschwindigkeit und Effizienz (Stromverbrauch). Ein niedriger Pitch ist aufgrund des geringeren Luftwiderstands beim Schweben effizienter und führt zu einem ruhigeren Flugverhalten. Ein hoher Pitch ermöglicht höhere Geschwindigkeiten und aggressivere Flugmanöver, steigert jedoch den Stromverbrauch.

\coleng

The pitch is specified in inches and describes the theoretical distance the propeller would move forward during one full rotation if there were no slippage.

\colger

Der Pitch wird in Zoll angegeben und beschreibt den theoretischen Vortrieb pro Umdrehung, also wie weit sich der Propeller bei einer vollständigen Umdrehung durch die Luft schrauben würde, wenn es keine Schlupfverluste gäbe.

\coleng

For example, with a pitch of 4.3 inches per rotation and a motor speed of 3000 RPM, the theoretical distance traveled per minute is \(3000 \times 4.3 = 12{,}900\) inches per minute. Since one inch equals 25.4 mm, the theoretical speed is \(12{,}900 \times 0.0254 \approx 327.7\,\text{m/min}\), or approximately \(327.7 / 60 \approx 5.46\,\text{m/s}\).

\colger

Bei einem Pitch von 4,3 Zoll pro Umdrehung und einer Motordrehzahl von 3000 U/min ergibt sich ein theoretischer Weg von \(3000 \times 4{,}3 = 12{,}900\,\text{Zoll/min}\). Ein Zoll entspricht 25,4 mm. Daraus ergibt sich eine theoretische Geschwindigkeit von \(12{,}900 \times 0{,}0254 \approx 327{,}7\,\text{m/min}\) bzw. \(327{,}7 / 60 \approx 5{,}46\,\text{m/s}\).

\coleng

In practice, these theoretical values are reduced by about 20--30\% due to slippage. This occurs because air is not a solid medium but a fluid one. As air is displaced and swirled, part of it \textsl{slips away}, meaning the propeller generates less thrust in reality than predicted theoretically.

\colger

In der Praxis werden diese theoretischen Werte durch Schlupfverluste um etwa 20--30\% verringert. Der Grund dafür ist, dass Luft kein festes, sondern ein bewegliches Medium ist. Durch das Ausweichen und Verwirbeln \textsl{rutscht} immer ein Teil der Luft weg, wodurch der Propeller in der Realität weniger Vortrieb erzeugt als in der Theorie.

\colende

\renewcommand{\deutschertitel}{Akkus}
\renewcommand{\englischertitel}{Batteries}
\makroabschnitt
\label{AbschnittAkkus}

FPV drones traditionally use lithium-polymer (LiPo) or lithium-ion (Li-Ion) batteries. Batteries are also classified by their capacity (mAh), voltage (number of cells), discharge rate (C rating, or capacity rate), and connector type. Table~\ref{TabelleLiPoLiIonAkkus} provides an overview of the characteristic properties of LiPo and Li-Ion batteries.

\colger

Klassischerweise verwenden FPV-Drohnen Lithium-Polymer- (LiPo) oder Lithium-Ionen-Akkus (Li-Ion). Akkus werden zudem nach ihrer Kapazität (mAh), Spannung (Zellenzahl), Entladerate bzw. C-Wert (\textsl{Capacity Rate}) und dem Steckertyp unterschieden. Eine Übersicht der charakteristischen Eigenschaften von LiPo- und Li-Ion-Akkus zeigt Tabelle~\ref{TabelleLiPoLiIonAkkus}.

\colende

\begin{table}[htb!]
\centering
\captionabove{Comparison of Lithium-Polymere (LiPo) and Lithium-Ion (Li-Ion) Characteristics}
\label{TabelleLiPoLiIonAkkus}
\setlength{\tabcolsep}{3pt} % Default value: 6pt
\renewcommand{\arraystretch}{1.0} % Default
\scriptsize
\begin{tabularx}{\textwidth}{lXX}
\toprule
Feature                         & Li-Ion & LiPo    \\
\midrule
Energy Density (Wh/kg)          & Higher                             & Lower  \\
Max. Discharge Current (C-Rate) & Lower (10-12\,C)                   & Higher (75-120\,C) \\
Weight (per stored energy)      & Lighter                            & Heavier \\   
Lifespan (number of charge cycles) & Longer                          & Shorter \\      
Safety                         & More stable, more robust            & More sensitive to over-/undercharge \\
Use Cases                      & Long-range, smooth flying, cruising & Racing, freestyle, aggressive maneuvers  \\
\bottomrule
\end{tabularx}
\end{table}

\renewcommand{\deutschertitel}{Lade- oder Entladerate (C-Wert)}
\renewcommand{\englischertitel}{Charge and Discharge Rate (C-Rate)}
\makrounterabschnitt
\label{AbschnittCWertAkkus}

Batteries consist of one or more cells connected in series. FPV drones typically use LiPo batteries because they offer very high C-rates, ranging from 75 to 120 C. The C-rate indicates how quickly a battery can be discharged. In contrast, Li-Ion batteries usually have C-rates of only 10--12 C.

\colger

Akkus bestehen aus einer oder mehreren in Reihe geschalteten Zellen. FPV-Drohnen verwenden meist LiPo-Akkus, da diese sehr hohe C-Werte zwischen 75 und 120 C erreichen. Der C-Wert gibt an, wie schnell ein Akku entladen werden kann. Li-Ion-Akkus bieten im Gegensatz dazu nur 10--12 C.

\coleng

For example, a 1500 mAh LiPo battery (see Figure~\ref{AbbildungLiPoAkku}) with a C-rate of 100 can deliver \(1.5 A \times 100 = 150 A\) of continuous current.

\colger

Ein 1500 mAh LiPo-Akku (siehe Abbildung~\ref{AbbildungLiPoAkku}) mit einem C-Wert von 100 kann somit \(1{,}5 A \times 100 = 150 A\) Dauerstrom liefern.

\colende

\begin{figure}[htbp]
  \centering
  \begin{minipage}[t]{0.48\textwidth}
    \centering
    \includegraphics[width=\linewidth]{CNHL_1500mAh_100C_4S_XT60_Lipo_front.jpg}
    \vspace{0pt} % sorgt für Top-Ausrichtung
  \end{minipage}\hfill
  \begin{minipage}[t]{0.48\textwidth}
    \centering
    \includegraphics[width=\linewidth]{CNHL_1500mAh_100C_4S_XT60_Lipo_back.jpg}
    \vspace{0pt}
  \end{minipage}
  \caption{A 4S Lithium-Polymere (LiPo) Battery with 1500\,mAh Capacity and 100\,C Capacity Rate}
  \label{AbbildungLiPoAkku}
\end{figure}

\colstart

Such high current levels are only needed for very dynamic flight maneuvers. For typical use in education and research projects, this is not relevant. Patrol flights or hovering for data collection do not require high C-rates.

\colger

Solch hohe Ströme werden nur für sehr dynamische Flugmanöver benötigt. Für typische Anwendungen in Lehre und Forschungsprojekten ist das jedoch uninteressant. Patrouillenflüge oder das Verharren an einer Position zur Datensammlung erfordern keine hohen C-Werte.

\coleng

LiPo batteries have several disadvantages. They can be easily over-discharged, which permanently damages or destroys them. Crashes often lead to physical damage, and improper handling — such as charging too quickly or using a damaged battery — carries a latent risk of fire. A much more robust alternative to LiPo batteries are lithium-ion batteries, although they provide much lower C-rates.

\colger

LiPo-Akkus haben mehrere Nachteile. Sie können leicht tiefentladen werden, was sie dauerhaft beschädigt oder zerstört. Abstürze führen häufig zu Beschädigungen, und bei unsachgemäßer Handhabung -- etwa durch zu schnelles Laden oder den Betrieb trotz beschädigter Außenhülle -- besteht ein latentes Brandrisiko. Eine deutlich robustere Alternative zu LiPo-Akkus sind Lithium-Ionen-Akkus, die allerdings wesentlich geringere C-Werte bieten.

\coleng

Li-Ion batteries are ideal for scenarios where long flight times, smooth flight behavior, and extended range are desired. LiPo batteries, on the other hand, are better suited for fast flight maneuvers and racing applications.

\colger

Li-Ion-Akkus sind ideal für Szenarien, in denen lange Flugzeiten, ruhiges Flugverhalten und gegebenenfalls große Reichweiten (Long-Range) angestrebt werden. LiPo-Akkus hingegen eignen sich besser für schnelle Flugmanöver und Rennanwendungen.

\coleng

The C-rate also indicates the recommended charging rate. Unless the manufacturer explicitly specifies a higher value, the standard charging rate is 1 C. For the 1500 mAh LiPo battery mentioned earlier, a 1 C charging rate corresponds to a current of 1.5 A, meaning that a full charge theoretically takes one hour. However, this is a hypothetical value, as a deeply discharged battery can no longer be recharged.

\colger

Der C-Wert gibt auch die empfohlene Laderate an. Diese beträgt 1 C, sofern der Hersteller nicht ausdrücklich eine höhere Laderate, wie etwa 2 C, freigegeben hat. Beim eingangs erwähnten 1500 mAh LiPo-Akku entspricht eine Laderate von 1 C einem Strom von 1,5 A, wodurch das vollständige Laden theoretisch eine Stunde dauern würde. Dies ist jedoch ein hypothetischer Wert, da ein derart tiefentladener Akku nicht mehr geladen werden kann.

\colende

\renewcommand{\deutschertitel}{Anzahl der Zellen (S4/S6)}
\renewcommand{\englischertitel}{Number of Cells (S4/S6)}
\makrounterabschnitt
\label{AbschnittSWertAkkusZellen}

LiPo and Li-Ion batteries contain one or more cells connected in series. The so-called S rating of a battery indicates the number of cells it contains. Each LiPo cell has a nominal voltage of approximately 3.7 V (fully charged around 4.2 V, discharged around 3.0 V). When multiple cells are connected in series, their voltages add up.

\colger

LiPo-Akkus und Li-Ion-Akkus enthalten eine oder mehrere in Reihe geschaltete Zellen. Der sogenannte S-Wert eines Akkus gibt die Anzahl der enthaltenen Zellen an. Jede LiPo-Zelle hat eine Nennspannung von etwa 3,7 V (voll geladen etwa 4,2 V, entladen etwa 3,0 V). Werden mehrere Zellen in Reihe geschaltet, addieren sich die Spannungen.

\coleng

The most common battery types are 4S and 6S. A 4S battery has four cells connected in series, resulting in a nominal voltage of \(4 \times 3.7\,\text{V} = 14.8\,\text{V}\). When fully charged, its voltage is 16.8 V. Table~\ref{TabelleAkkus} provides an overview of different LiPo battery configurations.

\colger

Die gängigsten Akkutypen sind 4S und 6S. Bei einem 4S-Akku sind vier Zellen in Reihe geschaltet, wodurch sich eine Nennspannung von \(4 \times 3,7\,\text{V} = 14,8\,\text{V}\) ergibt. Ist ein solcher Akku voll geladen, beträgt seine Spannung 16,8 V. Eine Übersicht über verschiedene LiPo-Akkus zeigt Tabelle~\ref{TabelleAkkus}.

\colende

\begin{table}[htb!]
\centering
\captionabove{Overview of Lithium-Polymere (LiPo) and Lithium-Ion (Li-Ion) Batteries}
\label{TabelleAkkus}
\begin{tabular}{l@{\hskip 8mm}l@{\hskip 8mm}r@{\hskip 8mm}r@{\hskip 8mm}r}
\toprule
Battery Type  & Cells & Nominal Voltage  & Fully Charged & Discharged   \\
\midrule
S1            & 1     & 3.7\,V           & 4,2,V        & 3.0\,V         \\
S2            & 2     & 7.4\,V           & 8.4,V        & 6.0\,V         \\
S3            & 3     & 11.1\,V          & 12.6,V       & 9.0\,V         \\
S4            & 4     & 14.8\,V          & 16.8,V       & 12.0\,V        \\
S5            & 5     & 18.5\,V          & 21.0,V       & 15.0\,V        \\
S6            & 6     & 22.2\,V          & 25.2,V       & 18.0\,V        \\
\bottomrule
\end{tabular}
\end{table}

\colstart

At higher voltages, a motor requires less current for the same power output and operates more efficiently. 6S components are more powerful but also heavier and more expensive. Larger or performance-oriented drones usually use 6S batteries, while smaller and medium-sized drones typically use 4S batteries and correspondingly lightweight components.

\colger

Bei höherer Spannung benötigt ein Motor bei gleicher Leistung weniger Strom und arbeitet effizienter. 6S-Komponenten sind leistungsstärker, jedoch auch schwerer und teurer. Größere oder leistungsorientierte Drohnen verwenden meist 6S-Akkus, während kleinere und mittlere Drohnen in der Regel 4S-Akkus und entsprechend leichte Komponenten nutzen.

\coleng

As battery capacity increases, not only does flight time improve, but total weight also increases. Very small 4S batteries with a capacity of 450 mAh weigh around 60 g, while the battery shown in Figure~\ref{AbbildungLiPoAkku} with a capacity of 1500 mAh weighs about 185 g. Higher total weight also affects flight characteristics.

\colger

Mit steigender Kapazität eines Akkus erhöht sich nicht nur die Flugdauer, sondern auch das Gesamtgewicht. Sehr kleine 4S-Akkus mit 450 mAh Kapazität wiegen etwa 60 g, während der in Abbildung~\ref{AbbildungLiPoAkku} gezeigte Akku mit 1500 mAh rund 185 g wiegt. Ein höheres Gesamtgewicht beeinflusst auch das Flugverhalten.

\coleng

Li-Ion batteries typically consist of 18650 cells connected in series. The capacity of individual cells depends on the manufacturer and cell quality, typically ranging from 2500 mAh to 3500 mAh. To achieve higher capacity, additional cells must be connected in parallel. The number of cells and their configuration are indicated on the Li-Ion battery using the letters S (series) and P (parallel). Common Li-Ion battery configurations on the market include 3S1P, 4S1P, 4S2P, 6S1P, and 6S2P.

\colger

Li-Ion-Akkus bestehen üblicherweise aus in Reihe geschalteten 18650-Zellen. Die Kapazität der einzelnen Zellen hängt vom Hersteller und von der Zellqualität ab und liegt typischerweise zwischen 2500 mAh und 3500 mAh. Soll eine höhere Kapazität erreicht werden, müssen zusätzliche Zellen parallel geschaltet werden. Die Anzahl der Zellen und deren Verschaltung sind auf den Li-Ion-Akkus mit den Buchstaben S (Serie) und P (Parallel) angegeben. Häufig verfügbare Li-Ion-Akkus sind beispielsweise 3S1P, 4S1P, 4S2P, 6S1P und 6S2P.

\colende

\renewcommand{\deutschertitel}{Stecker (XT30/XT60/XT90) zum Entladen und Laden}
\renewcommand{\englischertitel}{Plugs (XT30/XT60/XT90) for Discharge and Charge}
\makrounterabschnitt
\label{AbschnittSteckerAkkus}

The battery is usually connected to the flight controller via an XT30 or alternatively an XT60 DC connector. Some larger batteries use an XT90 plug. Two wires with a connector are soldered either to the electronic speed controller (in a stack) or directly to the flight controller in an AIO configuration (see Section~\ref{AbschnittFC}).

\colger

Der Anschluss der Akkus an den Flugcontroller erfolgt in der Regel über einen XT30- oder alternativ einen XT60-Gleichstromstecker. Einige größere Akkus verwenden einen XT90-Stecker. Zwei Kabel mit einer Buchse sind an die Motorsteuerung (bei einem Stack) oder am Flugcontroller eines AIO-Systems angelötet (siehe Abschnitt~\ref{AbschnittFC}).

\coleng

For small and lightweight drones up to 4 inches, XT30 connectors are commonly used. 5- and 6-inch drones typically use XT60 connectors, while larger drones often rely on XT90 connectors. The higher the number, the higher the current the connector can safely handle. However, larger connectors are also heavier and bulkier. The wire gauge must also match the connector size.

\colger

Bei kleinen und leichten Drohnen bis zu einer Größe von 4 Zoll sind XT30-Buchsen üblich. 5- und 6-Zoll-Drohnen verwenden üblicherweise XT60-Buchsen, während größere Drohnen häufig XT90-Buchsen nutzen. Je größer die Zahl, desto höhere Ströme kann der Stecker sicher übertragen. Größere Buchsen und Stecker sind allerdings auch schwerer und sperriger. Auch der Leitungsquerschnitt muss zur jeweiligen Buchse passen.

\coleng

In addition to the main DC connector, every battery with two or more cells has a balance connector for safe charging. This connector, which in modern batteries typically follows the JST-XHR standard, allows the charger to measure the voltage of each individual cell and control the charging process accordingly. The number of pins equals the number of cells plus one (for the common ground). Thus, a 4S battery has five wires, and a 6S battery has seven. Using the data from the balance connector, the charger balances the voltage across all cells, preventing overcharging of individual cells and reducing the risk of damage or fire, while also avoiding deep discharge of single cells during operation.

\colger

Neben dem Gleichstromstecker verfügt jeder Akku mit zwei oder mehr Zellen über einen Balancerstecker zum sicheren Laden. Dieser Stecker, der bei modernen Akkus meist dem JST-XHR-Standard entspricht, ermöglicht es dem Ladegerät, die Spannung jeder einzelnen Zelle zu messen und den Ladevorgang entsprechend zu steuern. Die Anzahl der Pins entspricht der Anzahl der Zellen plus eins (für die gemeinsame Masse). Ein 4S-Akku hat somit fünf Leitungen, ein 6S-Akku sieben. Mithilfe der über den Balancerstecker ausgelesenen Werte gleicht das Ladegerät die Spannung der einzelnen Zellen an. Dies verhindert die Überladung einzelner Zellen während des Ladevorgangs und damit das Risiko einer Beschädigung des Akkus oder gar eines Brandes und beugt einer Tiefentladung einzelner Zellen im späteren Betrieb vor.

\colende

\renewcommand{\deutschertitel}{GPS}
\renewcommand{\englischertitel}{GPS}
\makroabschnitt
\label{AbschnittGPS}

A GPS module (Global Positioning System) is essential for many useful functions such as Return-to-Home (automatic return flight), speed measurement, altitude determination, position hold, autonomous flight, and more.

\colger

Ein GPS-Modul (Global Positioning System) ist für viele nützliche Funktionen wie Return-to-Home (automatisches Heimfliegen), Geschwindigkeitsmessung, Höhenbestimmung, Unterstützung der Positionshaltung oder autonomen Flug unverzichtbar.

\coleng

Not every GPS module is compatible with all common flight controller firmwares. While Betaflight supports almost any GPS module, this is not the case for INAV and ArduPilot.

\colger

Nicht jedes GPS-Modul ist mit allen gängigen Firmwares für Flugcontroller kompatibel. Während Betaflight nahezu jedes GPS-Modul unterstützt, ist das bei INAV und ArduPilot nicht immer der Fall.

\coleng

A GPS module with six connections (pads or wires) typically includes a magnetic compass. If the module has only four connections, it lacks this compass. Without a magnetic compass, a drone cannot determine its heading when stationary—meaning it does not know which direction it is facing. Consequently, several useful functions, such as GPS position hold in ArduPilot, are unavailable.

\colger

Ein GPS-Modul mit sechs Anschlüssen (Lötpads oder Kabeln) verfügt in der Regel über einen integrierten magnetischen Kompass. Hat das Modul hingegen nur vier Anschlüsse, fehlt der Kompass. Ohne magnetischen Kompass kann eine Drohne im Stillstand nicht erkennen, in welche Richtung sie ausgerichtet ist. Dadurch sind mehrere sinnvolle Funktionen, wie etwa GPS-Position-Hold bei ArduPilot, nicht nutzbar.

\coleng

Common GPS modules (see Figure~\ref{AbbildungLiPoAkku}) with an integrated magnetic compass are typically between 20\(\times\)20 mm and 25\(\times\)25 mm in size.

\colger

Gängige GPS-Module (siehe Abbildung~\ref{AbbildungLiPoAkku}) mit integriertem magnetischem Kompass sind in der Regel zwischen 20\(\times\)20 mm und 25\(\times\)25 mm groß.

\colende

\begin{figure}[htbp]
  \centering
  \begin{minipage}[t]{0.48\textwidth}
    \centering
    \includegraphics[width=\linewidth]{HGLRS_GPS_M100_PRO_front.jpg}
    \vspace{0pt} % sorgt für Top-Ausrichtung
  \end{minipage}\hfill
  \begin{minipage}[t]{0.48\textwidth}
    \centering
    \includegraphics[width=\linewidth]{HGLRS_GPS_M100_PRO_back.jpg}
    \vspace{0pt}
  \end{minipage}
  \caption{HGLRC M100-5883 M10 GPS Module 21x21\,mm with built-in magnetic Compass (Front and Back)}
  \label{AbbildungLiPoAkku}
\end{figure}

\renewcommand{\deutschertitel}{Empfänger}
\renewcommand{\englischertitel}{Receiver}
\makroabschnitt
\label{AbschnittReceiver}

The receiver receives the control signals (various channel values) from the remote controller and passes them on to the flight controller. It demodulates the incoming signals and converts them into digital channel values.

\colger

Der Empfänger empfängt die Steuersignale (verschiedene Kanalwerte) der Fernsteuerung und gibt sie an den Flugcontroller weiter. Er demoduliert die empfangenen Signale und wandelt sie in digitale Kanalwerte um.

\coleng

The receiver and transmitter must be compatible, meaning they must use the same protocol and operate in the same frequency band. Common frequency bands are 2.4 GHz and, for long-range applications, 868 MHz in the EU or 915 MHz in the USA. In Germany, the use of these frequency ranges is permitted at a maximum transmission power of 25 mW.

\colger

Empfänger und Fernsteuerung müssen zueinander kompatibel sein, das heißt, sie müssen dasselbe Protokoll und denselben Frequenzbereich verwenden. Gängige Systeme nutzen die Frequenzbereiche 2,4 GHz und -- für Long-Range-Anwendungen -- 868 MHz in der EU bzw. 915 MHz in den USA. In Deutschland ist die Nutzung dieser Frequenzbereiche bei einer maximalen Sendeleistung von 25 mW zulässig.

\coleng

Two popular protocols are the open-source system ExpressLRS (ELRS) and the proprietary TBS Crossfire. Because ELRS is open source, there are many hardware providers, and development progresses rapidly. For this reason, this document focuses entirely on the ELRS standard. ELRS devices (transmitters and receivers) run the ExpressLRS firmware, and ideally, both should use the same firmware version.

\colger

Zwei populäre Protokolle sind das offene System ExpressLRS (ELRS) und das proprietäre TBS Crossfire. Da ELRS ein Open-Source-Protokoll ist, gibt es zahlreiche Anbieter von Hardwarekomponenten, und die Weiterentwicklung schreitet schnell voran. Aus diesem Grund konzentriert sich dieses Dokument vollständig auf den Standard ELRS. ELRS-Geräte (Sender und Empfänger) verwenden die ExpressLRS-Firmware, und idealerweise sollten beide die gleiche Firmware-Version nutzen.

\coleng

There are receivers with an integrated ceramic antenna (see Figure~\ref{AbbildungRadioMasterRP2}) mounted directly on the circuit board, as well as receivers with one (see Figure~\ref{AbbildungRadioMasterRP1}) or two detachable antennas. The commonly used connector is the delicate U.FL (also called \textsl{pigtail}). Receivers with fixed ceramic antennas are more compact, as no antenna cable needs to be routed, but the small antenna height of only a few millimeters significantly reduces range.

\colger

Es gibt Empfänger mit einer integrierten Keramikantenne (siehe Abbildung~\ref{AbbildungRadioMasterRP2}), die direkt auf der Platine montiert ist, sowie Empfänger mit einer (siehe Abbildung~\ref{AbbildungRadioMasterRP1}) oder zwei abnehmbaren Antennen. Der üblicherweise verwendete Steckverbinder ist auch hier der filigrane U.FL-Stecker (\textsl{Pigtail}). Empfänger mit fester Keramikantenne sind kompakter, da kein Antennenkabel verlegt werden muss; durch die nur wenige Millimeter hohe Antenne reduziert sich jedoch die Reichweite deutlich.

\colende

\begin{figure}[htbp]
  \centering
  \begin{minipage}[t]{0.48\textwidth}
    \centering
    \includegraphics[width=.5\linewidth]{RadioMaster_RP2_Receiver_ELRS_front.jpg}
    \vspace{0pt} % sorgt für Top-Ausrichtung
  \end{minipage}\hfill
  \begin{minipage}[t]{0.48\textwidth}
    \centering
    \includegraphics[width=.5\linewidth]{RadioMaster_RP2_Receiver_ELRS_back.jpg}
    \vspace{0pt}
  \end{minipage}
  \caption{RadioMaster RP2 ELRS 2.4GHz Nano Receiver 13x11\,mm with integrated Antenna (Front and Back)}
  \label{AbbildungRadioMasterRP2}
\end{figure}

\begin{figure}[htbp]
  \centering
  \begin{minipage}[t]{0.48\textwidth}
    \centering
    \includegraphics[width=.5\linewidth]{RadioMaster_RP1_Receiver_ELRS_front.jpg}
    \vspace{0pt} % sorgt für Top-Ausrichtung
  \end{minipage}\hfill
  \begin{minipage}[t]{0.48\textwidth}
    \centering
    \includegraphics[width=.5\linewidth]{RadioMaster_RP1_Receiver_ELRS_back.jpg}
    \vspace{0pt}
  \end{minipage}
  \caption{RadioMaster RP1 ELRS 2.4GHz Nano Receiver 13x11\,mm with U.FL Antenna Connector (Front and Back)}
  \label{AbbildungRadioMasterRP1}
\end{figure}

\renewcommand{\deutschertitel}{Fernbedienung (Sender)}
\renewcommand{\englischertitel}{Remote Control (Sender)}
\makroabschnitt
\label{AbschnittSender}

The remote control must implement the same protocol as the receiver in use and operate in the same frequency band. Manufacturers that produce remote controllers compatible with the ELRS protocol include RadioMaster, Axisflying, and Jumper.

\colger

Die verwendete Fernbedienung muss dasselbe Protokoll implementieren wie der eingesetzte Empfänger und denselben Frequenzbereich nutzen. Zu den Herstellern von mit dem ELRS-Protokoll kompatiblen Fernbedienungen gehören RadioMaster, Axisflying und Jumper.

\coleng

ELRS remote controls (transmitters) use ExpressLRS as the firmware for the radio protocol and transmitter module, and EdgeTX as the operating system -- a modernized version of OpenTX. EdgeTX generates the control signals, while ExpressLRS transmits them.

\colger

ELRS-Fernbedienungen (Sender) verwenden ExpressLRS als Firmware für das Funkprotokoll und das Sendermodul sowie EdgeTX als Betriebssystem, eine modernisierte Variante von OpenTX. EdgeTX erzeugt die Steuersignale, und ExpressLRS überträgt sie.

\colende

\renewcommand{\deutschertitel}{Videosender}
\renewcommand{\englischertitel}{Video Transmitter}
\makroabschnitt
\label{AbschnittVTX}

The video transmitter (VTX) transmits the live image from the FPV camera to the FPV goggles using the selected channel within the defined frequency band. In addition to analog video transmitters, there are several digital systems that are incompatible with one another. Video transmitters are sold either as standalone devices or as kits that include a camera and antenna(s). Figure~\ref{AbbildungSpeedyBeeTX800FPVVTX} shows an example of an analog video transmitter.

\colger

Der Videosender (VTX) überträgt das Livebild der FPV-Kamera über den eingestellten Kanal im festgelegten Frequenzband an die FPV-Brille. Neben analogen Videosendern existieren auch verschiedene digitale Systeme, die untereinander nicht kompatibel sind. Videosender werden entweder als Einzelgerät oder als Kit mit Kamera und Antenne(n) verkauft. Abbildung~\ref{AbbildungSpeedyBeeTX800FPVVTX} zeigt einen analogen Videosender.

\colende

\begin{figure}[htbp]
  \centering
  \begin{minipage}[t]{0.48\textwidth}
    \centering
    \includegraphics[width=\linewidth]{SpeedyBee_TX800_FPV_VTX_vorderseite_crop.jpg}
    \vspace{0pt} % sorgt für Top-Ausrichtung
  \end{minipage}\hfill
  \begin{minipage}[t]{0.48\textwidth}
    \centering
    \includegraphics[width=\linewidth]{SpeedyBee_TX800_FPV_VTX_rueckseite_crop.jpg}
    \vspace{0pt}
  \end{minipage}
  \caption{SpeedyBee TX800 FPV VTX Video Transmitter (Front and Back)}
  \label{AbbildungSpeedyBeeTX800FPVVTX}
\end{figure}

\colstart

In Germany, the frequency range from 5725 MHz to 5875 MHz may be used with a maximum transmission power of 25 mW. The channels permitted for analog video transmission within the respective bands are shown in Table~\ref{TabelleFrequenzenVTX}. Notably, the Race Band has wide channel spacing of 37 MHz to prevent channel overlap.

\colger

In Deutschland ist die Nutzung des Frequenzbereichs von 5725 MHz bis 5875 MHz bei einer maximalen Sendeleistung von 25 mW zulässig. Die für die analoge Bildübertragung in den jeweiligen Bändern zugelassenen Kanäle zeigt Tabelle~\ref{TabelleFrequenzenVTX}. Auffällig sind beim Race Band die großen Kanalabstände von 37 MHz, die Kanalüberlagerungen verhindern.

\colende

\begin{table}[htb!]
\centering
\captionabove{Overview about permitted VTX Channels in Germany}
\label{TabelleFrequenzenVTX}
\begin{tabular}{l@{\hskip 8mm}l}
\toprule
CPU (MCU)  & Channel Number (Frequency MHz)    \\
\midrule
Band A     & 7 (5745), 6 (5765), 5 (5785), 4 (5805), 3 (5825), 2 (5845), 1 (5865) \\
Band B     & 1 (5733), 2 (5752), 3 (5771), 4 (5790), 5 (5809), 6 (5828), 7 (5847), 8 (5866) \\
Band E     & none Frequency permitted in Germany \\
Band F     & 1 (5749), 2 (5760), 3 (5780), 4 (5800), 5 (5820), 6 (5840), 7 (5860) \\
Race Band  & 3 (5732), 4 (5769), 5 (5806), 6 (5843) \\
\bottomrule
\end{tabular}
\end{table}

\colstart

The channel and band used by the video transmitter for image transmission are configured in the graphical interface of the flight controller firmware.

\colger

Der vom Videosender zur Übertragung verwendete Kanal und das Band werden in der grafischen Benutzeroberfläche der Firmware des Flugcontrollers festgelegt.

\coleng

Analog video transmitters are cross-compatible among manufacturers, while this is not the case for digital systems. For digital video transmission, the transmitter and receiver must be from the same manufacturer. Digital systems for live video transmission are available from DJI, HDZero, and CaddxFPV under the brand name Walksnail.

\colger

Analoge Videosender sind herstellerübergreifend mit analogen Videobrillen kompatibel. Bei digitalen Systemen ist das nicht der Fall -- hier müssen Sender und Empfänger vom selben Hersteller stammen. Digitale Systeme zur Übertragung des Livebilds werden von DJI, HDZero und CaddxFPV unter dem Namen Walksnail angeboten.

\coleng

The advantages of digital live video transmission include excellent image quality and the ability to record HD footage directly without the need for an additional action camera.

\colger

Vorteile der digitalen Livebildübertragung sind die hervorragende Bildqualität und die Möglichkeit, Videos in HD-Qualität direkt ohne zusätzliche Action-Cam aufzuzeichnen.

\coleng

Analog transmission, on the other hand, offers advantages such as cross-manufacturer compatibility, low latency, a gradual rather than abrupt degradation of image quality under interference, and significantly lower hardware costs.

\colger

Analoge Übertragung bietet hingegen Vorteile wie herstellerübergreifende Kompatibilität, geringe Latenz, ein schrittweises statt abruptes Verschlechtern des Bildes bei Störungen und deutlich geringere Anschaffungskosten der Komponenten.

\coleng

Antennas have a major impact on both image quality and transmission range. Operating a video transmitter without an antenna is not recommended, as it can overheat and become damaged. Analog systems typically use delicate high-frequency connectors following the U.FL standard, commonly known as \textsl{pigtail} connectors (also used in Wi-Fi devices). Digital systems, such as those from DJI, generally use MMCX connectors. Antennas with larger and more robust SMA screw connectors can be attached using adapter cables, which are often included with the video transmitter.

\colger

Die verwendeten Antennen haben großen Einfluss auf Bildqualität und Reichweite. Der Betrieb eines Videosenders ohne Antenne ist nicht empfehlenswert, da dieser überhitzen und beschädigt werden kann. Analoge Systeme verwenden in der Regel sehr filigrane Hochfrequenzsteckverbinder nach dem U.FL-Standard, umgangssprachlich auch \textsl{Pigtail} genannt, wie sie auch bei WLAN-Geräten üblich sind. Digitale Systeme, z. B. von DJI, nutzen meist MMCX-Steckverbinder. Antennen mit größerem und robusterem SMA-Schraubgewinde können über Adapterkabel angeschlossen werden, die häufig dem Videosender beiliegen.

\coleng

The antennas on both the transmitter and receiver sides should match. Available video transmitter antennas differ by their polarization. There are two types:

\colger

Die Antennen auf Sender- und Empfängerseite sollten zueinander passen. Die verfügbaren Antennen für Videosender unterscheiden sich in ihrer Polarisation. Es gibt:

\coleng

\begin{itemize}
\item linearly polarized antennas (LP)
\item circularly polarized antennas (CP)
\end{itemize}

\colger

\begin{itemize}
\item Antennen mit linearer Polarisation (LP)
\item Antennen mit zirkularer Polarisation (CP)
\end{itemize}

\coleng

LP antennas are the simplest and most cost-effective type. They are often included with video transmitters and FPV goggles and typically have a rod-like shape.

\colger

LP-Antennen sind die einfachste und kostengünstigste Variante. Sie werden häufig mit dem Videosender oder der Videobrille mitgeliefert und besitzen meist eine stabförmige Bauweise.

\coleng

CP antennas provide better image quality and greater range. With circular polarization, the signal’s polarization rotates as it propagates. There are two variants: right-hand circular polarization (RHCP) and left-hand circular polarization (LHCP).

\colger

CP-Antennen ermöglichen eine bessere Bildqualität und größere Reichweite. Bei zirkularer Polarisation dreht sich die Polarisation des Signals während der Ausbreitung. Es gibt zwei Varianten: rechtsdrehende (RHCP) und linksdrehende (LHCP) Polarisation.

\coleng

RHCP antennas are typically used with analog FPV drones, while LHCP antennas are commonly used with digital FPV systems.

\colger

RHCP-Antennen werden in der Regel bei analogen FPV-Drohnen eingesetzt, während LHCP-Antennen häufig bei FPV-Drohnen mit digitaler Bildübertragung verwendet werden.

\colende

\renewcommand{\deutschertitel}{Kamera}
\renewcommand{\englischertitel}{Camera}
\makroabschnitt
\label{AbschnittKamera}

This section focuses exclusively on analog FPV cameras. Digital systems for live video transmission have not been used in the projects carried out so far.

\colger

Dieser Abschnitt befasst sich ausschließlich mit analogen FPV-Kameras. Digitale Systeme zur Übertragung des Livebilds kamen in den bisher durchgeführten Projekten nicht zum Einsatz.

\coleng

The camera generates a video signal (NTSC or PAL), which is passed through the flight controller to the video transmitter. In digital systems, the cameras are directly connected to the video transmitter.

\colger

Die Kamera erzeugt ein Videosignal (NTSC oder PAL), das über den Flugcontroller an den Videosender weitergeleitet wird. Bei digitalen Systemen sind die Kameras direkt mit dem Videosender verbunden.

\coleng

The image quality of analog cameras appears outdated due to noise and blurriness, far from HD standards. The perceived quality ranges from VHS-level at worst to DVD-level at best. However, analog video transmission offers several advantages, such as lower cost, minimal latency, lightweight design, and broad cross-manufacturer compatibility of available components.

\colger

Die Bildqualität analoger Kameras wirkt durch Bildfehler und Unschärfen nicht mehr zeitgemäß und liegt weit unter HD-Qualität. Der Bildeindruck ist subjektiv im schlechtesten Fall mit VHS-Kassetten und im besten Fall mit DVDs vergleichbar. Die analoge Bildübertragung hat jedoch mehrere Vorteile: geringere Kosten, minimale Latenz, geringes Gewicht und eine herstellerübergreifende Kompatibilität der Komponenten.

\coleng

Recording high-quality video is not possible with an analog camera. If this requirement exists, an additional action camera or a similar device must be mounted on the drone.

\colger

Die Aufnahme qualitativ hochwertiger Videos ist mit einer analogen Kamera nicht möglich. Wenn diese Anforderung besteht, muss eine zusätzliche Actioncam oder ein vergleichbares Gerät auf der Drohne mitgeführt werden.

\coleng

Cameras available on the market differ primarily in resolution, lens focal length (which determines the field of view), and physical size (width).

\colger

Die am Markt verfügbaren Kameras unterscheiden sich in erster Linie in der Auflösung, der Brennweite der Linse -- die das Sichtfeld beeinflusst -- und der Bauform (Breite).

\coleng

The camera must not be too wide for the frame used (see Section~\ref{AbschnittFrames}). Modern cameras typically have widths of 14 mm (nano cameras) or 19 mm (micro cameras). Figure~\ref{AbbildungCaddyAntNanoCamera} shows a nano camera, and Figure~\ref{AbbildungRunCamPhoenix2Camera} shows a micro camera.

\colger

Die Kamera darf für den verwendeten Rahmen (siehe Abschnitt~\ref{AbschnittFrames}) nicht zu breit sein. Moderne Kameras sind in der Regel 14 mm (Nano-Kamera) oder 19 mm (Micro-Kamera) breit. Abbildung~\ref{AbbildungCaddyAntNanoCamera} zeigt eine Nano-Kamera und Abbildung~\ref{AbbildungRunCamPhoenix2Camera} eine Micro-Kamera.

\colende

\begin{figure}[htbp]
  \centering
  \begin{minipage}[t]{0.48\textwidth}
    \centering
    \includegraphics[width=.5\linewidth]{Caddx_ANT_Nano_1200TVL_14mm_165FOV_top.jpg}
    \vspace{0pt} % sorgt für Top-Ausrichtung
  \end{minipage}\hfill
  \begin{minipage}[t]{0.48\textwidth}
    \centering
    \includegraphics[width=.5\linewidth]{Caddx_ANT_Nano_1200TVL_14mm_165FOV_back.jpg}
    \vspace{0pt}
  \end{minipage}
  \caption{Caddx ANT Nano Analog Camera 1200\,TVL 14x14\,mm (Top and Back)}
  \label{AbbildungCaddyAntNanoCamera}
\end{figure}

\begin{figure}[htbp]
  \centering
  \begin{minipage}[t]{0.48\textwidth}
    \centering
    \includegraphics[width=.5\linewidth]{RunCam_Phoenix_2_1000TVL_19mm_155FOV_top.jpg}
    \vspace{0pt} % sorgt für Top-Ausrichtung
  \end{minipage}\hfill
  \begin{minipage}[t]{0.48\textwidth}
    \centering
    \includegraphics[width=.5\linewidth]{RunCam_Phoenix_2_1000TVL_19mm_155FOV_back.jpg}
    \vspace{0pt}
  \end{minipage}
  \caption{RunCam Phoenix 2 Analog Camera 1000\,TVL 19x19\,mm (Top and Back)}
  \label{AbbildungRunCamPhoenix2Camera}
\end{figure}

\colstart

The horizontal resolution is specified in TV lines (TVL), typically 1000, 1200, or 1500 TVL. The higher this value, the better the potential image quality.

\colger

Die horizontale Auflösung wird in TV-Lines (TVL) angegeben und beträgt meist 1000, 1200 oder 1500 TVL. Je höher dieser Wert ist, desto besser ist die mögliche Bildqualität.

\coleng

The focal length determines the field of view. The smaller the value, the wider the field of view, which is advantageous for flying. However, a low focal length (e.g., 1.8 mm or 2.1 mm) creates a fisheye effect, resulting in image distortion that can cause eye fatigue and make small objects harder to recognize.

\colger

Die Brennweite definiert das Sichtfeld. Je kleiner der Wert ist, desto größer ist das Sichtfeld, was beim Fliegen vorteilhaft ist. Eine niedrige Brennweite (z. B. 1.8 mm oder 2.1 mm) führt jedoch zu einem Fischaugen-Effekt und damit zu Verzerrungen, die die Augen schneller ermüden lassen und besonders kleine Objekte schwerer erkennbar machen.

\coleng

Higher focal lengths (e.g., 2.5 mm or 2.8 mm) provide a more natural image but narrow the field of view, which can make orientation during flight more difficult.

\colger

Hohe Brennweiten (z. B. 2.5 mm oder 2.8 mm) bieten ein angenehmeres Bildgefühl, verengen jedoch das Sichtfeld und erschweren dadurch möglicherweise die Orientierung während des Fluges.

\colende

\renewcommand{\deutschertitel}{Videobrille (FPV-Brille)}
\renewcommand{\englischertitel}{FPV Goggles}
\makroabschnitt
\label{AbschnittFPVBrille}

The FPV goggles used must be compatible with the video transmitter. Components for digital live video transmission are generally only compatible if they come from the same manufacturer. Analog FPV goggles, on the other hand, are cross-compatible with all analog video transmitters.

\colger

Die verwendete Videobrille muss zum Videosender passen. Komponenten für die digitale Livebildübertragung sind in der Regel nur dann zueinander kompatibel, wenn sie vom gleichen Hersteller stammen. Analoge Videobrillen sind hingegen herstellerübergreifend mit allen analogen Videosendern kompatibel.

\coleng

FPV goggles available on the market differ primarily in resolution, display technology, and the available interfaces (inputs and outputs). Table~\ref{TabelleVideobrillen} compares several key characteristics of some analog FPV goggles currently available on the market.

\colger

Die am Markt verfügbaren Videobrillen unterscheiden sich in erster Linie hinsichtlich der Auflösung, der verwendeten Display-Technologie und der vorhandenen Schnittstellen (Ein- und Ausgänge). Tabelle~\ref{TabelleVideobrillen} zeigt eine Gegenüberstellung einiger relevanter Merkmale ausgewählter analoger Videobrillen.

\colende

\begin{table}[htb!]
\centering
\captionabove{Overview about analog FPV Goggles}
\label{TabelleVideobrillen}
\begin{tabular}{l@{\hskip 8mm}r@{\hskip 8mm}l@{\hskip 8mm}l}
\toprule
Goggle (Product)        & Resolution    & Display  & Video Output    \\
\midrule
Skyzone Cobra X V4      & 1280x720\,px  & LCD  & analog A/V          \\
Skyzone SKY04X Pro      & 1920x1080\,px & OLED & analog A/V          \\
Skyzone SKY04O Pro      & 1280x720\,px  & OLED & analog A/V          \\
Skyzone SKY02O          & 640x400\,px   & LCD  & analog A/V          \\
FatShark Dominator      & 1920x1080\,px & OLED & USB-C video output  \\
FatShark Dominator HDO+ & 1920x1080\,px & OLED & analog A/V          \\
FatShark Recon HD       & 1920x1080\,px & LCD  & USB-C video output  \\
Fat Shark Echo          & 800x480\,px   & LCD  & no video output     \\
Eachine EV300O          & 1024x768\,px  & OLED & analog A/V          \\ 
Eachine EV300D          & 1280x960\,px  & LCD  & analog A/V          \\ 
Rotorama 800D           & 800x480\,px   & LCD  & no video output     \\
\bottomrule
\end{tabular}
\end{table}

\colstart

The ability to output the video signal through an interface and process it in real time on another device is particularly interesting for many potential AI projects. Although numerous FPV goggles feature an HDMI port, it is typically used only for playing back video from external sources. No common FPV goggle currently supports video output via HDMI.

\colger

Besonders die Möglichkeit, das Videosignal über eine Schnittstelle auszugeben und in einem anderen Gerät in Echtzeit zu verarbeiten, ist für viele denkbare KI-Projekte eine interessante Option. Zahlreiche Videobrillen verfügen zwar über einen HDMI-Port, dieser dient jedoch ausschließlich zum Abspielen von Videos von externen Quellen. Eine Ausgabe des Videosignals über HDMI wird von keiner gängigen FPV-Brille unterstützt.

\coleng

Most FPV goggles include a 3.5 mm AV port that allows analog video and audio input and output. Using a video capture device, the output signal can be transmitted to a computer for further processing.

\colger

Die meisten Videobrillen verfügen über eine 3,5 mm AV-Schnittstelle, über die Video und Audio analog ein- und ausgegeben werden können. Mithilfe eines Videograbbers kann das Ausgabesignal an einen Computer weitergeleitet und dort verarbeitet werden.

\coleng

If FPV goggles have a USB-C port for video output, it depends on whether the signal is exported as USB Video Class (UVC), like a webcam, or as HDMI. In the case of a UVC device, the video stream can be processed directly. If the signal is HDMI, a suitable video capture device is also required to forward and process the signal on a computer.

\colger

Verfügt eine Videobrille über eine USB-C-Schnittstelle zur Ausgabe des Videosignals, hängt es davon ab, ob das Videosignal -- wie bei einer Webcam -- als USB Video Class (UVC) exportiert wird oder als HDMI-Signal. Bei einem UVC-Gerät kann der Videostream direkt weiterverarbeitet werden. Handelt es sich hingegen um ein HDMI-Signal, ist ein geeigneter Videograbber erforderlich, um das Ausgabesignal an einen Computer weiterzuleiten und zu verarbeiten.

\colende
