\setcounter{page}{5}


\renewcommand{\deutschertitel}{Vorwort}
\renewcommand{\englischertitel}{Preface}

\chapter*{}

% \end{paracol}
\begin{paracol}{2}[]

{\raggedright\huge\bfseries\sffamily \englischertitel \par\ } \\[1.8ex]

\switchcolumn

{\raggedright\huge\bfseries\sffamily \deutschertitel \par\ } \\[1.8ex]

\coleng

This document provides an introduction to the complex topic of drones with artificial intelligence. Its focus is on the development (including the selection of suitable hardware and software components), construction, and operation of drones for teaching and research purposes.

\colger

Dieses Dokument bietet einen Einstieg in das komplexe Thema Drohnen mit künstlicher Intelligenz. Schwerpunkte sind die Entwicklung (einschließlich der Auswahl geeigneter Hard- und Softwarekomponenten), der Bau und der Betrieb von Drohnen in der Lehre und für Forschungsprojekte.

\coleng

The insights presented in this document stem from the research project \textsl{AI-Assisted Drone Platform}, funded by the Connectom Networking and Innovation Fund of hessian.AI, and from the course \textsl{Drones with Artificial Intelligence} at the Frankfurt University of Applied Sciences.

\colger

Beim Schreiben dieses Dokuments flossen Erkenntnisse aus dem vom Connectom Vernetzungs- und Innovationsfonds des hessian.AI geförderten Forschungsprojekt \textsl{KI-gestützte Drohnenplattform} sowie aus der Lehrveranstaltung \textsl{Drohnen mit Künstlicher Intelligenz} an der Frankfurt University of Applied Sciences ein.

\coleng

Key criteria for selecting the components presented in this document include adaptability to different application scenarios, cost, robustness, long-term market availability, and the quality of documentation and manufacturer support.

\colger

Maßgebliche Kriterien bei der Auswahl der in diesem Dokument vorgestellten Komponenten sind unter anderem die Anpassbarkeit an unterschiedliche Einsatzszenarien, der Anschaffungspreis, die Robustheit, die langfristige Marktverfügbarkeit sowie die Qualität der Dokumentation und des Herstellersupports.

\coleng

The goal of this document is not to provide an exhaustive treatment of drones and AI, but rather to focus on the technologies and solutions that were current during its creation and with which practical experience was gained in the Computer Science program of Faculty 2 at the Frankfurt University of Applied Sciences.

\colger

Die vollständige Abhandlung zu den Themen Drohnen und KI ist nicht Ziel dieses Dokuments. Der Fokus liegt auf den Technologien und Lösungen, die während der Erstellung aktuell waren und mit denen im Studienfeld Informatik des Fachbereichs 2 der Frankfurt University of Applied Sciences praktische Erfahrungen gesammelt wurden.

\coleng

Chapters~1 and 2 provide a concise overview of the hardware and software components required for building and operating FPV drones. For the AI projects presented here, drones primarily serve as tools and/or transport platforms.

\colger

Kapitel~1 und 2 stellen knapp die Hard- und Softwarekomponenten vor, die zum Bau und Betrieb von FPV-Drohnen nötig sind. Diese dienen zur Realisierung der hier vorgestellten KI-Projekte in erster Linie als Werkzeug und/oder Transportvehikel.

\coleng

From Chapter~3 onward, various AI applications are described that use FPV drones for data collection and/or as transport platforms. The chapter presents applications developed, implemented, and evaluated in research projects, courses, and theses within the Computer Science unit at the Frankfurt University of Applied Sciences. Each project description includes the required additional hardware and software components, implementation steps, and associated costs.

\colger

Ab Kapitel~3 folgen Beschreibungen verschiedener KI-Anwendungen, die FPV-Drohnen zur Datenerfassung und/oder als Transportvehikel nutzen. Vorgestellt werden Anwendungen, die in Forschungsprojekten, Lehrveranstaltungen und Abschlussarbeiten in der Lehreinheit Informatik der Frankfurt University of Applied Sciences entwickelt, implementiert und evaluiert wurden. Das Kapitel beschreibt zu jeder dieser KI-Anwendungen die benötigten zusätzlichen Hard- und Softwarekomponenten, die Umsetzungsschritte und die Kosten.

\coleng

The components typically used to build FPV drones are not capable of providing AI functionality on their own. Flight controllers lack the necessary computing and memory resources to run AI applications locally. Additional hardware and software are therefore required. These additional components can either be integrated into the drone or remain on the ground to control the drone and/or process data received from it. The respective chapters discuss different implementation options, their advantages and disadvantages, and associated costs.

\colger

Die üblicherweise zum Bau von FPV-Drohnen verwendeten Komponenten sind nicht in der Lage, KI-Funktionalität bereitzustellen. Die Flugcontroller verfügen nicht über die notwendigen Rechen- und Speicherressourcen, um KI-Anwendungen lokal auszuführen. Zusätzliche Hard- und Software ist daher erforderlich. Diese kann entweder in die Drohne integriert werden und mitfliegen oder am Boden verbleiben, um die Drohne zu steuern und/oder Daten zu empfangen und dort die KI-Funktionalität zu realisieren. In den jeweiligen Kapiteln werden verschiedene Realisierungsmöglichkeiten, deren Vor- und Nachteile sowie die Anschaffungskosten beschrieben.

\coleng

We greatly appreciate your comments and suggestions for improvement.

\colger

Über Ihre Kommentare und Verbesserungsvorschläge freuen wir uns sehr.

\coleng

\vspace*{1cm}

\colger

\vspace*{1cm}

\coleng

{\noindent}%
Frankfurt am Main{\hfill}

\colger

\begin{flushright} {\noindent}%
\sffamily
\textit{Christian Baun},\\
\textit{Theodor Bloch},\\
\textit{Matthias Deegener},\\ 
\textit{Oliver Hahm},\\
\textit{Martin Kappes},\\ 
\textit{Nur Uddun Syeed}
\end{flushright}

\colende

