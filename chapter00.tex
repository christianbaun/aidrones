\setcounter{page}{5}


\renewcommand{\deutschertitel}{Vorwort}
\renewcommand{\englischertitel}{Preface}

\chapter*{}

% \end{paracol}
\begin{paracol}{2}[]

{\raggedright\huge\bfseries\sffamily \englischertitel \par\ } \\[1.8ex]

\switchcolumn

{\raggedright\huge\bfseries\sffamily \deutschertitel \par\ } \\[1.8ex]

\coleng

TBD

\colger

Dieses Dokument bietet einen Einstieg in das komplexe Thema Drohnen mit künstlicher Intelligenz. Schwerpunkte sind die Entwicklung (inkl. Auswahl geeigneter Hard- und Softwarekomponenten), Bau und Betrieb von Drohnen in der Lehre und für Forschungsprojekte.  

\coleng

TBD

\colger

Beim Schreiben dieses Dokuments flossen Erkenntnisse aus dem vom Connectom Vernetzungs- und Innovationsfond des hessian.AI geförderten Forschungsprojekt \textsl{KI-gestützte Drohnenplattform} und aus der Lehrveranstaltung \textsl{Drohnen mit Künstlicher Intelligenz} an der Frankfurt University of Applied Sciences an.

\coleng

TBD

\colger

Maßgebliche Kriterien der Auswahl der in dieses Dokument vorgestellten Komponenten sind unter anderem: Anpassbarkeit an verschiedenste Einsatzszenarien, Anschaffungspreis, Robustheit, langfristige Marktverfügbarkeit sowie Qualität der Dokumentation und Herstellersupport.

\coleng

TBD

\colger

Die Realisierung einer vollständigen Abhandlung zu den Themen Drohnen und KI ist nicht Ziel dieses Dokument. Der Fokus liegt auf den Technologien und Lösungen, die während der Erstellung aktuell waren und mit denen praktische Erfahrung im Studienfeld Informatik des Fachbereich 2 an der Frankfurt University of Applied Sciences gemacht wurden.

\coleng

TBD

\colger

Kapitel~1 und 2 stellen knapp die Hard- und Softwarekomponenten vor, die zum Bau und Betrieb von FPV-Drohnen nötig sind. Dieses sind zur Realisierung der hier vorgestellten KI-Projekte in erster Linie ein Werkzeug und/oder Transportvehikel. 

\coleng

TBD

\colger

Ab Kapitel~3 folgenden Beschreibungen verschiedener KI-Anwendungen, die FPV-Drohnen zur Datenerfassung und/oder als Tarnsportvehikel nutzen. Vorgestellt werden Anwendungen, die in Forschungsprojekten, Lehrveranstaltungen und Abschlussarbeiten in der Lehreinheit Informatik an der Frankfurt University of Applied Sciences entwickelt, implementiert und evaluiert wurde. Das Kapitel beschreibt zu jeder dieser KI-Anwendungen die benötigten zusätzlichen Hard- und Softwarekomponenten, nötige Schritte zur Realisierung und die Kosten.

\coleng

TBD

\colger

Die üblicherweise zum Bau von FPV-Drohnen verwendeten Komponenten sind nicht in der Lage eine KI-Funktionalität bereitzustellen. Die Flugcontroller haben nicht die nötigen Rechen- und Speicherressourcen, um KI-Anwendungen lokal zu betreiben. Es ist in jedem Fall zusätzliche Hard- und Software nötig. Prinzipiell können die zusätzlichen Komponenten in der Drohne integriert werden und mitfliegen. Alternativ können die zusätzlichen Komponenten die die Drohnen vom Boden aus Steuern und/oder Daten von Ihnen abrufen und die KI-Funktionalität am Boden realisieren. In den jeweiligen Kapiteln werden verschiedene Realisierungsmöglichkeiten, deren Vor- und Nachteile und die Anschaffungskosten beschrieben.

\coleng

TBD

\colger

Über Ihre Kommentare und Verbesserungsvorschläge freuen wir uns sehr.

\coleng

\vspace*{1cm}

\colger

\vspace*{1cm}

\coleng

{\noindent}%
Frankfurt am Main{\hfill}

\colger

\begin{flushright} {\noindent}%
\sffamily
\textit{Christian Baun},\\
\textit{Theodor Bloch},\\
\textit{Matthias Deegener},\\ 
\textit{Oliver Hahm},\\
\textit{Martin Kappes},\\ 
\textit{Nur Uddun Syeed}
\end{flushright}

\colende

