\setcounter{page}{5}


\renewcommand{\deutschertitel}{Vorwort}
\renewcommand{\englischertitel}{Preface}

\chapter*{}

% \end{paracol}
\begin{paracol}{2}[]

{\raggedright\huge\bfseries\sffamily \englischertitel \par\ } \\[1.8ex]

\switchcolumn

{\raggedright\huge\bfseries\sffamily \deutschertitel \par\ } \\[1.8ex]

\coleng

This document provides an introduction to the complex topic of drones with artificial intelligence. Its focus is on the development (including the selection of suitable hardware and software components), construction, and operation of drones for teaching and research purposes.

\colger

Dieses Dokument bietet einen Einstieg in das komplexe Thema Drohnen mit künstlicher Intelligenz. Schwerpunkte sind die Entwicklung (einschließlich der Auswahl geeigneter Hard- und Softwarekomponenten), der Bau und der Betrieb von Drohnen in der Lehre und für Forschungsprojekte.

\coleng

The insights presented in this document stem from the research project \textsl{AI-Assisted Drone Platform}, funded by the Connectom Networking and Innovation Fund of hessian.AI, as well as from the course \textsl{Drones with Artificial Intelligence} at the Frankfurt University of Applied Sciences.

\colger

Die in diesem Dokument dargestellten Erkenntnisse stammen aus dem vom Connectom Vernetzungs- und Innovationsfonds des hessian.AI geförderten Forschungsprojekt \textsl{KI-gestützte Drohnenplattform} sowie aus der Lehrveranstaltung \textsl{Drohnen mit Künstlicher Intelligenz} an der Frankfurt University of Applied Sciences.

\coleng

Key criteria for selecting the components presented in this document include adaptability to different application scenarios, cost, robustness, long-term market availability, and the quality of documentation and manufacturer support.

\colger

Maßgebliche Kriterien bei der Auswahl der in diesem Dokument vorgestellten Komponenten sind unter anderem die Anpassbarkeit an unterschiedliche Einsatzszenarien, die Anschaffungskosten, die Robustheit, die langfristige Marktverfügbarkeit sowie die Qualität der Dokumentation und des Herstellersupports.

\coleng

The goal of this document is not to provide an exhaustive treatment of drones and AI, but rather to focus on technologies and solutions that were current during its creation and with which practical experience was gained in the Computer Science program of Faculty~2 (Computer Science and Engineering) at the Frankfurt University of Applied Sciences.

\colger

Eine vollständige Abhandlung der Themen Drohnen und KI ist nicht Ziel dieses Dokuments. Der Fokus liegt auf den Technologien und Lösungen, die zum Zeitpunkt der Erstellung aktuell waren und mit denen im Studienfeld Informatik des Fachbereichs~2 (Informatik und Ingenieurwissenschaften) der Frankfurt University of Applied Sciences praktische Erfahrungen gesammelt wurden.

\coleng

Chapters~1 and 2 provide a concise overview of the hardware and software components required for building and operating FPV drones. For the AI projects presented here, drones primarily serve as tools and transport platforms. Chapter~3 contains helpful information for before and after the first flight.

\colger

Kapitel~1 und 2 geben einen kompakten Überblick über die Hard- und Softwarekomponenten, die für den Bau und den Betrieb von FPV-Drohnen erforderlich sind. In den hier vorgestellten KI-Projekten dienen die Drohnen in erster Linie als Werkzeug und als Transportplattform. Kapitel~3 enthält hilfreiche Hinweise vor und nach dem ersten Flug.

\coleng

From Chapter~4 onward, various AI applications are described that use FPV drones for data collection and as transport platforms. The chapter presents applications developed, implemented, and evaluated in research projects, courses, and theses within the Computer Science unit at the Frankfurt University of Applied Sciences. Each project description includes the required additional hardware and software components, implementation steps, and associated costs.

\colger

Ab Kapitel~4 werden verschiedene Anwendungen künstlicher Intelligenz vorgestellt, die FPV-Drohnen zur Datenerfassung oder als Transportplattform nutzen. Vorgestellt werden Anwendungen aus Forschungsprojekten, Lehrveranstaltungen und Abschlussarbeiten der Lehreinheit Informatik der Frankfurt University of Applied Sciences. Jede Projektbeschreibung enthält Angaben zu den zusätzlich benötigten Hard- und Softwarekomponenten, zu den Umsetzungsschritten und zu den entstehenden Kosten.

\coleng

The components typically used to build FPV drones are not capable of providing AI functionality on their own. Flight controllers lack the computing and memory resources required to run AI applications locally. Additional hardware and software are therefore necessary. These components may be integrated into the drone or remain on the ground to enable data processing and control. The respective chapters discuss different implementation approaches, their advantages and disadvantages, and the associated costs.

\colger

Die üblicherweise zum Bau von FPV-Drohnen verwendeten Komponenten können keine Funktionalität im Bereich künstlicher Intelligenz bereitstellen. Flugcontroller verfügen nicht über die erforderlichen Rechen- und Speicherressourcen, um Anwendungen künstlicher Intelligenz lokal auszuführen. Zusätzliche Hard- und Software ist daher notwendig. Diese kann entweder in die Drohne integriert werden oder am Boden verbleiben, um Daten zu verarbeiten und die Drohne zu steuern. In den folgenden Kapiteln werden verschiedene Möglichkeiten der Realisierung mit ihren Vorteilen, Nachteilen und Kosten beschrieben.

\coleng

We greatly appreciate your comments and suggestions for improvement.

\colger

Über Ihre Kommentare und Verbesserungsvorschläge freuen wir uns sehr.

\coleng

\vspace*{1cm}

\colger

\vspace*{1cm}

\coleng

{\noindent}%
Frankfurt am Main{\hfill}

\colger

\begin{flushright} {\noindent}%
\sffamily
\textit{Christian Baun},\\
\textit{Theodor Bloch},\\
\textit{Matthias Deegener},\\ 
\textit{Oliver Hahm},\\
\textit{Martin Kappes},\\ 
\textit{Nur Uddun Syeed}
\end{flushright}

\colende

