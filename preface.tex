\setcounter{page}{5}


\renewcommand{\deutschertitel}{Vorwort}
\renewcommand{\englischertitel}{Preface}

\chapter*{}

% \end{paracol}
\begin{paracol}{2}[]

{\raggedright\huge\bfseries\sffamily \englischertitel \par\ } \\[1.8ex]

\switchcolumn

{\raggedright\huge\bfseries\sffamily \deutschertitel \par\ } \\[1.8ex]

\coleng

TBD

\colger

Dieses Dokument bietet einen Einstieg in das komplexe Thema Drohnen mit künstlicher Intelligenz. Schwerpunkte sind die Entwicklung (inkl. Auswahl geeigneter Hard- und Softwarekomponenten), Bau und Betrieb von Drohnen in der Lehre und für Forschungsprojekte.  

\coleng

TBD

\colger

Beim Schreiben dieses Dokuments flossen Erkenntnisse aus dem vom Connectom Vernetzungs- und Innovationsfond des hessian.AI geförderten Forschungsprojekt \textsl{KI-gestützte Drohnenplattform} und aus der Lehrveranstaltung \textsl{Drohnen mit Künstlicher Intelligenz} an der Frankfurt University of Applied Sciences an.

\coleng

TBD

\colger

Maßgebliche Kriterien der Auswahl der in dieses Dokument vorgestellten Komponenten sind unter anderem: Anpassbarkeit an verschiedenste Einsatzszenarien, Anschaffungspreis, Robustheit, langfristige Marktverfügbarkeit sowie Qualität der Dokumentation und Herstellersupport.

\coleng

TBD

\colger

Die Realisierung einer vollständigen Abhandlung zu den Themen Drohnen und KI ist nicht Ziel dieses Dokument. Der Fokus liegt auf den Technologien und Lösungen, die während der Erstellung aktuell waren und mit denen praktische Erfahrung im Studienfeld Informatik des Fachbereich 2 an der Frankfurt University of Applied Sciences gemacht wurden.

\coleng

TBD

\colger

Über Ihre Kommentare und Verbesserungsvorschläge freuen wir uns sehr.

\colend

{\vspace{\baselineskip}}%

\textit{Christian Baun}


