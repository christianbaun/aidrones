\renewcommand{\deutschertitel}{Software zum Betrieb von FPV-Drohnen}
\renewcommand{\englischertitel}{Software for Using FPV Drones}
\chapter[\protect{\vspace{2pt}\englischertitel}]{}
\kapitel{\deutschertitel}

\label{KapitelSoftware}

\begin{paracol}{2}[]

{\raggedright\huge\bfseries\sffamily \englischertitel \par\ } \\[1.8ex]

\switchcolumn

{\raggedright\huge\bfseries\sffamily \deutschertitel \par\ } \\[1.8ex]

\coleng

TBD

\colger

Dieses Kapitel stellt die wichtigen Software-Komponenten zum Betrieb von FPV-Drohnen vor. Genau wie Kapitel~\ref{KapitelHardware} hat auf dieses Kapitel nicht den Anspruch, einen vollständigen Überblick über den Stand der verfügbaren Software zu geben. Vorgestellt werden eine Auswahl populärer Firmwares für Flugcontroller, Sender und Empfänger sowie einzelne nützliche Werkzeuge. Das Kapitel soll einen Überblick über die grundlegenden Funktionsumfang und sinnvolle Einsatzbereich geben und bei den ersten Schritten der Installation und Administration unterstützten.

\colende

\renewcommand{\deutschertitel}{Flight Controller Firmware}
\renewcommand{\englischertitel}{Flight Controller Firmware}
\makroabschnitt
\label{AbschnittFirmwareFC}

TBD

\colger

Jeder Flugcontroller braucht ein Betriebssystem, das in erster Linie die Sensordaten (u.a. Gyroskop, Beschleunigungssensor, GPS, magnetischer Kompass) und die Steuerbefehle des Benutzers in Motorbefehle umsetzt. Zusätzlich implementieren die Betriebssysteme für Flugcontroller verschiedene Flugmodi (z.B. Acro, Angle, Horizon) und sie leiten Datenströme der Kamera an den Videosender weiter, implementieren ein On-Screen-Display (OSD), erstellen und speichern Log-Daten (Blackbox-Logs) zur nachträglichen Analyse von Flugverhalten, ermöglichen die Feineinstellung des Flugverhaltens mit Hilfe von Filtern und Mixern und bieten verschiedene Schnittstellen zur Kommunikation und Konfiguration. 

\coleng

TBD

\colger

Das Betriebssystem des Flugcontrollers implementiert die Firmware, die im Flash-Speicher abgelegt ist. Da alle gängigen Flugcontrollers-Firmwares Open-Source-Projekte sind, passiert es alle paar Jahre, dass sich Entwicklergruppen abspalten und Forks initiieren, die durch neue Features und aktivere Weiterentwicklung mittelfristig eventuell dem Projekt, aus dem Sie hervor gekommen sind, verdrängen. Auch passiert es, dass Projekte durch den Fortgang von Entwicklern oder veränderte Lebensentwürfe nicht mehr weiterentwickelt werden und mit der Zeit Obsolet werden.

\coleng

TBD

\colger

Dieses Dokument stellt die Flugcontroller-Firmwares Betaflight, INAV und ArduPilot vor.

\colende

\renewcommand{\deutschertitel}{Betaflight}
\renewcommand{\englischertitel}{Betaflight}
\makrounterabschnitt
\label{AbschnittBetaflight}

TBD

\colger

Betaflight ist eine Firmware für Flugcontroller von FPV-Drohnen. Die Installation und Konfiguration der Firmware geschieht bis Betaflight 4.5 mit der lokal installierten Software Betaflight Configurator. Neue Versionen verwenden hierfür die Webanwendung \url{app.betaflight.com}.

\coleng

TBD

\colger

Der Fokus bei der Entwicklung von Betaflight ist die Unterstützung des manuellen Fliegens (Freestyle, Racing, etc.). Ziele von Betaflight sind schnelle Reaktion auf Benutzereingaben und umfangreiche Einstellungsmöglichkeiten um das Flugerhalten zu optimieren (tuning). 

Einstellungsmöglichkeiten Autonomes Fliegen unterstützt Betaflight -- ganz im  Gegensatz zu den alternativem Firmwares INAV oder ArduPilot -- fast gar nicht. Ausnahmen sind der Rettungsmodus (\textsl{Rescue Mode}), der die Drohne im Fall von Verbindungsproblemen mit Hilfe eines GPS-Moduls automatisch zurückzukehren lässt, und ein experimenteller Modus um Höhe und Position automatisch zu halten. 
 
\coleng

TBD

\colger

Betaflight ist kompatibel zu praktisch allen am Markt verfügbaren Flugcontrollern und Empfängern und bietet eine intuitive Benutzeroberfläche. 

\colende

\renewcommand{\deutschertitel}{Installation}
\renewcommand{\englischertitel}{Installation}
\makrounterunterabschnitt
\label{AbschnittInstallationBetaflight}

TBD

\colger

TBD

\colende

\renewcommand{\deutschertitel}{Administration}
\renewcommand{\englischertitel}{Administration}
\makrounterunterabschnitt
\label{AbschnittAdministrationBetaflight}

TBD

\colger

TBD

\colende

\renewcommand{\deutschertitel}{INAV}
\renewcommand{\englischertitel}{INAV}
\makrounterabschnitt
\label{AbschnittINAV}

TBD

\colger

INAV ist ein Fork von einer früheren Cleanflight-Version. Cleanflight ist ein verwandtes Projekt, welches allerdings nicht mehr weiterentwickelt wird. INAV besteht aus zwei Hauptkomponenten: Die Firmware, welche auf dem Flugcontrollerboard läuft und dem Konfigurationstool, welches für die Installation und Aministration genutzt werden kann.

Neben der Return-To-Home- (RTH) bzw. Rescuefunktionalität bietet INAV weitere Autopilotenfunktionalitäten, wie Wegpunktmissionen, \textsl{Position halten}, \textsl{Flughöhe halten} und mehr. Diese können auch ohne Kompass genutzt werden (seit Version Version 7.1), jedoch mit geringerer Genauigkeit.

Die Software erhält jedes Jahr einen großen Release und kleinere Support-Updates wie sie benötigt werden. Die aktuelle Version 8 unterstützt ebenfalls eine breite Spanne an gängigen Flugcontrollerchips, jedoch nicht ganz so viele wie Betaflight.

\colende

\renewcommand{\deutschertitel}{Installation}
\renewcommand{\englischertitel}{Installation}
\makrounterunterabschnitt
\label{AbschnittInstallationINAV}

TBD

\colger

Die Installation erfolgt über den \textsl{INAV Configurator}. Hier ist die Verwandschaft zu Betaflight sehr deutlich sichtbar, da der Aufbau sehr ähnlich zum Betaflight Configurator ist, welcher für die Versionen bis 4.5 genutzt wird. Dieser Guide geht davon aus, dass der Flugcontroller via USB an den Computer angeschlossen ist, mit welchem das Flashing durchgeführt wird.

Wird die Software nicht zum ersten Mal auf dem Flugcontroller installiert, sollte ein Backup der Konfiguration gemacht werden, da manche Upgrades die Konfiguration zurücksetzen. Dies geschieht über den \textsl{CLI}-Tab. Der Befehl \texttt{diff all} gibt die entsprechenden Parameter aus. Mit dem Button \textsl{Save to File} können diese dann in eine Textdatei gespeichert werden. Danach besteht der Prozess aus den folgenden Schritten:

\begin{enumerate}
	\item \textsl{Device Firmware Upgrade (DFU)-Modus} aktivieren ist der erste notwendige Schritt. Dies kann in der Regel über einen Button am Flugcontrollerboard realisiert werden, oder mit dem Befehl \texttt{dfu} im \textsl{CLI}-Tab.

	\item  Befindet sich das Board im DFU-Modus, so muss der \textsl{Firmware Flasher} Tab im INAV Configurator aufgerufen werden. In diesem muss zunächst das Target ausgewählt werden, sollte  \textsl{Auto-select Target} nicht funktionieren muss der Modellname händisch in der Liste gesucht werden.

	\item Im Ausklappmenü unter dem Board werden dann die verfügbaren (stable) Releases angezeigt. Mit der Option \textsl{Show unstable releases} können noch aktuellere Versionen gewählt werden. Diese funktionieren aber unter Umständen nicht wie erwartet.

	\item INAV bietet drei Optionen die für das Flashing gesetzt werden können:
	\begin{itemize}
		\item \textsl{No reboot sequence} - Wird nur benötigt, wenn die Bootpins überbrückt sind oder der Bootbutton gefrückt wird (modellabhängig).
		\item \textsl{Full chip erase} - Löscht alle Konfigurationsoptionen (Backup Vorhanden?). Sollte immer gesetzt werden, wenn vorher eine andere Software installiert war.
		\item \textsl{Manual baud rate} - Festlegen der Baudrate für Bluetooth oder falls ein Modell die Standardgeschwindigkeit nicht unterstützt.
	\end{itemize}

	\item Der Prozess des Flashens besteht aus zwei Schritten: Zunächst wird die Firmware geladen (online oder über ein File), dann wird das eigentliche Flashing über einen separaten Button gestartet. Das eigentliche Flashing sollte nicht unterbrochen werden.

	\item TODO: Backup wieder einspielen.

\end{enumerate}

\colende

\renewcommand{\deutschertitel}{Administration}
\renewcommand{\englischertitel}{Administration}
\makrounterunterabschnitt
\label{AbschnittAdministrationINAV}

TBD

\colger
% TODO: Add chapter for PID-Tuning guide and for INAV-SITL?
Die Administration von INAV erfolgt ebenfalls über den \textsl{INAV Configurator}. Wenn die Software das erste mal installiert wurde, müssen einige Punkte konfiguriert werden, um das Fliegen initial zu ermöglichen.

Wird INAV zum ersten Mal nach der Installation aufgerufen erscheinen Fenster zum Setzen der \textsl{Default-Values} (Presets). Hier gilt es das Preset auszuwählen, was am nächsten an dem gebauten Modell dran ist. Danach können die UART-Ports (für GPS, VTX, etc.) konfiguriert werden. Dies kann allerdings auch über den Ports-Tab durchgeführt werden. Die weiteren Konfigurationsschritte gestalten sich wie folgt (TODO):

\begin{enumerate}

	\item Status-Tab

	\item Calibration-Tab

	\item Mixer-Tab

	\item Outputs-Tab

	\item Ports-Tab

	\item Configuration-Tab

	\item Failsafe-Tab

	\item Receiver-Tab

	\item GPS-Tab

	\item Modes-Tab

	\item OSD

\end{enumerate}

Darüber hinaus gibt es erweiterte Konfigurationsmöglichkeiten um die Performance der Drohne zu optimieren. Diese sind im folgenden kurz beschrieben (TODO):

\begin{enumerate}

	\item Tuning-Tab

	\item Advanced-Tuning Tab

	\item Programming

	\item Adjustments-Tab

	\item Alignment Tool

	\item Mission Control

	\item LED Strip

	\item Sensors

	\item Tethered Logging

	\item Blackbox

	\item CLI

\end{enumerate}

\colende

\renewcommand{\deutschertitel}{ArduPilot}
\renewcommand{\englischertitel}{ArduPilot}
\makrounterabschnitt
\label{AbschnittArduPilot}

TBD

\colger

ArduPilot legt als Flugcontroller-Firmware den Fokus auf vollautonomen Flug und komplexe Missionsprofile. Es ermöglicht Missionsplanung mit Wegpunkten, automatische Starts und automatisches Landen, Hindernisvermeidung, Follow-Me, etc. Hierfür nutzt ArduPilot nicht nur die üblichen Sensoren wie GPS, magnetischer Kompass, Geschwindigkeit, sondern kann auch Lidar-Sensoren zur Entfernungen und Kameras einbeziehen.

\coleng

TBD

\colger

Von den in diesem Dokument vorgestellten Flugcontroller-Firmwares braucht ArduPilot am meisten Speicher. Idealerweise sind zum Betrieb von ArduPilot 2\,MB Flash-Speicher verfügbar. Das bieten nur Flugcontroller mit einem H743 Mikrocontroller. Mit 1\,MB Flash-Speicher (F405 und F745 Mikrocontroller) ist der Betrieb mit einem reduzierten Funktionsumfang möglich. Mit nur 512\,kB (F411 und F722 Mikrocontroller) kann ArduPilot gar nicht verwendet werden.

\colende

\renewcommand{\deutschertitel}{Installation}
\renewcommand{\englischertitel}{Installation}
\makrounterunterabschnitt
\label{AbschnittInstallationArduPilot}

TBD

\colger

TBD

\colende

\renewcommand{\deutschertitel}{Administration}
\renewcommand{\englischertitel}{Administration}
\makrounterunterabschnitt
\label{AbschnittAdministrationArduPilot}

TBD

\colger

TBD

\colende

\renewcommand{\deutschertitel}{Fernbedienung Firmware}
\renewcommand{\englischertitel}{Remote Control Firmware}
\makroabschnitt
\label{AbschnittFirmwareRemoteControl}

TBD

\colger

TBD

\colende

\renewcommand{\deutschertitel}{EdgeTX}
\renewcommand{\englischertitel}{EdgeTX}
\makrounterabschnitt
\label{AbschnittEdgeTX}

TBD

\colger

TBD

\colende

\renewcommand{\deutschertitel}{Installation}
\renewcommand{\englischertitel}{Installation}
\makrounterunterabschnitt
\label{AbschnittInstallationEdgeTX}

TBD

\colger

TBD

\colende

\renewcommand{\deutschertitel}{Sendemodul- und Empfänger-Firmware}
\renewcommand{\englischertitel}{Transmitter and Receiver Firmware}
\makroabschnitt
\label{AbschnittFirmwareReceiverTransmitter}

TBD

\colger

TBD

\colende

\renewcommand{\deutschertitel}{ExpressLRS}
\renewcommand{\englischertitel}{ExpressLRS}
\makrounterabschnitt
\label{AbschnittExpressLRS}

TBD

\colger

TBD

\colende

\renewcommand{\deutschertitel}{Installation}
\renewcommand{\englischertitel}{Installation}
\makrounterunterabschnitt
\label{AbschnittInstallationExpressLRS}

TBD

\colger

TBD

\colende
