\renewcommand{\deutschertitel}{Software zum Betrieb von FPV-Drohnen}
\renewcommand{\englischertitel}{Software for Using FPV Drones}
\chapter[\protect{\vspace{2pt}\englischertitel}]{}
\kapitel{\deutschertitel}

\label{KapitelSoftware}

\begin{paracol}{2}[]

{\raggedright\huge\bfseries\sffamily \englischertitel \par\ } \\[1.8ex]

\switchcolumn

{\raggedright\huge\bfseries\sffamily \deutschertitel \par\ } \\[1.8ex]

\coleng

TBD

\colger

Dieses Kapitel stellt die wichtigen Software-Komponenten zum Betrieb von FPV-Drohnen vor. Genau wie Kapitel~\ref{KapitelHardware} hat auf dieses Kapitel nicht den Anspruch, einen vollständigen Überblick über den Stand der verfügbaren Software zu geben. Vorgestellt werden eine Auswahl populärer Firmwares für Flugcontroller, Sender und Empfänger sowie einzelne nützliche Werkzeuge. Das Kapitel soll einen Überblick über die grundlegenden Funktionsumfang und sinnvolle Einsatzbereich geben und bei den ersten Schritten der Installation und Administration unterstützten.

\colende

\renewcommand{\deutschertitel}{Flight Controller Firmware}
\renewcommand{\englischertitel}{Flight Controller Firmware}
\makroabschnitt
\label{AbschnittFirmwareFC}

TBD

\colger

Jeder Flugcontroller braucht ein Betriebssystem, das in erster Linie die Sensordaten (u.a. Gyroskop, Beschleunigungssensor, GPS, magnetischer Kompass) und die Steuerbefehle des Benutzers in Motorbefehle umsetzt. Zusätzlich implementieren die Betriebssysteme für Flugcontroller verschiedene Flugmodi (z.B. Acro, Angle, Horizon) und sie leiten Datenströme der Kamera an den Videosender weiter, implementieren ein On-Screen-Display (OSD), erstellen und speichern Log-Daten (Blackbox-Logs) zur nachträglichen Analyse von Flugverhalten, ermöglichen die Feineinstellung des Flugverhaltens mit Hilfe von Filtern und Mixern und bieten verschiedene Schnittstellen zur Kommunikation und Konfiguration. 

\coleng

TBD

\colger

Das Betriebssystem des Flugcontrollers implementiert die Firmware, die im Flash-Speicher abgelegt ist. Da alle gängigen Flugcontrollers-Firmwares Open-Source-Projekte sind, passiert es alle paar Jahre, dass sich Entwicklergruppen abspalten und Forks initiieren, die durch neue Features und aktivere Weiterentwicklung mittelfristig eventuell dem Projekt, aus dem Sie hervor gekommen sind, verdrängen. Auch passiert es, dass Projekte durch den Fortgang von Entwicklern oder veränderte Lebensentwürfe nicht mehr weiterentwickelt werden und mit der Zeit Obsolet werden.

\coleng

TBD

\colger

Dieses Dokument stellt die Flugcontroller-Firmwares Betaflight, INAV und ArduPilot vor.

\colende

\renewcommand{\deutschertitel}{Betaflight}
\renewcommand{\englischertitel}{Betaflight}
\makrounterabschnitt
\label{AbschnittBetaflight}

TBD

\colger

Betaflight ist eine Firmware für Flugcontroller von FPV-Drohnen. Der Fokusvon Betaflight ist die Unterstützung des manuellen Fliegens (Freestyle, Racing, etc.). Ziele von Betaflight sind schnelle Reaktion auf Benutzereingaben und umfangreiche Einstellungsmöglichkeiten um das Flugerhalten zu optimieren (tuning). 

\coleng

TBD

\colger

Einstellungsmöglichkeiten Autonomes Fliegen unterstützt Betaflight -- ganz im  Gegensatz zu den alternativem Firmwares INAV oder ArduPilot -- fast gar nicht. Ausnahmen sind der Rettungsmodus (\textsl{Rescue Mode}), der die Drohne im Fall von Verbindungsproblemen mit Hilfe eines GPS-Moduls automatisch zurückzukehren lässt, und ein experimenteller Modus um Höhe und Position automatisch zu halten. 
 
\coleng

TBD

\colger

Betaflight ist kompatibel zu praktisch allen am Markt verfügbaren Flugcontrollern und Empfängern und bietet eine intuitive Benutzeroberfläche. 

\colende

\renewcommand{\deutschertitel}{Installation und Konfiguration}
\renewcommand{\englischertitel}{Installation and Configuration}
\makrounterunterabschnitt
\label{AbschnittInstallationBetaflight}

TBD

\colger

Die Installation und Konfiguration der Firmware geschieht mit der lokal installierten Software Betaflight Configurator oder mit der Webanwendung \url{app.betaflight.com}. Abbildung~\ref{AbbildungBetaflighInstallation} zeigt die Möglichkeit der Installation von Betaflight über die Webanwendung. Das Modell des Flugcontrollers wird üblicherweise automatisch erkannt. Diese Seite erreicht man über den Button \textsl{Update Firmware} am oberen Rand der Webseite. Ein bestehendes Backup von Einstellungen kann auch auf dieser Seite in den Flugcontroller eingespielt werden. Ein Backup der aktuellen Einstellungen wird bei der Installation einer neuen Version automatisch angefertigt.

\colende

\begin{figure}[htb!]
  \centering
    \includegraphics[width=\linewidth]{Betaflight_4_5_3_installation_update_backup_firmware_screenshot_app.png}
  \caption{Installation of Betaflight and Option to Backup / Restore of Configuration Settings}
  \label{AbbildungBetaflighInstallation}
\end{figure}

\begin{figure}[htb!]
  \centering
    \includegraphics[width=\linewidth]{Betaflight_4_5_3_hauptseite_firmware_screenshot_app.png}
  \caption{Main Page of Betaflight}
  \label{AbbildungBetaflightAppHauptseite}
\end{figure}

\colstart

TBD

\colger

Abbildung~\ref{AbbildungBetaflightAppHauptseite} zeigt die Startseite (\textsl{Setup}) von Betaflight beim Zugriff aus der Webanwendung heraus. Auf der Startseite können der Beschleunigungssensor und der magnetische Kompass kalibriert werden. Die Startseite enthält auch generelle Informationen zum Zustand der Drohne und ihrer Komponenten. Ein wertvolles Werkzeug ist die Startseite bei der Kontrolle der korrekten Konfiguration der Ausrichtung des Flugcontrollers im Rahmen. Häufig ergibt es sich bei Flugcontroller-Stacks und bei kleineren Drohnen, dass der Flugcontroller gedreht oder auf dem Kopf stehend im Rahmen verbaut ist. Die Startseite zeigt die Position der Drohne als Live-Bild entsprechen der Konfiguration der Position des Flugcontrollers in der Seite \textsl{Configuration}.

\coleng

TBD

\colger

Abbildung~\ref{AbbildungBetaflightAppPorts} zeigt die Seite \textsl{Ports}. Hier wird die Konfiguration der verfügbaren UART-Schnittstellen vorgenommen. Die Korrekte Einstellung der Funktionswiese der UARTs ist Voraussetzung für die Nutzung vieler an der Drohne angeschlossenen Komponenten wie zum Beispiel Videosender (in Abbildung~\ref{AbbildungBetaflightAppPorts} an UART3), GPS-Modul (in Abbildung~\ref{AbbildungBetaflightAppPorts} an UART5) und Empfänger (in Abbildung~\ref{AbbildungBetaflightAppPorts} an UART6). 

\colende

\begin{figure}[htb!]
  \centering
    \includegraphics[width=\linewidth]{Betaflight_4_5_3_ports_firmware_screenshot_app.png}
  \caption{Ports Page of Betaflight}
  \label{AbbildungBetaflightAppPorts}
\end{figure}

\colstart

TBD

\colger

Auf der Seite \textsl{Configuration} (siehe Abbildungen~\ref{AbbildungBetaflightAppConfiguration1} und~\ref{AbbildungBetaflightAppConfiguration2}) werden unter anderem Einstellungen zur Ausrichtung des Flugcontrollers und des für jede Drohne unverzichtbaren Gyroskops (Kreiselinstrument zur Lagebestimmung) vorgenommen. 

\colende

\begin{figure}[htb!]
  \centering
    \includegraphics[width=\linewidth]{Betaflight_4_5_3_configuration1_firmware_screenshot_app.png}
  \caption{Configuration Page of Betaflight (Part 1/2)}
  \label{AbbildungBetaflightAppConfiguration1}
\end{figure}

\colstart

TBD

\colger

Auch sicherheitsrelevante Einstellungen wie der maximale Aufstellwindel der Drohne, bei der das Arming (das Aktivieren der Motoren, um sie für den Flug vorzubereiten) überhaupt möglich ist, werden auf dieser Seite \textsl{Configuration} festgelegt. Soll im Livebild der FPV-Drohne durch den Flugcontroller ein on-screen display (OSD) eingeblendet werden, muss es hier grundsätzlich aktiviert werden. Die in Abbildung~\ref{AbbildungBetaflightAppConfiguration2} aktivierte Einstellung Airmode sorgt dafür, dass die Motoren auch bei 0 Throttle leicht weiterlaufen, was die Stabilität der Drohne und die Kontrollierbarkeit verbessert.

\colende

\begin{figure}[htb!]
  \centering
    \includegraphics[width=\linewidth]{Betaflight_4_5_3_configuration2_firmware_screenshot_app.png}
  \caption{Configuration Page of Betaflight (Part 2/2)}
  \label{AbbildungBetaflightAppConfiguration2}
\end{figure}


\colstart

TBD

\colger

Die Seite \textsl{Power \& Battery} erlaubt die Spannungen zu definieren, bei denen der Akku der vollständig geladen und entladen ist und bei welcher Spannungen der Flugcontroller entsprechend warnen soll. 

\coleng

TBD

\colger

Auf der Seite \textsl{Failsafe} kann sehr detailliert festgelegt werden, wie der Flugcontroller reagieren soll, wenn der Drohnennpilot ein Problem anzeigt oder wenn es zum Signalverlust kommt. Sinnvolle Einstellungen hier können sein, dass die Drohne kontrolliert landen oder alternativ mit Hilfe des GPS-Moduls selbständig zum Startpunkt zurückfliegen soll.

\coleng

TBD

\colger

Die Seite \textsl{Presets} ermöglicht es die eigene Betaflight-Konfiguration zu speichern und zu wiederherzustellen. Zudem ist es möglich, eine Datenbank von veröffentlichten Konfiguration nach verschiedenen Parametern zu durchsuchen und auf diese zuzugreifen. 

\coleng

TBD

\colger

Auf der Seite \textsl{PID Tuning} (Proportional-Integral-Derivative) können Feineinstellungen in der Motorsteuerung durch den Flugcontrollers vorgenommen werden. Ziel hierbei ist meist ein präziseres oder Flugverhalten zu erreichen und unerwünschtem Verhalten der Drohne entgegenzuwirken, wie zum Beispiel Zittern oder zu langsamem oder zu aggressivem Reagieren auf Steuerbefehle.

\coleng

TBD

\colger

Die korrekte Funktionsweise des Empfängers kann auf der Seite \textsl{Receiver} (siehe Abbildung~\ref{AbbildungBetaflightAppReceiver}) kontrolliert werden.  Bewegungen an den Schaltern des Senders werden, wenn sie vom Empfänger empfangen werden, hier angezeigt. Auch die Anschlussart des Empfängers und das verwendete Protokoll sind hier definiert. 

\coleng

TBD

\colger

Soll der Empfänger an den Sender Telemtriedaten übertragen, muss der entsprechende Regler auf der Seite \textsl{Receiver} aktiviert sein. Zu den Telemtriedaten gehören u.a. Informationen zu Akkuspannung und Stromverbrauch, Signalqualität, GPS-Informationen (Breitengrad, Längengrad, Geschwindigkeit, Anzahl der gefundenen Satelliten), Höhe und Temperatur.

\colende

\begin{figure}[htb!]
  \centering
    \includegraphics[width=\linewidth]{Betaflight_4_5_3_receiver_firmware_screenshot_app.png}
  \caption{Receiver Page of Betaflight}
  \label{AbbildungBetaflightAppReceiver}
\end{figure}

\colstart

TBD

\colger

Die Verknüpfung wichtiger Funktionen der Drohne an einzelne Schalter des Senders geschieht auf der Seite \textsl{Modes}. Die wichtigste Funkion hier ist das Anlaufen (\glqq scharf schalten\grqq) der Motoren, um sie für den Flug vorzubereiten. Weitere Funktionen, die sinnvollerweise Schaltern des Senders zugeordnet werden, sind unter anderem die Flugmodi (Angle, Horizon und Acro), der Pieper (Buzzer), um abgestürzte Drohnen leichter wiederzufinden und die automatische Rückkehr (GPS Rescue) zum Startpunkt.

\coleng

Sehr häufig wir der der Schalter SA (AUX1) mit dem Arming belegt. Da bei nur einem Schalter auch ein versehentliches Arming vorkommen kann, kommen Verletzungen z.B. an Händen und Fingern häufiger vor. Zur Verbesserung der Sicherheit beim Halten der Drohne empfiehlt es daher das Arming mit zwei Schalten -- z.B. SA (AUX1) + SB (AUX4) -- die nicht direkt nebeneinander liegen, zu verknüpfen.

\colende


\begin{figure}[htb!]
  \centering
    \includegraphics[width=\linewidth]{Betaflight_4_5_3_modes_firmware_screenshot_app.png}
  \caption{Modes Page of Betaflight}
  \label{AbbildungBetaflightAppModes}
\end{figure}

\renewcommand{\deutschertitel}{INAV}
\renewcommand{\englischertitel}{INAV}
\makrounterabschnitt
\label{AbschnittINAV}

TBD

\colger

INAV ist ein Fork von einer früheren Cleanflight-Version. Cleanflight ist ein verwandtes Projekt, welches allerdings nicht mehr weiterentwickelt wird. INAV besteht aus zwei Hauptkomponenten: Die Firmware, welche auf dem Flugcontroller läuft und dem Konfigurationstool, welches für die Installation und Administration genutzt werden kann.

Neben der Return-To-Home- (RTH) bzw. Rescuefunktionalität bietet INAV weitere Autopilotenfunktionalitäten, wie Wegpunktmissionen, \textsl{Position halten}, \textsl{Flughöhe halten} und mehr. Diese können auch ohne Kompass genutzt werden (seit Version 7.1), jedoch mit geringerer Genauigkeit.

Die Software erhält jedes Jahr einen großen Release und kleinere Support-Updates, wie sie benötigt werden. Die aktuelle Version 8 unterstützt ebenfalls eine breite Spanne an gängigen Flugcontrollerchips, jedoch nicht ganz so viele wie Betaflight.

\colende

\renewcommand{\deutschertitel}{Installation}
\renewcommand{\englischertitel}{Installation}
\makrounterunterabschnitt
\label{AbschnittInstallationINAV}

TBD

\colger

Die Installation erfolgt über den \textsl{INAV Configurator}. Hier ist die Verwandtschaft zu Betaflight sehr deutlich sichtbar, da der Aufbau sehr ähnlich zum Betaflight Configurator ist, welcher für die Versionen bis 4.5 genutzt wird. Dieser Guide geht davon aus, dass der Flugcontroller via USB an den Computer angeschlossen ist, mit welchem das Flashen durchgeführt wird.

Wird die Software nicht zum ersten Mal auf dem Flugcontroller installiert, sollte ein Backup der Konfiguration gemacht werden, da manche Upgrades die Konfiguration zurücksetzen. Dies geschieht über den \textsl{CLI}-Tab. Der Befehl \texttt{diff all} gibt die entsprechenden Parameter aus. Mit dem Button \textsl{Save to File} können diese dann in eine Textdatei gespeichert werden. Danach besteht der Prozess aus den folgenden Schritten:

\begin{enumerate}
	\item \textsl{Device Firmware Upgrade (DFU)-Modus} aktivieren ist der erste notwendige Schritt. Dies kann in der Regel über einen Button am Flugcontrollerboard realisiert werden oder mit dem Befehl \texttt{dfu} im \textsl{CLI}-Tab.

	\item Befindet sich das Board im DFU-Modus, so muss der \textsl{Firmware Flasher} Tab im INAV Configurator aufgerufen werden. In diesem muss zunächst das Target ausgewählt werden, sollte  \textsl{Auto-select Target} nicht funktionieren, muss der Modellname händisch in der Liste gesucht werden.

	\item Im Ausklappmenü unter dem Board werden dann die verfügbaren (stable) Releases angezeigt. Mit der Option \textsl{Show unstable releases} können noch aktuellere Versionen gewählt werden. Diese funktionieren aber unter Umständen nicht wie erwartet.

	\item INAV bietet drei Optionen die für das Flashen gesetzt werden können:
	\begin{itemize}
		\item \textsl{No reboot sequence} - Wird nur benötigt, wenn die Bootpins überbrückt sind oder der Bootbutton gedrückt wird (modellabhängig).
		\item \textsl{Full chip erase} - Löscht alle Konfigurationsoptionen (Backup Vorhanden?). Sollte immer gesetzt werden, wenn vorher eine andere Software installiert war.
		\item \textsl{Manual baud rate} - Festlegen der Baudrate für Bluetooth oder falls ein Modell die Standardgeschwindigkeit nicht unterstützt.
	\end{itemize}

	\item Der Prozess des Flashens besteht aus zwei Schritten: Zunächst wird die Firmware geladen (online oder über ein File), dann wird das eigentliche Flashen über einen separaten Button gestartet. Das eigentliche Flashen sollte nicht unterbrochen werden.

	\item Gegebenenfalls kann nun das Backup über den \textsl{CLI}-Tab wieder eingespielt werden. Dies geschieht über den Button \textsl{Load from File}. Damit kann die entsprechende Textdatei herausgesucht werden, um sie mit dem \textsl{Execute}-Button auszuführen.

\end{enumerate}

\colende

\renewcommand{\deutschertitel}{Administration}
\renewcommand{\englischertitel}{Administration}
\makrounterunterabschnitt
\label{AbschnittAdministrationINAV}

TBD

\colger
Die Administration von INAV erfolgt ebenfalls über den \textsl{INAV Configurator}. Wenn die Software das erste Mal installiert wurde, müssen einige Punkte konfiguriert werden, um das Fliegen initial zu ermöglichen.

Wird der Configurator zum ersten Mal nach der Installation aufgerufen, erscheinen Fenster zum Setzen der \textsl{Default-Values} (Presets). Hier gilt es das Preset auszuwählen, was am nächsten an dem gebauten Modell dran ist. Danach können die UART-Ports (für GPS, VTX, etc.) konfiguriert werden. Dies kann allerdings auch über den Ports-Tab durchgeführt werden. Die weiteren Konfigurationsschritte gestalten sich wie folgt:

\begin{enumerate}

	\item Der \textsl{Status-Tab} stellt einen Überblick zur Verfügung. Unter anderem ist in der Übersicht \textsl{Pre-arming checks} zu sehen, ob die Drohne abheben kann. Eventuell muss hier vorher die Boardorientierung angepasst werden. Dazu wird die Drohne aufrecht mit Kamera Richtung Bildschirm des konfigurierenden Computers gestellt, um dann den \textsl{Reset Z axis} Button zu drücken.

	\item Im \textsl{Calibration-Tab} werden der Beschleunigungsmesser und der Kompass kalibriert. Dazu muss die Drohne jeweils in unterschiedliche Posen gebracht werden.

	\item Der \textsl{Mixer-Tab} konfiguriert zusammen mit dem Outputs-Tab die Motoren der Drohne. Hier ist entscheidend, dass der richtige  \textsl{Platform type} und das richtige \textsl{Mixer preset} entsprechend des Typs der verwendeten Drohne konfiguriert sind. Sollten die Motoren nicht in die richtige Richtung drehen, kann die diese über die \textsl{Motor direction} angepasst werden und die Zuordnung über den \textsl{Motor Mixer Wizard}.

	\item Im \textsl{Outputs-Tab} werden die Motoren aktiviert mit der Option \textsl{Enable motor and servo output}. Außerdem sollte sichergestellt werden, dass  \texttt{DSHOT300} als \textsl{ESC protocol} verwendet wird. Außerdem kann die \textsl{Motors IDLE power} auf 5\% gesetzt werden, wenn eine Multirotor-Drohne verwendet wird.

	\item Im \textsl{Ports-Tab} werden die Peripheriegeräte konfiguriert, insofern dies nicht im Pop-Up-Menü am Anfang gemacht wurde. Wie diese konfiguriert werden, hängt von den angeschlossenen Geräten und vom Flugcontrollerchip ab. Meist kann die Konfiguration für Betaflight aus den jeweiligen Handbüchern entnommen und übertragen werden.

	\item Im \textsl{Configuration-Tab} können die Sensoren aktiviert werden, insofern dies nicht nach Konfiguration der Ports automatisch geschehen ist. Wenn möglich, sollte hier der \textsl{I2C Speed} auf \texttt{800KHZ} gestellt werden.

	\item Der \textsl{Failsafe-Tab} ist sehr wichtig, da in diesem das Verhalten definiert ist, sollte das Fernsteuerungssignal abreißen. Hier sollte sichergestellt werden, dass entweder textsl{Drop}, \textsl{Land} oder \textsl{Return-to-Home (RTH)} konfiguriert sind, damit die Drohne nicht unkontrolliert weiterfliegt.

	\item Im \textsl{Receiver-Tab} wird Flugcontrollerseitig die Verbindung zur Fernbedienung konfiguriert. Wird eine ExpressLRS-Fernbedienung verwendet, muss \texttt{SERIAL} als \textsl{Receiver type} und \texttt{CRSF} als \textsl{Serial Receiver Provider} konfiguriert werden. Darüber hinaus kann kontrolliert werden, ob die Channel-Mappings korrekt sind.

	\item Im \textsl{GPS-Tab} muss die Option \textsl{GPS for navigation and telemetry} aktiviert werden, um das GPS nutzen zu können. Die Option \textsl{Protocol} ist entsprechend der Spezifikation des verwendeten Modells zu setzen.

	\item Im \textsl{Modes-Tab} muss grundsätzlich nur \texttt{ARM} konfiguriert werden (CH5 empfohlen). Darüber hinaus lohnt es sich auch \texttt{ANGLE} zu konfigurieren (vor allem für Fluganfänger) und \texttt{FAILSAFE} sollte sich die Drohne mal unerwartet verhalten.

	\item Das \textsl{on-screen display (OSD)} kann so konfiguriert werden, wie es vom Piloten bevorzugt wird. Nützliche Elemente, welche nicht standardmäßig eingeblendet werden sind zum Beispiel \textsl{Remaining Flight Time} und \textsl{Battery Remaining Percentage}. Je nach VTX-Modell muss in diesem Tab auch das \textsl{Video Format} konfiguriert werden.

\end{enumerate}

Darüber hinaus gibt es erweiterte Konfigurationsmöglichkeiten, um die Performance der Drohne zu optimieren.
% TODO: Kapitel für Flight-Mode-Übersicht (wie starte ich den Autopiloten?):
% TODO: Add chapter for PID-Tuning guide and for INAV-SITL?
% TODO: Restliche Tabs erklären? -> Tuning, Advanced Tuning, Programming, Adjustments, Alignment, Mission Control, LED Strip Sensors, Tethered Logging, Blackbox, CLI

\colende

\renewcommand{\deutschertitel}{ArduPilot}
\renewcommand{\englischertitel}{ArduPilot}
\makrounterabschnitt
\label{AbschnittArduPilot}

TBD

\colger

ArduPilot legt als Flugcontroller-Firmware den Fokus auf vollautonomen Flug und komplexe Missionsprofile. Es ermöglicht Missionsplanung mit Wegpunkten, automatische Starts und automatisches Landen, Hindernisvermeidung, Follow-Me, etc. Hierfür nutzt ArduPilot nicht nur die üblichen Sensoren wie GPS, magnetischer Kompass, Geschwindigkeit, sondern kann auch Lidar-Sensoren zur Entfernungen und Kameras einbeziehen.

\coleng

TBD

\colger

Von den in diesem Dokument vorgestellten Flugcontroller-Firmwares braucht ArduPilot am meisten Speicher. Idealerweise sind zum Betrieb von ArduPilot 2\,MB Flash-Speicher verfügbar. Das bieten nur Flugcontroller mit einem H743 Mikrocontroller. Mit 1\,MB Flash-Speicher (F405 und F745 Mikrocontroller) ist der Betrieb mit einem reduzierten Funktionsumfang möglich. Mit nur 512\,kB (F411 und F722 Mikrocontroller) kann ArduPilot gar nicht verwendet werden.

\colende

\renewcommand{\deutschertitel}{Installation}
\renewcommand{\englischertitel}{Installation}
\makrounterunterabschnitt
\label{AbschnittInstallationArduPilot}

TBD

\colger

TBD

\colende

\renewcommand{\deutschertitel}{Administration}
\renewcommand{\englischertitel}{Administration}
\makrounterunterabschnitt
\label{AbschnittAdministrationArduPilot}

TBD

\colger

TBD

\colende

\renewcommand{\deutschertitel}{Fernbedienung Firmware}
\renewcommand{\englischertitel}{Remote Control Firmware}
\makroabschnitt
\label{AbschnittFirmwareRemoteControl}

TBD

\colger

TBD

\colende

\renewcommand{\deutschertitel}{EdgeTX}
\renewcommand{\englischertitel}{EdgeTX}
\makrounterabschnitt
\label{AbschnittEdgeTX}

TBD

\colger

TBD

\colende

\renewcommand{\deutschertitel}{Installation}
\renewcommand{\englischertitel}{Installation}
\makrounterunterabschnitt
\label{AbschnittInstallationEdgeTX}

TBD

\colger

TBD

\colende

\renewcommand{\deutschertitel}{Sendemodul- und Empfänger-Firmware}
\renewcommand{\englischertitel}{Transmitter and Receiver Firmware}
\makroabschnitt
\label{AbschnittFirmwareReceiverTransmitter}

TBD

\colger

TBD

\colende

\renewcommand{\deutschertitel}{ExpressLRS}
\renewcommand{\englischertitel}{ExpressLRS}
\makrounterabschnitt
\label{AbschnittExpressLRS}

TBD

\colger

TBD

\colende

\renewcommand{\deutschertitel}{Installation}
\renewcommand{\englischertitel}{Installation}
\makrounterunterabschnitt
\label{AbschnittInstallationExpressLRS}

TBD

\colger

TBD

\colende
