\renewcommand{\deutschertitel}{Software}
\renewcommand{\englischertitel}{Software}

% !!! Das hier war vorher !!!
% \chapter[\englischertitel]{\englischertitel\newline\deutschertitel}
% !!! Das hier war vorher !!!


% Das vspace fügt im Inhaltsverzeichnis einen kleinen Abstand unter dem Kapiteleintrag ein.
% Beim deutschen Inhaltsverzeichnis ist es im book.tex an nur einer Stelle in der Zeile 
% \addcontentsline{deutschestoc}{chapter}{\protect{\vspace{2pt}\thechapter}~#1}}
\chapter[\protect{\vspace{2pt}\englischertitel}]{}
\kapitel{\deutschertitel}

\label{KapitelSoftware}

\begin{paracol}{2}[]

{\raggedright\huge\bfseries\sffamily \englischertitel \par\ } \\[1.8ex]

\switchcolumn

{\raggedright\huge\bfseries\sffamily \deutschertitel \par\ } \\[1.8ex]

\coleng

TBD

\colger

TBD

\colende

\renewcommand{\deutschertitel}{Flight Controller Firmware}
\renewcommand{\englischertitel}{Flight Controller Firmware}
\makroabschnitt
\label{AbschnittFirmwareFC}

TBD

\colger

Jeder Flugcontroller braucht ein Betriebssystem, das in erster Linie die Sensordaten (u.a. Gyroskop, Beschleunigungssensor, GPS, magnetischer Kompass) und die Steuerbefehle des Benutzers in Motorbefehle umsetzt. Zusätzlich implementieren die Betriebssysteme für Flugcontroller verschiedene Flugmodi (z.B. Acro, Angle, Horizon) und sie leiten Datenströme der Kamera an den Videosender weiter, implementieren ein On-Screen-Display (OSD), erstellen und speichern Log-Daten (Blackbox-Logs) zur nachträglichen Analyse von Flugverhalten, ermöglichen die Feineinstellung des Flugverhaltens mit Hilfe von Filtern und Mixern und bieten verschiedene Schnittstellen zur Kommunikation und Konfiguration. 

\coleng

TBD

\colger

Das Betriebssystem des Flugcontrollers implementiert die Firmware, die im Flash-Speicher abgelegt ist. Da alle gängigen Flugcontrollers-Firmwares Open-Source-Projekte sind, passiert es alle paar Jahre, dass sich Entwicklergruppen abspalten und Forks initiieren, die durch neue Features und aktivere Weiterentwicklung mittelfristig eventuell dem Projekt, aus dem Sie hervor gekommen sind, verdrängen. Auch passiert es, dass Projekte durch den Fortgang von Entwicklern oder veränderte Lebensentwürfe nicht mehr weiterentwickelt werden und mit der Zeit Obsolet werden.

\coleng

TBD

\colger

Dieses Dokument stellt die Flugcontroller-Firmwares Betaflight, INAV und ArduPilot vor.

\colende

\renewcommand{\deutschertitel}{Betaflight}
\renewcommand{\englischertitel}{Betaflight}
\makrounterabschnitt
\label{AbschnittBetaflight}

TBD

\colger

Betaflight ist eine Firmware für Flugcontroller von FPV-Drohnen. Die Installation und Konfiguration der Firmware geschieht bis Betaflight 4.5 mit der lokal installierten Software Betaflight Configurator. Neue Versionen verwenden hierfür die Webanwendung \url{app.betaflight.com}.

\coleng

TBD

\colger

Der Fokus bei der Entwicklung von Betaflight ist die Unterstützung des manuellen Fliegens (Freestyle, Racing, etc.). Ziele von Betaflight sind schnelle Reaktion auf Benutzereingaben und umfangreiche Einstellungsmöglichkeiten um das Flugerhalten zu optimieren (tuning). 

Einstellungsmöglichkeiten Autonomes Fliegen unterstützt Betaflight -- ganz im  Gegensatz zu den alternativem Firmwares INAV oder ArduPilot -- fast gar nicht. Ausnahmen sind der Rettungsmodus (\textsl{Rescue Mode}), der die Drohne im Fall von Verbindungsproblemen mit Hilfe eines GPS-Moduls automatisch zurückzukehren lässt, und ein experimenteller Modus um Höhe und Position automatisch zu halten. 
 
\coleng

TBD

\colger

Betaflight ist kompatibel zu praktisch allen am Markt verfügbaren Flugcontrollern und Empfängern und bietet eine intuitive Benutzeroberfläche. 

\colende

\renewcommand{\deutschertitel}{INAV}
\renewcommand{\englischertitel}{INAV}
\makrounterabschnitt
\label{AbschnittINAV}

TBD

\colger

TBD

\colende

\renewcommand{\deutschertitel}{ArduPilot}
\renewcommand{\englischertitel}{ArduPilot}
\makrounterabschnitt
\label{AbschnittArduPilot}

TBD

\colger

ArduPilot legt als Flugcontroller-Firmware den Fokus auf vollautonomen Flug und komplexe Missionsprofile. Es ermöglicht Missionsplanung mit Wegpunkten, automatische Starts und automatisches Landen, Hindernisvermeidung, Follow-Me, etc. Hierfür nutzt ArduPilot nicht nur die üblichen Sensoren wie GPS, magnetischer Kompass, Geschwindigkeit, sondern kann auch Lidar-Sensoren zur Entfernungen und Kameras einbeziehen.

\coleng

TBD

\colger

Von den in diesem Dokument vorgestellten Flugcontroller-Firmwares braucht ArduPilot am meisten Speicher. Idealerweise sind zum Betrieb von ArduPilot 2\,MB Flash-Speicher verfügbar. Das bieten nur Flugcontroller mit einem H743 Mikrocontroller. Mit 1\,MB Flash-Speicher (F405 und F745 Mikrocontroller) ist der Betrieb mit einem reduzierten Funktionsumfang möglich. Mit nur 512\,kB (F411 und F722 Mikrocontroller) kann ArduPilot gar nicht verwendet werden.

\colende

\renewcommand{\deutschertitel}{Fernbedienung Firmware}
\renewcommand{\englischertitel}{Remote Control Firmware}
\makroabschnitt
\label{AbschnittFirmwareRemoteControl}

TBD

\colger

TBD

\colende

\renewcommand{\deutschertitel}{EdgeTX}
\renewcommand{\englischertitel}{EdgeTX}
\makrounterabschnitt
\label{AbschnittEdgeTX}

TBD

\colger

TBD

\colende

\renewcommand{\deutschertitel}{Sendemodul- und Empfänger-Firmware}
\renewcommand{\englischertitel}{Transmitter and Receiver Firmware}
\makroabschnitt
\label{AbschnittFirmwareReceiverTransmitter}

TBD

\colger

TBD

\colende

\renewcommand{\deutschertitel}{ExpressLRS}
\renewcommand{\englischertitel}{ExpressLRS}
\makrounterabschnitt
\label{AbschnittExpressLRS}

TBD

\colger

TBD

\colende

