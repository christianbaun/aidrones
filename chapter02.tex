\renewcommand{\deutschertitel}{Software}
\renewcommand{\englischertitel}{Software}

% !!! Das hier war vorher !!!
% \chapter[\englischertitel]{\englischertitel\newline\deutschertitel}
% !!! Das hier war vorher !!!


% Das vspace fügt im Inhaltsverzeichnis einen kleinen Abstand unter dem Kapiteleintrag ein.
% Beim deutschen Inhaltsverzeichnis ist es im book.tex an nur einer Stelle in der Zeile 
% \addcontentsline{deutschestoc}{chapter}{\protect{\vspace{2pt}\thechapter}~#1}}
\chapter[\protect{\vspace{2pt}\englischertitel}]{}
\kapitel{\deutschertitel}

\label{KapitelSoftware}

\begin{paracol}{2}[]

{\raggedright\huge\bfseries\sffamily \englischertitel \par\ } \\[1.8ex]

\switchcolumn

{\raggedright\huge\bfseries\sffamily \deutschertitel \par\ } \\[1.8ex]

\switchcolumn*
\selectlanguage{english}

TBD

\switchcolumn
\selectlanguage{ngerman}

TBD

\coleng

TBD

\colger

TBD
  
\colend

\renewcommand{\deutschertitel}{Flight Controller Firmware}
\renewcommand{\englischertitel}{Flight Controller Firmware}
\makroabschnitt
\label{AbschnittFirmwareFC}

TBD

\switchcolumn
\selectlanguage{ngerman}

TBD

\switchcolumn*
\selectlanguage{english}

TBD

\switchcolumn
\selectlanguage{ngerman}

TBD

\colend
