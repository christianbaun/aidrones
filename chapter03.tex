\renewcommand{\deutschertitel}{KI-Anwendungen}
\renewcommand{\englischertitel}{AI Applications}

% !!! Das hier war vorher !!!
% \chapter[\englischertitel]{\englischertitel\newline\deutschertitel}
% !!! Das hier war vorher !!!


% Das vspace fügt im Inhaltsverzeichnis einen kleinen Abstand unter dem Kapiteleintrag ein.
% Beim deutschen Inhaltsverzeichnis ist es im book.tex an nur einer Stelle in der Zeile 
% \addcontentsline{deutschestoc}{chapter}{\protect{\vspace{2pt}\thechapter}~#1}}
\chapter[\protect{\vspace{2pt}\englischertitel}]{}
\kapitel{\deutschertitel}

\label{KapitelKI}

\begin{paracol}{2}[]

{\raggedright\huge\bfseries\sffamily \englischertitel \par\ } \\[1.8ex]

\switchcolumn

{\raggedright\huge\bfseries\sffamily \deutschertitel \par\ } \\[1.8ex]

\coleng

TBD

\colger

TBD

\colende

\renewcommand{\deutschertitel}{Objekterkennung}
\renewcommand{\englischertitel}{Object Detection}
\makroabschnitt
\label{AbschnittObjekterkennung}

TBD

\colger

TBD

\colende


\renewcommand{\deutschertitel}{Autopilot}
\renewcommand{\englischertitel}{Auto Pilot}
\makroabschnitt
\label{AbschnittAutopilot}

TBD

\colger

TBD

\colende



\renewcommand{\deutschertitel}{Autopilot}
\renewcommand{\englischertitel}{Auto Pilot}
\makroabschnitt
\label{AbschnittAutopilot}

TBD

\colger

TBD

\colende
