\renewcommand{\deutschertitel}{Wichtige Punkte vor dem ersten Flug}
\renewcommand{\englischertitel}{Important Things before the first Flight}
\chapter[\protect{\vspace{2pt}\englischertitel}]{}
\kapitel{\deutschertitel}

\label{KapitelWichtigePunkte}

\begin{paracol}{2}[]

{\raggedright\huge\bfseries\sffamily \englischertitel \par\ } \\[1.8ex]

\switchcolumn

{\raggedright\huge\bfseries\sffamily \deutschertitel \par\ } \\[1.8ex]

\coleng

This chapter provides useful advice for both the preparations before the first flight and the checks afterwards.

\colger

Dieses Kapitel enthält hilfreiche Hinweise für die Vorbereitung vor dem ersten Flug sowie für die Überprüfungen danach.

\colende

\renewcommand{\deutschertitel}{Sicherheit beim Flug}
\renewcommand{\englischertitel}{Security during Flight}
\makroabschnitt
\label{AbschnittSicherheitBeimFlug}

Always keep as much distance as possible from other people during flight. Do not fly over other people's buildings or private property without prior permission.

\colger

Während des Fluges ist stets ein möglichst großer Abstand zu anderen Personen einzuhalten. Fremde Gebäude und private Grundstücke dürfen nicht ohne vorherige Zustimmung überflogen werden.

\colende

\renewcommand{\deutschertitel}{Sicherheit vor dem Flug}
\renewcommand{\englischertitel}{Security before Flight}
\makroabschnitt
\label{AbschnittSicherheitVorFlug}

Before each flight, the arming procedure takes place. This means activating the motors in preparation for flight. Although the motors run at a relatively low speed during this process, even at this speed injuries to arms or hands can easily occur. To prevent accidental arming, it is advisable to configure the flight controller so that two switches—ideally not positioned directly next to each other—must be activated simultaneously. A configuration in Betaflight that ensures exactly this is shown in Figure~\ref{AbbildungBetaflightAppModes}.

\colger

Vor dem Flug findet das Arming statt. Dabei handelt es sich um das Aktivieren der Motoren zur Flugvorbereitung. Die Motoren laufen dabei zwar nur mit relativ niedriger Geschwindigkeit, aber selbst bei dieser Drehzahl kann es leicht zu Verletzungen, beispielsweise an Armen und Händen, kommen. Um ein unbeabsichtigtes Arming zu vermeiden, ist es ratsam, den Flugcontroller so zu konfigurieren, dass zum Arming zwei Schalter betätigt werden müssen, die idealerweise nicht direkt nebeneinander liegen. Eine Konfiguration in Betaflight, die genau dies sicherstellt, zeigt Abbildung~\ref{AbbildungBetaflightAppModes}.

\colende

\renewcommand{\deutschertitel}{Sichere Verwendung von LiPo-Akkus}
\renewcommand{\englischertitel}{Secure use of LiPo Batteries}
\makroabschnitt
\label{AbschnittSicherheitLiPoAkkus}

LiPo batteries should never be charged unattended, and they must not be charged while sleeping. Damaged batteries must no longer be used, especially if they have become swollen due to deep discharge, overcharging, or physical damage. Swollen LiPo batteries pose a significant fire hazard. A swollen battery cannot be repaired and must be properly disposed of. Disposal must not be done through household waste but through designated recycling or collection points provided by the local municipality.

\colger

LiPo-Akkus dürfen niemals unbeaufsichtigt geladen werden, insbesondere nicht während man schläft. Beschädigte Akkus dürfen nicht weiterverwendet werden, insbesondere wenn sie durch Tiefentladung, Überladung oder physische Einwirkungen aufgebläht sind. Aufgeblähte LiPo-Akkus stellen ein erhebliches Brandrisiko dar. Ein solcher Akku kann nicht repariert werden und muss ordnungsgemäß entsorgt werden. Die Entsorgung darf nicht über den Hausmüll erfolgen, sondern über die entsprechenden Sammelstellen der örtlichen Gemeinde.

\colende

\renewcommand{\deutschertitel}{Laden von Akkus}
\renewcommand{\englischertitel}{Loading of Batteries}
\makroabschnitt
\label{AbschnittLadenAkkus}

Batteries (LiPo or Li-Ion) with more than one cell must always be charged in balance mode using the voltage values read via the balancer connector (see Figure~\ref{AbbildungLiIonAkkuLaden}). This prevents individual cells from being overcharged during the charging process and helps to avoid deep discharge of individual cells during later use. The maximum recommended charge rate (see Section~\ref{AbschnittCWertAkkus}) is typically 1\,C. If this limit is exceeded, the cells will age more quickly, and in the worst case, a fire may occur. For safety reasons, Li-Ion batteries should only be stored in specially designed fireproof containers.

\colger

Akkus (LiPo oder Li-Ion) mit mehr als einer Zelle müssen immer unter Verwendung der über den Balancerstecker ausgelesenen Spannungswerte im Balancermodus geladen werden (siehe Abbildung~\ref{AbbildungLiIonAkkuLaden}). Dies verhindert eine Überladung einzelner Zellen während des Ladevorgangs und beugt einer Tiefentladung einzelner Zellen im späteren Betrieb vor. Die maximal empfohlene Laderate (siehe Abschnitt~\ref{AbschnittCWertAkkus}) beträgt üblicherweise 1\,C. Wird diese Rate überschritten, altern die Zellen schneller, und es kann im schlimmsten Fall zu einem Brand kommen. Zur Sicherheit sollten Li-Ion-Akkus nur in dafür vorgesehenen, feuerfesten Behältnissen gelagert werden.

\colende

\begin{figure}[htb!]
  \centering
    \includegraphics[width=\linewidth]{LiIo-Akku-laden.png}
  \caption{Balanced Loading of a Li-Ion Battery}
  \label{AbbildungLiIonAkkuLaden}
\end{figure}

\renewcommand{\deutschertitel}{Sichere Befestigung loser Kabel und sonstiger Komponenten}
\renewcommand{\englischertitel}{Secure Attachment of loose Cables and other Components}
\makroabschnitt
\label{AbschnittBefestigungLoseKabel}

If cables or other drone components are not properly secured and can reach the propellers, they eventually will. This can damage either the loose components or the propellers, potentially leading to a crash or costly damage. A common example of components that frequently and unintentionally come into contact with the propellers are the battery balancer connectors (see Figure~\ref{AbbildungBalancerConnectorDestroyed}) , which are not always easy to replace. Therefore, all components must be securely fastened, and cables must be fixed using screws, cable ties, Velcro straps, or at least electrical tape.

\colger

Wenn Kabel oder andere Bauteile der Drohne nicht zuverlässig befestigt sind und die Rotoren erreichen können, werden sie diese früher oder später tatsächlich erreichen. Dabei werden entweder die losen Komponenten oder die Rotoren beschädigt, was zu einem Absturz oder zu teuren Beschädigungen führen kann. Ein typisches Beispiel für Bauteile, die häufig ungewollt mit den Rotoren in Kontakt kommen, sind die Balancerstecker (siehe Abbildung~\ref{AbbildungBalancerConnectorDestroyed}) der Akkus, deren Austausch nicht immer einfach ist. Daher ist die Befestigung aller Bauteile und die Sicherung aller Kabel mit Schrauben, Kabelbindern, Klettbändern oder zumindest Isolierband zwingend erforderlich.

\colende

\begin{figure}[htb!]
  \centering
    \includegraphics[width=\linewidth]{LiPo_Akku_Schaden_durch_Propeller_crop.png}
  \caption{A Balancer Connector got destroyed because it touched a Propeller during Flight}
  \label{AbbildungBalancerConnectorDestroyed}
\end{figure}

\renewcommand{\deutschertitel}{Den geeigneten Flugmodus verwenden}
\renewcommand{\englischertitel}{Use the Appropriate Flight Mode}
\makroabschnitt
\label{AbschnittFlugmodusEinstellen}

There are three flight modes: Angle Mode (stabilized), Horizon Mode (semi-stabilized), and Acro Mode (manual).

\colger

Es gibt die Flugmodi Angle Mode (stabilisiert), Horizon Mode (halb-stabilisiert) und Acro Mode (manuell).

\coleng

In Angle Mode, the drone stabilizes itself and remains level without explicit control input from the pilot. This flight mode is the most beginner-friendly and ideal for indoor flights and smooth, slow cinematic shots. Because the maximum tilt angle is limited, spectacular flight maneuvers such as flips or rolls are not possible.

\colger

Im Angle Mode stabilisiert sich die Drohne selbst und hält sich ohne explizite Steueranweisungen des Piloten automatisch waagrecht. Dieser Flugmodus ist besonders anfängerfreundlich und ideal für Indoor-Flüge sowie für langsame, ruhige Filmaufnahmen. Da die maximale Neigung (Tilt) begrenzt ist, sind spektakuläre Flugmanöver wie Flips oder Rollen nicht möglich.
 
\coleng

In Horizon Mode, flips and rolls are possible, but the drone automatically stabilizes itself when no control input is applied.

\colger

Im Horizon Mode sind Flips und Rollen möglich, jedoch stabilisiert sich die Drohne automatisch, sobald keine Steuerkommandos mehr gegeben werden.

\coleng

In Acro Mode, which is actually the default mode, there is no automatic stabilization. The drone maintains its current rotational speed until the pilot counteracts it. This mode provides full control on all axes and allows spectacular flight maneuvers at any time. It is the least beginner-friendly mode but ideal for freestyle flying and racing.

\colger

Im Acro Mode, der eigentlich der Standardmodus ist, erfolgt keine automatische Stabilisierung. Die Drohne hält stets die aktuelle Drehgeschwindigkeit bei, bis der Pilot aktiv gegensteuert. In diesem Modus hat man die volle Kontrolle über alle Achsen, und spektakuläre Flugmanöver sind jederzeit möglich. Dieser Flugmodus ist am wenigsten anfängerfreundlich und ideal für Freestyle-Flüge und Rennen.

\coleng

Since a drone is set to Acro Mode by default, it is advisable to configure a three-position switch on the transmitter for selecting flight modes so that Angle Mode can be activated automatically when desired. A configuration in Betaflight that enables this is shown in Figure~\ref{AbbildungBetaflightAppModes}.

\colger

Da Drohnen standardmäßig im Acro Mode betrieben werden, ist es sinnvoll, einen Schalter des Senders mit drei möglichen Positionen zur Auswahl des Flugmodus so zu konfigurieren, dass bei Bedarf automatisch der Angle Mode aktiviert werden kann. Eine entsprechende Konfiguration in Betaflight, die dies ermöglicht, zeigt Abbildung~\ref{AbbildungBetaflightAppModes}.

\colende

\renewcommand{\deutschertitel}{Die Leistung des Videosenders reduzieren oder diesen abschalten}
\renewcommand{\englischertitel}{Reduce the Power of the Video Transmitter or Switch It Off}
\makroabschnitt
\label{AbschnittVTXreduzieren}

It is highly recommended to switch off the VTX or significantly reduce its output power during ground or laboratory tests, as it can overheat within a few minutes without the airflow present during flight. Even if it does not overheat, it will still become very hot quickly and draw considerable current from the batteries.

\colger

Es ist sehr empfehlenswert, bei Boden- oder Labortests den VTX abzuschalten oder seine Leistung stark zu reduzieren, da er ohne die Luftkühlung während des Flugs innerhalb weniger Minuten überhitzen kann. Selbst wenn er nicht überhitzt, wird er sehr schnell heiß und zieht viel Strom aus den Akkus.

\coleng

In Betaflight, the VTX can be completely switched off via the CLI:

\colger

In Betaflight kann man den VTX über die CLI komplett ausschalten:

\colende

\lstdefinestyle{shell}{
  backgroundcolor=\color{gray!10},
  basicstyle=\ttfamily\small,
  frame=single,
  breaklines=true,        % <-- Zeilenumbruch aktivieren
  breakatwhitespace=true, % <-- nur bei Leerzeichen umbrechen 
  postbreak=\mbox{\textcolor{gray}{$\hookrightarrow$}\space}, % Pfeil am Zeilenende
  showstringspaces=false,
  xleftmargin=0em,
  xrightmargin=0em,
  framerule=0.5pt,
  rulecolor=\color{gray!60}
}

\begin{lstlisting}[style=shell]
vtx_power = 0
save
\end{lstlisting}

\colstart

These commands switch the VTX to the reduced 25 mW power mode:

\colger

Diese Kommandos schalten den VTX in den reduzierten 25-mW-Leistungsbetrieb:

\colende

\lstdefinestyle{shell}{
  backgroundcolor=\color{gray!10},
  basicstyle=\ttfamily\small,
  frame=single,
  breaklines=true,        % <-- Zeilenumbruch aktivieren
  breakatwhitespace=true, % <-- nur bei Leerzeichen umbrechen 
  postbreak=\mbox{\textcolor{gray}{$\hookrightarrow$}\space}, % Pfeil am Zeilenende
  showstringspaces=false,
  xleftmargin=0em,
  xrightmargin=0em,
  framerule=0.5pt,
  rulecolor=\color{gray!60}
}

\begin{lstlisting}[style=shell]
vtx_power = 1
save
\end{lstlisting}

\colstart

Another option is to greatly reduce the VTX’s output power by switching it to so-called PIT mode. PIT mode lowers the transmit power to approximately 0.1\,mW -- just enough to receive the signal on the bench or within the same room, while preventing overheating. The following command switches the VTX to PIT mode:

\colger

Eine andere Möglichkeit ist, die Leistung des VTX stark zu reduzieren, indem man ihn in den sogenannten PIT-Mode schaltet. Der PIT-Mode reduziert die Sendeleistung auf ungefähr 0,1\,mW. Das ist gerade genug, um das Signal auf dem Tisch oder im Raum zu sehen und eine Überhitzung auszuschließen. Das folgende Kommando schaltet den VTX in den PIT-Mode:

\colende


\lstdefinestyle{shell}{
  backgroundcolor=\color{gray!10},
  basicstyle=\ttfamily\small,
  frame=single,
  breaklines=true,        % <-- Zeilenumbruch aktivieren
  breakatwhitespace=true, % <-- nur bei Leerzeichen umbrechen 
  postbreak=\mbox{\textcolor{gray}{$\hookrightarrow$}\space}, % Pfeil am Zeilenende
  showstringspaces=false,
  xleftmargin=0em,
  xrightmargin=0em,
  framerule=0.5pt,
  rulecolor=\color{gray!60}
}

\begin{lstlisting}[style=shell]
set vtx_pit_mode = ON
save
\end{lstlisting}

\colstart

To deactivate PIT mode again:

\colger

Um den PIT-Mode wieder zu deaktivieren:

\colende



\lstdefinestyle{shell}{
  backgroundcolor=\color{gray!10},
  basicstyle=\ttfamily\small,
  frame=single,
  breaklines=true,        % <-- Zeilenumbruch aktivieren
  breakatwhitespace=true, % <-- nur bei Leerzeichen umbrechen 
  postbreak=\mbox{\textcolor{gray}{$\hookrightarrow$}\space}, % Pfeil am Zeilenende
  showstringspaces=false,
  xleftmargin=0em,
  xrightmargin=0em,
  framerule=0.5pt,
  rulecolor=\color{gray!60}
}

\begin{lstlisting}[style=shell]
set vtx_pit_mode = OFF
save
\end{lstlisting}
\colstart

The PIT mode can also be enabled and disabled on the \textsl{Video Transmitter} page in the Betaflight Configurator or via the Betaflight web application. It is useful to assign PIT mode to a switch (AUX channel) on the transmitter. This can be done in the Betaflight Configurator or the Betaflight web application on the \textsl{Modes} page using the \textsl{VTX Pit Mode} entry.

\colger

Der PIT-Mode kann auch auf der Seite \textsl{Video Transmitter} im Betaflight Configurator oder über die Betaflight Webanwendung ein- und ausgeschaltet werden. Sinnvoll ist es, dem PIT-Mode einen Schalter (AUX-Kanal) auf der Fernbedienung zuzuweisen. Dies ist im Betaflight Configurator oder in der Betaflight Webanwendung auf der Seite \textsl{Modes} über den Eintrag \textsl{VTX Pit Mode} möglich.

\coleng

Additionally, it is possible to operate the VTX in \textsl{low-power mode} before arming and after disarming -- i.e., during flight preparation and after the flight. The following command defines that the VTX transmission power will only be increased to the configured level after arming. Before arming and after disarming, the transmission power is reduced to the value defined by \verb!vtx_power=1!.

\colger

Zudem ist es möglich, den VTX vor dem Arming und nach dem Disarming im \textsl{Low-Power-Modus} zu betreiben, also während der Flugvorbereitung und nach dem Flug. Das folgende Kommando definiert, dass die Sendeleistung des VTX erst nach dem Arming auf den eingestellten Wert erhöht wird. Vor dem Arming und nach dem Disarming ist die Sendeleistung auf den Wert von \verb!vtx_power=1! reduziert.

\colende

\lstdefinestyle{shell}{
  backgroundcolor=\color{gray!10},
  basicstyle=\ttfamily\small,
  frame=single,
  breaklines=true,        % <-- Zeilenumbruch aktivieren
  breakatwhitespace=true, % <-- nur bei Leerzeichen umbrechen 
  postbreak=\mbox{\textcolor{gray}{$\hookrightarrow$}\space}, % Pfeil am Zeilenende
  showstringspaces=false,
  xleftmargin=0em,
  xrightmargin=0em,
  framerule=0.5pt,
  rulecolor=\color{gray!60}
}

\begin{lstlisting}[style=shell]
set vtx_low_power_disarm = ON
save
\end{lstlisting}

\renewcommand{\deutschertitel}{Mit einem Simulator üben}
\renewcommand{\englischertitel}{Practice with a Simulator}
\makroabschnitt
\label{AbschnittSimulator}

Flying a drone is not easy, and crashes are almost inevitable at the beginning, even in Angle Mode. Therefore, it is highly advisable to practice with a simulator. Suitable simulators are available for all operating systems. Some well-known examples include Liftoff, VelociDrone, Uncrashed, FPV Freerider, Quadsim FPV, and Freerider Lite. 

\colger

Das Fliegen einer Drohne ist nicht einfach, und Abstürze sind insbesondere zu Beginn selbst im Angle Mode kaum zu vermeiden. Daher ist es sehr empfehlenswert, mit einem Simulator zu üben. Geeignete Simulatoren sind für alle Betriebssysteme verfügbar. Einige bekannte Produkte sind Liftoff, VelociDrone, Uncrashed, FPV Freerider, Quadsim FPV und Freerider Lite. 

\coleng

The transmitter has a USB interface and should be used as the input device for training. The USB mode must be set to \textsl{Joystick}. It is recommended to create a new model in OpenTX for this purpose, for example named \textsl{Simulator}. This model should have no active internal or external RF module, since the module consumes a significant amount of power and produces heat. This is unnecessary when using a simulator because no radio signals are transmitted.

\colger

Der Sender verfügt über eine USB-Schnittstelle und sollte als Eingabegerät zum Üben verwendet werden. Als USB-Modus muss \textsl{Joystick} ausgewählt werden. Es ist sinnvoll, hierfür in OpenTX ein neues Modell anzulegen, beispielsweise mit dem Namen \textsl{Simulator}. Dieses Modell sollte über kein aktives internes oder externes Sendemodul verfügen, da das Sendemodul signifikant Strom verbraucht und Wärme erzeugt. Dies ist beim Einsatz mit einem Simulator unnötig, da keine Funksignale gesendet werden.

\colende

\renewcommand{\deutschertitel}{Die Kopplung automatisieren}
\renewcommand{\englischertitel}{Automate the Pairing}
\makroabschnitt
\label{AbschnittBindingPhrase1}

To speed up the pairing process between transmitter and receiver and to prevent accidental pairing with other devices, a binding phrase can be defined in the ExpressLRS firmware (see Section~\ref{AbschnittExpressLRS}). This phrase must be identical on both communication partners and cannot be changed without reinstalling (flashing) the firmware.

\colger

Um die Kopplung zwischen Sender und Empfänger zu beschleunigen und ein versehentliches Koppeln mit anderen Geräten zu verhindern, kann in der ExpressLRS-Firmware eine Binding Phrase definiert werden (siehe Abschnitt~\ref{AbschnittExpressLRS}). Diese muss bei beiden Kommunikationspartnern identisch sein und kann nur durch eine Neuinstallation (Flashen) der Firmware geändert werden.

\colende


\renewcommand{\deutschertitel}{Überwachung der Telemetriedaten mit geeigneten Werkzeugen}
\renewcommand{\englischertitel}{Monitoring telemetry data with suitable tools}
\makroabschnitt
\label{AbschnittTelemetriedatenFM2MToolBox}

For convenient real time monitoring of telemetry data various tools are available. One example is the FM2M ToolBox (Fly Me 2 the Moon) by Robert Janiszewski which can be installed on radio transmitters running EdgeTX. The software aggregates sensor data that is relevant for flight preparation and for monitoring the aircraft during flight. This includes flight mode GPS position satellite count speed altitude battery voltage current draw signal quality and additional sensor data such as barometer and magnetic compass readings. Real time monitoring of telemetry data increases flight safety and enables error detection and analysis.

\colger

Zur komfortablen Überwachung der Telemetriedaten in Echtzeit existieren verschiedene Werkzeuge. Ein Beispiel ist die FM2M ToolBox (Fly Me 2 the Moon) von Robert Janiszewski die auf Fernbedienungen mit EdgeTX installiert werden kann. Die Software ermöglicht die Aggregation der Sensordaten die für die Flugvorbereitung und die Überwachung während des Fluges relevant sind. Dazu gehören Flugmodus GPS Position Anzahl der Satelliten Geschwindigkeit Höhe Batteriespannung Stromaufnahme Signalqualität sowie zusätzliche Sensordaten wie Barometer und magnetischer Kompass. Die Echtzeitüberwachung der Telemetriedaten erhöht die Flugsicherheit und ermöglicht Fehlererkennung und Fehleranalyse.

\coleng

To use FM2M or similar scripts the flight controller must provide telemetry data via serial interfaces such as UART and must run compatible firmware. FM2M supports the Betaflight and iNAV firmwares and is available for radio transmitters with color or monochrome displays.

\colger

Voraussetzung für die Nutzung von FM2M oder vergleichbarer Skripte ist dass der Flugcontroller Telemetriedaten über serielle Schnittstellen wie UART bereitstellt und kompatible Firmware verwendet. FM2M unterstützt die Firmwares Betaflight und iNAV und ist für Fernbedienungen mit Farb und Schwarz Weiß Display erhältlich.

\colende

