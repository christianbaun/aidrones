\renewcommand{\deutschertitel}{Wichtige Punkte vor dem ersten Flug}
\renewcommand{\englischertitel}{Important Things before the first Flight}
\chapter[\protect{\vspace{2pt}\englischertitel}]{}
\kapitel{\deutschertitel}

\label{KapitelWichtigePunkte}

\begin{paracol}{2}[]

{\raggedright\huge\bfseries\sffamily \englischertitel \par\ } \\[1.8ex]

\switchcolumn

{\raggedright\huge\bfseries\sffamily \deutschertitel \par\ } \\[1.8ex]

\coleng

TBD

\colger

Dieses Kapitel enthält hilfreiche Hinweise vor dem ersten Flug und danach.

\colende

\renewcommand{\deutschertitel}{Sicherheit beim Flug}
\renewcommand{\englischertitel}{Security during Flight}
\makroabschnitt
\label{AbschnittSicherheitBeimFlug}

TBD

\colger

Beim Flug auf jeden Fall einen möglichst großen Abstand zu anderen Personen einhalten. Auch fremde Gebäude und private Flächen dürfen nicht ohne vorherige Absprachen überflogen werden. 

\colende

\renewcommand{\deutschertitel}{Sicherheit vor dem Flug}
\renewcommand{\englischertitel}{Security before Flight}
\makroabschnitt
\label{AbschnittSicherheitVorFlug}

TBD

\colger

Vor dem Flug findet das Arming statt. Dabei handelt es ich um das Aktivieren der Motoren zur Flugvorbereitung. Die Motoren laufen dabei zwar nur mit relativ niedriger Geschwindigkeit, aber auch bei dieser Geschwindigkeit kann es leicht zu Verletzungen z.B. an Armen und Händen kommen. Um ein versehentliches Arming auszuschließen, ist es sinnvoll, den Flugcontroller dahingehend zu konfigurieren, dass zum Arming zwei Schalter, die idealerweise auch nicht direkt nebeneinander liegen, betätigt werden müssen. Eine Konfiguration in Betaflight, die genau dieses garantiert, zeigt  Abbildung~\ref{AbbildungBetaflightAppModes}.

\colende


\renewcommand{\deutschertitel}{Sichere Verwendung von LiPo-Akkus}
\renewcommand{\englischertitel}{Secure use of LiPo Batteries}
\makroabschnitt
\label{AbschnittSicherheitLiPoAkkus}

TBD

\colger

LiPo-Akkus dürfen nie alleine geladen werden. Sie dürfen auch nicht geladen werden während man schläft. Beschädigte Akkus dürfen nicht mehr verwendet werden. Das gilt besonders wenn Sie durch Tiefentladung, Überladung oder physische Schäden aufgebläht sind. Aufgeblähte LiPo-Akkus haben ein enorm hohes Brandrisiko. Ein aufgeblähter Akku kann nicht repariert werden und muss entsorgt werden. Die Entsorgung darf nicht über den Hausmüll geschehen sondern über entsprechende Sammelstellen der lokalen Gemeinde. 

\colende


\renewcommand{\deutschertitel}{Laden von Akkus}
\renewcommand{\englischertitel}{Loading of Batteries}
\makroabschnitt
\label{AbschnittLadenAkkus}

TBD

\colger

Akkus (LiPo oder Li-Ion) mit mehr als einer Zelle müssen immer mithilfe der über den Balancerstecker ausgelesenen Werte balancierend geladen werden (siehe Abbildung~\ref{AbbildungLiIonAkkuLaden}). Dies verhindert die Überladung einzelner Zellen während des Ladevorgangs und beugt einer Tiefentladung einzelner Zellen beim späteren Betrieb vor. Die maximale empfohlene Laderate (siehe Abschnitt~\ref{AbschnittCWertAkkus}) ist üblicherweise 1\,C. Wird die maximale empfohlene Laderate altern die Zellen schneller und es kann im schlimmsten Fall zum Brand kommen. Die Lagerung von Li-Ion-Akkus sollte zur Sicherheit nur in dafür vorgesehenen feuerfesten Behältnissen erfolgen. 

\colende

\begin{figure}[htb!]
  \centering
    \includegraphics[width=\linewidth]{LiIo-Akku-laden.png}
  \caption{Balanced Loading of a Li-Ion Battery}
  \label{AbbildungLiIonAkkuLaden}
\end{figure}


\renewcommand{\deutschertitel}{Sichere Befestigung loser Kabel und sonstiger Komponenten}
\renewcommand{\englischertitel}{Secure Attachment of loose Cables and other Components}
\makroabschnitt
\label{AbschnittBefestigungLoseKabel}

TBD

\colger

Wenn Kabel oder sonstige Bauteile der Drohne nicht zuverlässig befestigt sind und Rotoren erreichen können, dann werden sie auch früher oder später die laufenden Rotoren erreichen. Dabei werden entweder die losen Komponenten oder die Rotoren beschädigt und es kommt unter Umständen zum Absturz oder zu ärgerlichen Beschädigungen. Ein typisches Beispiel für Komponenten, die häufig ungewollt mit Rotoren in Berührung kommen sind die Balancerstecker der Akkus. Diese Ersetzen ist nicht immer einfach. Die Befestigung aller Bauteile und Sicherung von Kabeln mit Schrauben, Kabelbindern, Klettbändern oder zumindest Isolierband ist daher zwingend erforderlich. 

\colende

\renewcommand{\deutschertitel}{Den geeigneten Flugmodus verwenden}
\renewcommand{\englischertitel}{Use the appropriate flight mode}
\makroabschnitt
\label{AbschnittFlugmodusEinstellen}

TBD

\colger

Es gibt die Flugmodi Angle Mode (stabilisiert), Horizon Mode (halb-stabilisiert) und Acro Mode (manual). 

\coleng

TBD

\colger

Beim Angle Mode stabilisiert sich die Drohne selbst. Sie hält sich ohne explizite Steueranweisungen durch den Benutzer automatisch waagrecht. Dieser Flugmodus ist am meisten anfängerfreundlich und ideal für Indoor-Flüge und langsames Filmen. Da die maximale Neigung (Tilt) begrenzt ist, sind spektakuläre Flugmanöver wie Flips oder Rollen unmöglich. 
 
\coleng

TBD

\colger

Beim Horizon Mode sind Flips oder Rollen möglich, aber die Drohne stabilisiert er sich  automatisch, wenn keine Steuerkommandos vorliegen.

\coleng

TBD

\colger

Beim Acro Mode, der eigentlich der normale Modus ist, erfolgt keine automatische Stabilisierung. Die Drohne hält stets die aktuelle Drehgeschwindigkeit bis der Benutzer gegensteuert. Auf allen Achsen gibt es volle Kontrolle und spektakuläre Flugmanöver sind jederzeit möglich. Dieser Flugmodus ist am wenigsten anfängerfreundlich und ideal für Freestyle und Rennen.

\coleng

TBD

\colger

Da standardmäßig eine Drohne im Acro Mode ist, ist es sinnvoll einen Schalter des Senders mit drei möglichen Positionen zur Auswahl des Flugmodus dahingehend zu konfigurieren, dass automatisch Angle Mode verwendet wird. Eine Konfiguration in Betaflight, die dieses ermöglicht, zeigt Abbildung~\ref{AbbildungBetaflightAppModes}.

\colende

\renewcommand{\deutschertitel}{Mit einem Simulator üben}
\renewcommand{\englischertitel}{Training with a Simulator}
\makroabschnitt
\label{AbschnittSimulator}

TBD

\colger

Drohnen fliegen ist nicht einfach und Abstütze gerade zu Beginn auch mit dem Flugmodus Angle Mode unvermeidlich. Es ist daher sinnvoll mit einem Simulator zu üben. Geeignete Simulatoren gibt es für alle Betriebssysteme. Einige bekannte Produkte sind Liftoff, VelociDrone, Uncrashed, FPV Freerider, Quadsim FPV und Freerider Lite.

\coleng

TBD

\colger

Der Sender verfügt über eine USB-Schnittstelle und sollte als Eingabegerät zum Üben verwendet werden. Als USB-Modus muss \textsl{Joystick} ausgewählt werden. Es ist sinnvoll, hierfür in OpenTX ein neues Modell anzulegen. Hierfür bietet sich ein Name wie \textsl{Simulator}. Dieses neue Modell sollte über kein aktives internes oder externes Sendemodul verfügen. Hintergrund ist, dass das Sendemodul signifikant Strom verbraucht und Abwärme erzeugt, was beim Verwendung mit einem Simulator unnötig ist, weil hier keine Funksignale gesendet werden.

\colende
